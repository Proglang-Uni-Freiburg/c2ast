\declrel{Transition labels}
\begin{align*}
  \sigma \grmdef&
    !v \grmor ?v \grmor !l \grmor ?l \grmor ! \grmor ?
  && \text{label suffixes for session operations} \\
  \lambda \grmdef&
    a\sigma \grmor \beta \grmor \PScope[a,a] \EFork v
  &&\text{labels for expression reduction} \\
  \pi \grmdef&
               a\sigma \grmor \tau \grmor \PScope[a,a] a!a \RPar \pi
               \pi
  &&\text{labels for process reduction}
\end{align*}

\declrel{Free variables of process labels}[$\FV{p}$]
\begin{mathpar}
  \FV{a!v} = \FV{a?v} = \{a\} \cup \FV{v} \and
  \FV{a!l} = \FV{a?l} = \FV{a!} = \FV{a?} = \{a\} \and
  \FV{\tau} = \emptyset \and
  \FV{\PScope[c,d] a!x} = \{a,x\} \setminus \{c,d\} \and
  \FV{\RPar {\pi_1} {\pi_2}} = \FV{\pi_1} \cup \FV{\pi_2}
\end{mathpar}

\declrel{Labeled transition system for expressions}[$\reduce[\lambda] e e$]
\begin{mathpar}
  \ltsrule \ActApp { (\ELam x e) v } \beta { e[v / x] }

  \ltsrule \ActRec { (\ERec x v) v' } \beta { v[\ERec x v / x] v' }

  \ltsrule \ActLet { \ELetP xy {(v_1,v_2)} e } \beta { e[v_1/x, v_2/y] }

  \ltsrule \ActUnit { \ELetU \EUnit e } \beta { e }

  \ltsrule \ActSend { \ESend v a } {a!v} { a }

  \ltsrule \ActRecv { \ERecv a } {a?v} { (v,a) }

  \ltsrule \ActSelect { \ESelect l a } {a!l} { a }

  \ltsrule \ActCase { \ECase a { l_i \rightarrow v_i } } {a?l_j} { v_{\!j} \, a }

  \ltsrule \ActWait { \EWait a } {a?} { \EUnit }

  \ltsrule \ActTerm { \ETerm a } {a!} { \EUnit }

  \ltsrule \ActFork { \EFork v } { \PScope \EFork v } { a }
\end{mathpar}
The standard structural call-by-value rules are omitted.

\declrel{Labeled transition system for processes}[$\reduce[\pi] p p$]
\begin{mathpar}
  \ltsrule \ActSession [
    \reduce[\sigma]{ e }{ e' }
  ]{ \PExp e } \sigma { \PExp {e'} }

  \ltsrule \ActBeta [
    \reduce[\beta]{ e }{ e' }
  ]{ \PExp e } \tau { \PExp {e'} }

  \ltsrule \ActForkP [
    \reduce[ \PScope \EFork v ]{ e }{ e' }
  ]{ \PExp e } \tau { \PScope (\PPar {\PExp{e'}} {\PExp{v\;b}}) }

  \ltsrule \ActJoin [
    \reduce[\pi_1]{p}{p'} \\
    \reduce[\pi_2]{q}{q'} \\
  ]{ \PPar p q } {\RPar{\pi_1\!}{\pi_2}} { \PPar {p'} {q'} }

  \ltsrule \ActMsg [
    \reduce[\RPar{a!v}{b?v}]{p}{p'}
  ]{ \PScope p } \tau { \PScope p' }

  \ltsrule \ActBranch [
    \reduce[\RPar{a!l}{b?l}]{p}{p'}
  ]{ \PScope p } \tau { \PScope p' }

  \ltsrule \ActEnd [
    \reduce[\RPar{a!}{b?}]{p}{p'}
  ]{ \PScope p } \tau { p' }

  \ltsrule \ActScope [
    \reduce[\pi]{p}{p'} \\
    a,b \not\in \FV{\pi}
  ]{ \PScope p } \pi { \PScope p' }

  \ltsrule \ActPar [
    \reduce[\pi]{p}{p'}
  ]{ \PPar p q } \pi { \PPar {p'} q }

 \ltsrule \ActOpen [
   \reduce[a!c]{p}{p'} \\
   a \not\in \{c,d\}
 ]{ \PScope[c,d] p } { \PScope[c,d] a!c } { p' }

 \ltsrule \ActClose [
   \reduce[\RPar{\PScope[c,d] a!c}{b?c}]{p}{p'}
 ]{ \PScope[a,b] p } \tau { \PScope[a,b] \PScope[c,d] p' }

  \ltsrule \ActCong [
    p \equiv q \\
    \reduce[\pi]{q}{q'}
  ]{ p } \pi { q' }
\end{mathpar}
\js{I'm unclear about why/if \textsc{\ActOpen} and \textsc{\ActClose} are needed.}

%%% Local Variables:
%%% mode: latex
%%% TeX-master: "main"
%%% End:
