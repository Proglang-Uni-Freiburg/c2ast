\declrel{Transition labels}
\begin{align*}
  \sigma \grmdef&
    !v \grmor ?v \grmor !l \grmor ?l \grmor ! \grmor ?
  && \text{label suffixes for session operations} \\
  \lambda \grmdef&
    a\sigma \grmor \beta \grmor  \LLet \grmor \PScope \EFork v
  &&\text{labels for expression reduction} \\
  \pi \grmdef&
               a\sigma
               \grmor \tau
               % \grmor \PScope c!a
               \grmor \RPar \pi \pi
  &&\text{labels for process reduction}
\end{align*}

\declrel{Duality for label suffixes}[$\TDual\sigma$]
\begin{align*}
  \TDual{!v} &= ?v
  &\TDual{ ?v} &= !v
  &\TDual{ !l} &= ?l
  &\TDual{ ?l} &= !l
  & \TDual ! &= ?
  &\TDual ? &= !
\end{align*}

\declrel{Free variables of process labels}[$\FV{p}$]
\begin{mathpar}
  \FV{a!v} = \FV{a?v} = \{a\} \cup \FV{v} \and
  \FV{a!l} = \FV{a?l} = \FV{a!} = \FV{a?} = \{a\} \and
  \FV{\tau} = \emptyset \and
  \FV{\PScope[c,d] a!x} = \{a,x\} \setminus \{c,d\} \and
  \FV{\RPar {\pi_1} {\pi_2}} = \FV{\pi_1} \cup \FV{\pi_2}
\end{mathpar}

\declrel{Labeled transition system for expressions (where $n$ ranges over
  non-value expressions)}[$\reduce[\lambda] e e$]
\begin{mathpar}
  \ltsrule \ActAppLetL {\EApp n e} \LLet {\ELet x n \EApp x e}

  \ltsrule \ActAppLetR {\EApp v n} \LLet {\ELet x n \EApp v x}

  \ltsrule \ActPairLetL {\EPair n e} \LLet {\ELet x n \EPair x e}

  \ltsrule \ActPairLetR {\EPair v n} \LLet {\ELet x n \EPair v x}

  \ltsrule \ActLetLet {\ELet {t_2} {(\ELet {t_1}{e_1}{e_2} )}{e_3}}
  \LLet {\ELet{t_1}{e_1}{(\ELet{t_2}{e_2}{e_3})}}

  \ltsrule \ActLetEta {\ELet t e t} \LLet e

  \ltsrule \ActApp { (\ELam x e) v } \beta { e[v / x] }

  \ltsrule \ActRec { (\ERec x v) v' } \beta { v[\ERec x v / x] v' }

  \ltsrule \ActLetBeta { \ELet x {v} e } \beta { e[v/x] }

  \ltsrule \ActLetPair { \ELetP xy {(v_1,v_2)} e } \beta { e[v_1/x, v_2/y] }

  \ltsrule \ActUnit { \ELetU \EUnit e } \beta { e }

  \ltsrule \ActSend { \ESend v a } {a!v} { a }

  \ltsrule \ActRecv { \ERecv a } {a?v} { (v,a) }

  \ltsrule \ActSelect { \ESelect l a } {a!l} { a }

  \ltsrule \ActCase { \ECase a { l_i \rightarrow v_i } } {a?l_j} { v_{\!j} \, a }

  \ltsrule \ActWait { \EWait a } {a?} { \EUnit }

  \ltsrule \ActTerm { \ETerm a } {a!} { \EUnit }

  \ltsrule \ActFork { \EFork v } { \PScope \EFork v } { a }
\end{mathpar}
The standard structural call-by-value rules are omitted.

\declrel{Labeled transition system for processes}[$\reduce[\pi] p p$]
\begin{mathpar}
  \ltsrule \ActSession [
    \reduce[a\sigma]{ e }{ e' }
  ]{ \PExp e } {a\sigma} { \PExp {e'} }

  \ltsrule \ActBeta [
    \reduce[\beta]{ e }{ e' }
  ]{ \PExp e } \tau { \PExp {e'} }

  \ltsrule \ActLet [
    \reduce[\LLet]{ e }{ e' }
  ]{ \PExp e } \tau { \PExp {e'} }

  \ltsrule \ActForkP [
    \reduce[ \PScope \EFork v ]{ e }{ e' }
  ]{ \PExp e } \tau { \PScope (\PPar {\PExp{e'}} {\PExp{v\;b}}) }

  \ltsrule \ActJoin [
    \reduce[\pi_1]{p}{p'} \\
    \reduce[\pi_2]{q}{q'} \\
  ]{ \PPar p q } {\RPar{\pi_1\!}{\pi_2}} { \PPar {p'} {q'} }

  \ltsrule \ActSync [
    \reduce[\RPar{a\sigma}{b\TDual\sigma}]{p}{p'}
  ]{ \PScope p } \tau { \PScope p' }

  % \ltsrule \ActMsg [
  %   \reduce[\RPar{a!v}{b?v}]{p}{p'}
  % ]{ \PScope p } \tau { \PScope p' }
  % 
  % \ltsrule \ActBranch [
  %   \reduce[\RPar{a!l}{b?l}]{p}{p'}
  % ]{ \PScope p } \tau { \PScope p' }
  % 
  % \ltsrule \ActEnd [
  %   \reduce[\RPar{a!}{b?}]{p}{p'}
  % ]{ \PScope p } \tau { p' }
  % 
  \ltsrule \ActScope [
    \reduce[\pi]{p}{p'} \\
    a,b \not\in \FV{\pi}
  ]{ \PScope p } \pi { \PScope p' }

  \ltsrule \ActPar [
    \reduce[\pi]{p}{p'}
  ]{ \PPar p q } \pi { \PPar {p'} q }

  \ltsrule \ActCong [
    p \equiv q \\
    \reduce[\pi]{q}{q'}
  ]{ p } \pi { q' }
\end{mathpar}

We define multistep labeled reductions in the usual way by
concatenating the labels using $\_\cdot\_$ as the concatenation
operator. We treat the silent label $\tau$ as a 
neutral element in the concatenation, i.e., $\tau \cdot \overline\pi = \overline\pi
\cdot \tau = \overline\pi$, where $\overline\pi$ is a sequence of labels. 
\begin{mathpar}
  \ltsrulemany \ActRefl {p} \tau p

  \ltsrulemany \ActOne[
  \reduce[\pi] p {p'} \\
  \reducemany[\overline\pi] {p'} {p''}
  ] p {\pi \cdot \overline\pi } {p''}
\end{mathpar}
We write $\reducemany[\alpha] p {p'}$ to indicate that the
$\alpha$-labeled transition comes from the final $\textsc{\ActOne}$ step in the
reduction sequence and all preceding transitions are $\tau$ transitions.

%%% Local Variables:
%%% mode: latex
%%% TeX-master: "main"
%%% End:
