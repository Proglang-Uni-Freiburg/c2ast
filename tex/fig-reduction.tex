%\begin{figure}
  \declrel{Expression reduction}[$\reduce{e}{e}$]
  \begin{mathpar}
    \reducerule {
      (\ELam x e) v
    }{
      e[ v / a ]
    }

    \reducerule {
      (\ERec x v) v'
    }{
      v[ \ERec x v / x ] \; v'
    }

    \reducerule {
      \ELetU \EUnit e
    }{
      e
    }

    \reducerule {
      \ELetP {x_1} {x_2} {(v_1,v_2)} e
    }{
      e[v_1 / x_1][v_2 / x_2]
    }

    \reducerule[\reduce{e_1}{e_2}]{
      E[e_1]
    }{
      E[e_2]
    }
  \end{mathpar}
  \declrel{Process reduction (iso-recursive system)}[$\reduce{p}{p}$]
  \begin{mathpar}
    \reducerule{
      \PScope (\PPar* {\ESend v a} {\ERecv b} )
    }{
      \PScope (\PPar* {a} {(v,b)})
    }

    \reducerule{
      \PScope (\PPar* {\ESelect {l_j} a} {\ECase b { l_i \rightarrow e_i }})
    }{
      \PScope (\PPar* {a} {e_j \; b})
    }

    \reducerule{
      \PScope (\PPar* {\ETerm a} {\EWait b})
    }{
      \PPar* \EUnit \EUnit
    }

    \reducerule{
      E[\EFork e]
    }{
      \PScope (\PPar {E[a]} {e \; b})
    }

    \reducerule{
      \PScope (\PPar* {\ERoll a} {\ERoll b})
    }{
      \PPar* a b
    }

    \reducerule{
      \PScope (\PPar* {\EUnroll a} {\EUnroll b})
    }{
      \PPar* a b
    }

    \reducerule[\reduce{p}{p^\prime}] { \PPar p q } { \PPar {p^\prime} q } \and
    \reducerule[\reduce{p}{p^\prime}] { \PScope p } { \PScope p^\prime }   \and
    \reducerule[p \equiv q \\ \reduce{q}{q^\prime}] { p } { q^\prime }
  \end{mathpar}
%   \label{fig:equi-reduction}
%   \caption{Reduction relation for the equi-recursive system}
% \end{figure}

%%% Local Variables:
%%% mode: latex
%%% TeX-master: "main"
%%% End:
