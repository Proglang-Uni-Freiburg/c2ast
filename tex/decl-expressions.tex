\declrel{Syntax for constants, values, expressions and processes}
\begin{align*}
  &\text{constants}&
                     c \grmdef & \EUnit \grmor \EkwWait \grmor \EkwTerm \grmor \EkwSend \grmor \EkwRecv \grmor \EkwSelect[l] \grmor \EkwFork \\
  &\text{values}&
                   v \grmdef&
    c                        \grmor
    a                        \grmor
    \EPair v v            \grmor
    \ELam x e           \grmor
    \ERec x v
  \\
  &\text{tuple patterns}&
                          t \grmdef& \EUnit \grmor x \grmor \EPair x y
  \\
  &\text{expressions}&
                       e \grmdef&
    v                       \grmor
    x                       \grmor
    \EApp e e           \grmor
    \ELet t e e              \grmor
    \EPair e e                  \grmor
    % \ELetP x y e e          \grmor
    % \\ &&&
    %    \ESend e e \grmor
    %    \ERecv e \grmor
    % \ESelect l e            \grmor
         \ECase e { l_i \rightarrow v }
    %      \grmor
    % \EFork v
  \\
  & \text{processes}&
  p \grmdef&
    \PExp e                 \grmor
    \PPar p p               \grmor
    \PScope p
\end{align*}

The expression $\ELam {(x,y)} e$ abbreviates $\ELam z \ELetP xy z e$
for some variable $z$ not free in $e$.

We sometimes write $\ELetU {e_1} e_2$ for $\ELet{()}{e_1} e_2$.

The set of free variables $\FV{e}$ for an expression $e$ is standard.


%%% Local Variables:
%%% mode: latex
%%% TeX-master: "main"
%%% End:
