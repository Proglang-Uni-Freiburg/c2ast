
\section{Properties}
\label{sec:properties}

TODO (PJT): what do we mean with $\cong$?
\begin{itemize}
\item for ``normal'' types, it should be contextual equivalence(?)
\item for channels, it might be bisimilarity(?)
\item might be simpler with an LTS semantics, see proposed definition:
\end{itemize}

\begin{definition}
  For $\typing[\CNil]{e_1, e_2} T$ define contextual equivalence $e_1 \cong e_2$
  iff, for all contexts $C : T \to \TUnit$, $C[e_1] \Downarrow$ iff $C[e_2] \Downarrow$. 
\end{definition}

\begin{lemma}[Conversions]~\\[-\baselineskip]
  \begin{enumerate}
    \item If $f : T \eqt T'$, then $f' : T' \eqt T$
      such that $\typing[\CNil]{f}{\TFun T{T'}}$
      and  $\typing[\CNil]{f'}{\TFun {T'}T}$
      and  $f \circ f' \cong \ELam x x$
      and  $f' \circ f \cong \ELam x x$.
    \item If $g : S \eqs S'$, then $g' : S' \eqs S$
      such that $\typing[\CNil]{g}{\TFun {\TPair {S} {\TDual{S'}}} \TUnit}$
      and $\typing[\CNil]{g'}{\TFun {\TPair {S'} {\TDual{S}}} \TUnit}$
      and,\\
      for all channels $c : S$,  $c \cong \ELet{c'}{\EFork{( \ELam {\bar c'} g \; (c, \bar c')) }}{\EFork{( \ELam {\bar c} g \; (c', \bar c)) }} $.
      \\ \js{one of the $g$ should probably be $g'$, can't quite decipher which right now}
\end{enumerate}
\end{lemma}

Proving contextual equivalence is daunting. Maybe set up a logical relation? But how to do the session part?
(I don't think it works like this!)

\begin{align*}
  V (\TUnit) & = \{ (\EUnit, \EUnit) \}
  \\
  V (\TPair TU) & = \{ (v_1, v_2), (w_1, w_2) \mid (v_1, w_1) \in V (T), (v_2, w_2) \in V (U) \}
  \\
  V (\TFun TU) &= \{ (\ELam xe_1, \ELam xe_2) \mid \forall v, w. (v, w) \in V (T) \Rightarrow (e_1[v/x], e_2[w/x]) \in E (U)  \}
  \\
  V (\TEnd!, \TEnd!) &= ?
  \\
  V (\TIn TS) &= \{ (c_1, c_2) \mid \forall v, w. (v,w) \in V(T) \Rightarrow
                c_1 \stackrel{?v}{\to} e_1[x,c_1],
                c_2 \stackrel{?w}{\to} e_2[x,c_2],
                (e_1[v/x], e_2[w/x]) \in E(S) \}
  \\
  V (\TOut TS) &= \dots
  \\
  V (\TSelect{l_i: S_i}) &=  \{ (c_1, c_2) \mid \forall j. j \in I \Rightarrow
                c_1 \stackrel{!l_j}{\to} e_1[c_1],
                c_2 \stackrel{!l_j}{\to} e_2[c_2],
                           (e_1, e_2) \in E(S_j) \}
  \\
  V (\TCase{l_i: S_i}) &=  \{ (c_1, c_2) \mid \forall j. j \in I \Rightarrow
                c_1 \stackrel{?l_j}{\to} e_1[c_1],
                c_2 \stackrel{?l_j}{\to} e_2[c_2],
                         (e_1, e_2) \in E(S_j) \}
  \\
  E (T) &= \{ (e_1, e_2) \mid \forall v, w. e_1 \to^* v, e_2 \to^* w \Rightarrow (v, w)\in V (T) \}
\end{align*}

The session part seems wrong. I think we have to talk about processes stuck on a communication.
The following is inspired by a paper by Perez, Toninho, Pfenning ($+$ will need step indexing to deal with recursion): 

\begin{align*}
  L (c : \TEnd!) &= \{ (E[ \ETerm c ], E'[ \ETerm c ]) \mid (E[\EUnit], E'[\EUnit]) \in E (\TUnit)
                   \}
  \\
  L (c : \TOut TS) &= \{ (E[ \ESend v c ], E'[ \ESend w c ]) \mid \forall v, w. (v, w) \in V (T) \Rightarrow (E[c], E'[c]) \in L^* (c : S) \}
  \\
  L (c : \TIn TS) &= \{ (E[ \ERecv c ], E'[ \ERecv c ]) \mid \forall v, w. (v, w) \in V (T) \Rightarrow (E[(v,c)], E'[(w,c)]) \in L^* (c : S) \}
  \\
  L^* (c : S) &= \{ (e_1, e_2) \mid \forall e_1', e_2'. e_1 \to^* e_1', e_2 \to^* e_2' \Rightarrow (e_1', e_2') \in L (c : S) \}
\end{align*}

%%% Local Variables:
%%% mode: latex
%%% TeX-master: "main"
%%% End:
