\documentclass[adraft]{eptcs}

\usepackage[T1]{fontenc}
\usepackage{amsmath}
\usepackage{amssymb}
\usepackage{amsthm}
\usepackage{textcomp}
\usepackage{mathpartir}
\usepackage{mathtools}
\usepackage{microtype}
\usepackage{stmaryrd}
\usepackage{xcolor}
\usepackage{xparse}

\newtheorem{definition}{Definition}
\newtheorem{lemma}{Lemma}

\title{A silent semantics for isorecursive session types}
\newcommand\titlerunning{Isorecursive Sessions}
\author{Janek Spaderna \and Peter Thiemann \and Vasco Vasconcelos}
\newcommand\authorrunning{J. Spaderna, P. Thiemann, V. Vasconcelos}

\newcommand\note[2]{[\textcolor{red}{#1}: #2]}
\newcommand\js{\note{js}}
\newcommand\vv{\note{vv}}
\newcommand\pt{\note{pt}}
\newcommand\highlight[1]{\colorbox{yellow}{#1}}

% Macros to define type and term grammars
\newcommand\grmdef{\; \Coloneqq \;\;}
\newcommand\grmor{\; \mid \;}

% Formatting of type and expression keywords.
\newcommand\Tkw[1]{\mathtt{#1}}
\newcommand\Ekw[1]{\mathtt{#1}}
\newcommand\Econst[1]{\mathtt{#1}}

% Type syntax macros
\newcommand\TInt{\Tkw{Int}}
\newcommand\TUnit{()}
\newcommand\TPair[2]{#1 \otimes #2}
\newcommand\TFun[2]{#1 \rightarrow #2}
\newcommand\TEnd[1]{\Tkw{end #1}}
\newcommand\TIn[2]{{}?#1.#2}
\newcommand\TOut[2]{{}!#1.#2}
\newcommand\TSelect[2][i \in I]{\oplus\{\, #2 \,\}_{#1}}
\newcommand\TCase[2][i \in I]{\&\{\, #2 \,\}_{#1}}
\newcommand\TRec[1][X]{\mu #1.\;}
\newcommand\TDual[1]{\overline{#1}}

% Expression syntax macros
\newcommand\EUnit{()}
\newcommand\ELam[1]{\lambda #1.\;}
\newcommand\ERec[1]{\Ekw{rec}\; #1.\;}
\newcommand\ELet[2]{\Ekw{let}\; #1 = #2 \;\Ekw{in}\;}
\newcommand\ELetU[1]{#1;\;}%{\ELet{\EUnit}{#1}}
\newcommand\ELetP[3]{\ELet{(#1,#2)}{#3}}
\newcommand\EkwWait{\Econst{wait}}
\newcommand\EkwTerm{\Econst{term}}
\newcommand\EkwSend{\Econst{send}}
\newcommand\EkwRecv{\Econst{recv}}
\newcommand\EkwSelect{\Econst{sel}}
\newcommand\EkwRoll{\Econst{roll}}
\newcommand\EkwUnroll{\Econst{unroll}}
\newcommand\EkwFork{\Econst{fork}}
\newcommand\EWait[1]{\EkwWait\; #1}
\newcommand\ETerm[1]{\EkwTerm\; #1}
\newcommand\ESend[2]{\EkwSend\; #1 \; #2}
\newcommand\ERecv[1]{\EkwRecv\; #1}
\newcommand\ESelect[2]{\EkwSelect\; #1 \; #2}
\newcommand\ECase[2]{\Ekw{case}\; #1 \; \left\{\, #2 \,\right\}}
\newcommand\ERoll[1]{\EkwRoll\; #1}
\newcommand\EUnroll[1]{\EkwUnroll\; #1}
\newcommand\EFork[1]{\EkwFork\; #1}

\newcommand\EHole{[\;]}

% Type equivalence names
\newcommand\EqUnit{Eq-Unit}
\newcommand\EqPair{Eq-Pair}
\newcommand\EqFun{Eq-Fun}
\newcommand\EqS{Eq-S}
\newcommand\EqEnd[1]{Eq-End#1}
\newcommand\EqIn{Eq-In}
\newcommand\EqOut{Eq-Out}
\newcommand\EqSelect{Eq-Select}
\newcommand\EqCase{Eq-Case}
\newcommand\EqUnrollL{Eq-Unroll-L}
\newcommand\EqUnrollR{Eq-Unroll-R}
\newcommand\EqUnroll{Eq-Unroll}
\newcommand\EqRoll{Eq-Roll}

% Outputs a header for defining a new relation.
%   {#1} description/name
%   [#2] relation syntax, optional
\NewDocumentCommand \declrel
  { m o }
  {%
    \noindent%
    \emph{#1}%
    \IfValueT{#2}{\hfill\fbox{#2}}%
  }

% Outputs a rule for definitions of session type dualization.
%   {#1} type to be dualized
%   {#2} dualized result
\newcommand\dualdef[2]{\inferrule{\TDual{#1} {}={} #2}{}}

% Outputs a typing context, can be embellished with an index `i` like this
%
%     \Ctxt_i    or    \Ctxt_{longer_index}
\NewDocumentCommand \Ctxt
  { e{_} }
  { \IfNoValueTF{#1}{\Gamma}{\Gamma_{#1}} }

% The empty context.
\newcommand\CNil{\cdot}

% Typesets a single binding in a context
%
%   *       given => reusable binding, not given => linear binding
%   {#1}    binding name
%   {#2}    binding typing
\NewDocumentCommand \CBind
  { s m m }
  { #2 \IfBooleanTF{#1}{:^*}{:} #3 }

% A split context.
%
%   [#1]   1st sub-context, defaults to \Ctxt_1  
%   [#2]   2nd sub-context, defaults to \Ctxt_2  
\NewDocumentCommand \CSplit
  { O{\Ctxt_1} O{\Ctxt_2} }
  { #1 \circ #2 }

% Predicate on contexts: only unrrestricted bindings remaining
\newcommand\CExhausted[1][\Ctxt]{all^*(#1)}

% Macros for typing relations.
%
%   [#1]  typing context, defaults to  \Ctxt
%   {#2}  expression to type
%   {#3}  type
\newcommand\typing[3][\Ctxt]{#1 \vdash #2 : #3}

\newcommand\eqt{\sim_{\mathsf T}}
\newcommand\eqs{\sim_{\mathsf S}}

% Macros for reductions
\newcommand\reduce[3][]{#2 \stackrel{#1}\rightarrow #3}
\newcommand\reducemany[3][]{#2 \stackrel{#1}{\twoheadrightarrow} #3}
\newcommand\reduceto[3][]{#2 \stackrel{#1}\Downarrow #3}
\newcommand\reducerule[3][]{\inferrule{#1}{\reduce{#2}{#3}}}

\newcommand\PScope[1][a,b]{(\nu #1)}
\NewDocumentCommand \PPar
  { s m m }
  { \IfBooleanTF{#1}
    { \PParImpl {E_1[{#2}]} {E_2[{#3}]} }
    { \PParImpl {#2} {#3} } }
\newcommand\PParImpl[2]{#1 \| #2}

% Macros for translation
%
%   [#1] |- #2 ~> #3 : #4
%
% Context defaults to \Ctxt.
\newcommand\translation[4][\Ctxt]{#1 \vdash #2 \leadsto #3 : #4}

%%% Local Variables:
%%% mode: latex
%%% TeX-master: "main"
%%% End:


\begin{document}

\maketitle
\begin{abstract}
  Almost all formalizations of recursive session types rely on
  equirecursion because the explicit isomorphism to roll/unroll the
  recursive type is said to give rise to additional communication and
  synchronization. We examine this statement from first principles and
  find mild conditions under which these communications can be
  elided.
\end{abstract}


\section{Introduction}
\label{sec:introduction}

Recursion in session types is almost always interpreted as
equirecursion. For example, the two session types $\TRec{\TOut \TInt X} $
and $\TRec{\TOut \TInt {\TOut \TInt X}} $ are considered equal because
their unfoldings as infinite trees are equal. That means, servers with
these types are interchangeable without breaking the protocol and
functions that expect arguments of one type also accept arguments of
the other type.

Using an isorecursive interpretation the two types are not equal
because unrolling of the recursion is a distinct communication
step. While channels of these two types are no longer
interchangeable, we can nevertheless define a pair of conversion
functions
$f : \TFun{\TRec{\TOut \TInt X}}{\TRec{\TOut \TInt {\TOut \TInt X}}}$
and
$f' : \TFun{\TRec{\TOut \TInt {\TOut \TInt X}}}{\TRec{\TOut \TInt X}}$
that mediate between the two types. Concretely:
\begin{align*}
  f &=       \ELam {c_1} \EFork{ \ELam {c_2} \ERec g %{
      \begin{array}[t]{l}
      \ELam{(c_1, c_2)} \\
      \ELet{(x_2,c_2)}{\ERecv{(\EUnroll{c_2})}} \\
      \ELet{c_1}{\ESend{x_2}{(\EUnroll{c_1})}} \\
      \ELet{(x_2,c_2)}{\ERecv{({c_2})}} \\
      \ELet{c_1}{\ESend{x_2}{(\EUnroll{c_1})}} \\
        g \; ({c_1}, {c_2}) %}
      \end{array}}
\end{align*}
As the session types are recursive, so are the conversion
functions. This conversion function creates one additional
channel and forks one additional process. Moreover, it leads to
additional communication as evidenced by the folklore rule for
isorecursive processes:
\begin{gather*}
  \reducerule{
    \PScope (\PPar* {\EUnroll a} {\EUnroll b})
  }{
    \PScope (\PPar* a b)
  }
\end{gather*}
However, if the underlying recursive session type is contractive, then we can
elide this extra communication, a move which essentially reinterprets
isorecursion as equirecursion.
This move makes a type conversion like $f$ act like an identity
function, which forwards messages from one channel to another, as the
$\EUnroll\_$ operations are just identities. 


Clearly, this transformation goes both ways. If we start from an
equirecursive program, we can transform it into an isorecursive one by
making the type conversion explicit. We can execute the isorecursive
program either with its ``natural'' semantics or with the silent
semantics that treats $\EUnroll\_$ as an identity.
The latter interpretation also makes the equirecursive version
``wire-compatible'' with the isorecursive one.\footnote{As we have
  mostly conjectures to offer on the operational side, the reviewers
  may consider downgrading this submission to a talk proposal.}
  

%%% Local Variables:
%%% mode: latex
%%% TeX-master: "main"
%%% End:

\section{Equi-recursion}

\subsection{Types}

\declrel{Type syntax}
\begin{align*}
  T \grmdef&
    S               \grmor
    \TUnit          \grmor
    \TPair T T      \grmor
    \TFun  T T      \\
  S \grmdef&
    \TEnd !         \grmor
    \TEnd ?         \grmor
    \TOut T S       \grmor
    \TIn  T S       \grmor
    \TSelect{ l_i: S_i } \grmor
    \TCase{ l_i: S_i }   \grmor
    \TRec[X] S      \grmor
    X
\end{align*}

\vv{\TUnit\ can be confused with a linear logic proposition (for which we use $\TEnd!$); the old
  $()$ may be better.}

\declrel{Dualization of session types}[$\TDual S = S$]
\begin{align*}
  \TDual X &= X                               &
  \TDual{\TEnd !} &= \TEnd ?                  &
  \TDual{\TOut T S} &= \TIn T \TDual{S}       &
  \TDual{\TSelect{ l_i: S_i }} &=
    \TCase{ l_i: \TDual{S_i} }                \\
  \TDual{\TRec S} &= \TRec{\TDual S}          &
  \TDual{\TEnd ?} &= \TEnd !                  &
  \TDual{\TIn T S} &= \TOut T \TDual{S}       &
  \TDual{\TCase{ l_i: S_i }} &=
    \TSelect{ l_i: \TDual{S_i} }
\end{align*}

\vv{
  \begin{itemize}
  \item The dual function for recursive types is a bit more complicated; pls have
    a look at ``the final cut''.
  \item Iso-recursive expressions $e$ must appear before type equivalence.
  \item IMO, $T,U$ reads better than $T_1, T_2$.
  \item The expression for session types isomorphism may be curried. Then we
    seek these results, right? (do they hold?)

    Lemma. If $e : T \eqt U$, then $\typingE{e}{\TFun{T}{U}}$.

    Lemma. If $e : R \eqs S$, then $\typingE{e}{\TFun{R}{\TFun{S}{\TUnit}}}$.
\item $f^{-1}$ is confusing; why not a simple $g$?
\item Added an easier to read, IMO, rule for \EqFun{} and (curried) for \EqOut
\item \EqPair{} uses an abbreviation, right? $\lambda p. \text{let}(x,y) = p
  \text{ in } \dots$
\item Added suggestion for roll and unroll; these constants may have function
  types. (types don't work :(
  \end{itemize}
}

\begin{figure}[tp]
  \declrel{Type syntax}
\begin{align*}
  T,U \grmdef&
    S^\emptyset               \grmor
    \TUnit          \grmor
    \TPair TU      \grmor
    \TFun  T U      \\
  R^{\mathcal X}, S^{\mathcal X} \grmdef&
    \TEnd !         \grmor
    \TEnd ?         \grmor
    \TOut T S^{\mathcal X}       \grmor
    \TIn  T S^{\mathcal X}       \grmor
    \TSelect{ l_i: S^{\mathcal X}_i } \grmor
    \TCase{ l_i: S^{\mathcal X}_i }   \grmor
    \TRec[X] S^{\mathcal X \cup \{X\}}      \grmor
    X^{\in \mathcal X}
\end{align*}

\declrel{Dualization of session types}[$\TDual S = S$]
\begin{align*}
  \TDual X &= X                               &
  \TDual{\TEnd !} &= \TEnd ?                  &
  \TDual{\TOut T S} &= \TIn T \TDual{S}       &
  \TDual{\TSelect{ l_i: S_i }} &=
    \TCase{ l_i: \TDual{S_i} }                \\
  \TDual{\TRec S} &= \TRec{\TDual S}          &
  \TDual{\TEnd ?} &= \TEnd !                  &
  \TDual{\TIn T S} &= \TOut T \TDual{S}       &
  \TDual{\TCase{ l_i: S_i }} &=
    \TSelect{ l_i: \TDual{S_i} }
\end{align*}

%%% Local Variables:
%%% mode: latex
%%% TeX-master: "main"
%%% End:

  \declrel{Type equivalence}[$\convexpr{f} : T \eqt U$]
  \begin{mathpar}
    \inferrule[\EqUnit]{
    }{
      \convexpr{\ELam x x} :
      \TUnit \eqt \TUnit
      % : \ELam x x
    }

    \inferrule[\EqPair]{
      \convexpr{f_T} : T \eqt T' \\ %: f'_T \\
      \convexpr{f_U} : U \eqt U' %: f'_U
    }{
      \convexpr{\ELam {(x_1,x_2)} (f_T \; x_1, f_U \; x_2)} :
      \TPair T U \eqt \TPair {T'} {U'}
      % : \ELam {(x_1,x_2)} (f'_T \; x_1, f'_U \; x_2)
    }

    \inferrule[\EqFun]{
      \convexpr{f_{T'}} : T' \eqt T \\ %: f'_T \\
      \convexpr {f_U} : U \eqt U' %: f'_U
    }{
      \convexpr{\ELam x f_U \circ x \circ f_{T'}} :
      \TFun {T} {U} \eqt \TFun {T'} {U'}
      % : \ELam x f'_U \circ x \circ f_T
    }

    \inferrule[\EqS]{
      \convexpr{g} : R \eqs S
    }{
      \convexpr{\ELam {c_1} \EFork{ \ELam {c_2} g \; (c_1, c_2) }} :
      R \eqt S
      %: \ELam {c_1} \EFork{ \ELam {c_2} g' \; (c_1, c_2) }
    }
  \end{mathpar}
  \declrel{Session type equivalence}[$\eqsrel{g}{R}{S}$]
  \begin{mathpar}
    \inferrule[\EqAssump]{
      \convexpr{x} : \TRec R \eqs S \in \eqsctxt
    }{
      \eqsrel{x}{\TRec R}{S}
    }

    \inferrule[\EqUnrollL]{
      \eqsrel[\eqsctxt , \convexpr{x} : \TRec R \eqs S]
        {g}{R[\TRec R / X]}{S} \\
      \convexpr{x'} : \TRec R \eqs S \notin \eqsctxt
    }{
      \eqsrel{
        \ERec x
        \ELam{(c_1, c_2)}
        g \; (\EUnroll c_1, c_2)
      }{\TRec R}{S}
    }

    \inferrule[\EqUnrollR]{
      \eqsrel
      {g}{R}{S[\TRec S / X]}
      \\
      R \ne \TRec R'
    }{
      \eqsrel{
        \ELam{(c_1, c_2)}
        g \; (c_1, \EUnroll c_2)
      }{R}{\TRec S}
    }

    % \inferrule[\EqUnrollL]{
    %   \eqsrel[\eqsctxt , \convexpr{x} : \TRec S \eqs S']
    %     {g}{S[\TRec S / X]}{S'}
    % }{
    %   \eqsrel{
    %     \ERec x
    %     \ELam{(c_1, c_2)}
    %     g \; (\EUnroll c_1, c_2)
    %   }{\TRec S}{S'}
    % }
    %   
    % \inferrule[\EqUnrollR]{
    %   \eqsrel[\eqsctxt , \convexpr{x} : S \eqs \TRec S']
    %     {g}{S}{S'[\TRec S' / X]}
    % }{
    %   \eqsrel{
    %     \ERec x
    %     \ELam{(c_1, c_2)}
    %     g \; (c_1, \EUnroll c_2)
    %   }{S}{\TRec S'}
    % }
    %   
    \inferrule[\EqEnd !]{
    }{
      \eqsrel{
        \ELam {(c_1, c_2)} \ELetU {\EWait {c_2}} {\ETerm {c_1}}
      }{\TEnd !}{\TEnd !}
    }

    \inferrule[\EqEnd ?]{
    }{
      \eqsrel{
        \ELam {(c_1, c_2)} \ELetU {\EWait {c_1}} {\ETerm {c_2}}
      }{\TEnd ?}{\TEnd ?}
    }

    \inferrule[\EqOut]{
      \convexpr{f} : U \eqt T \\
      \eqsrel{g}{R}{S}
    }{
      \colorlet{outer}{.}
      {\begin{array}{r@{}l}
        \eqsctxt \vdash
            \convcolor \ELam {(c_1, c_2)}
          & \convcolor \ELetP {u} {c_2} {\ERecv{c_2}} \\
          & \convcolor \hspace{-2em} \ELet {c_1} {\ESend {f \; u} {c_1}} \\
          & \convcolor \hspace{-2em} g \; (c_1, c_2)
            \color{outer} \hspace{4em}
              : \TOut {T} {R} \eqs \TOut {U} {S}
      \end{array}}
    }

    \inferrule[\EqSelect]{
      \eqsrel{g_i}{R_i}{S_i}
    }{
      \eqsctxt \vdash \convexpr{
        \ELam {(c_1, c_2)}
        \ECase {c_2} { l_i \rightarrow \ELam {c_2} g_i \; (\ESelect {l_i} c_1, c_2) } \\ {}
      } \\\\
      : \TSelect { l_i : R_i } \eqs \TSelect { l_i : S_i }
    }


    \inferrule[\EqIn]{
      \convexpr{f} : T \eqt U \\
      \eqsrel{g}{R}{S}
    }{
      \colorlet{outer}{.}
      {\begin{array}{r@{}l}
        \eqsctxt \vdash
            \convcolor \ELam {(c_1, c_2)}
          & \convcolor \ELetP {t} {c_1} {\ERecv{c_1}} \\
          & \convcolor \hspace{-2em} \ELet {c_2} {\ESend {f \; t} {c_2}} \\
          & \convcolor \hspace{-2em} g \; (c_1, c_2)
            \color{outer} \hspace{4em}
              : \TIn {T} {R} \eqs \TIn {U} {S}
      \end{array}}
    }

    \inferrule[\EqCase]{
      \eqsrel{g_i}{R_i}{S_i}
    }{
      \eqsctxt \vdash \convexpr{
        \ELam {(c_1, c_2)}
        \ECase {c_1} { l_i \rightarrow \ELam {c_1} g_i \; (c_1, \ESelect {l_i} c_2) }
      } \\\\
      : \TCase { l_i : R_i } \eqs \TCase { l_i : S_i }
    }
  \end{mathpar}
  \caption{Equi-recursive system: types, duality, inductive type and session type equivalence}
  \label{fig:equi-equivalence}
\end{figure}

%%% Local Variables:
%%% mode: latex
%%% TeX-master: "main"
%%% End:


\subsection{Expressions}

\declrel{Syntax for values, expressions and processes}
\begin{align*}
  v \grmdef&
    \EUnit                   \grmor
    (v, v)                   \grmor
    \ELam x e                \grmor
    \ERec x e                \grmor
    \EkwSend \grmor \EkwRecv \grmor
    \EkwTerm \grmor \EkwWait
  \\
  e \grmdef&
    v                       \grmor
    x                       \grmor
    e \; e                  \grmor
    \ELetU e e              \grmor
    (e, e)                  \grmor
    \ELetP x y e e          \grmor
  \\ &
    \ESelect l e            \grmor
    \ECase e { l_i \rightarrow e } \grmor
    \EFork e
  \\
  p \grmdef&
    e                       \grmor
    \PPar p p               \grmor
    \PScope p
\end{align*}


\subsubsection{Expression typing}

\declrel{Typing contexts}
\begin{align*}
  \Ctxt \grmdef&
    \CNil \grmor \Ctxt, \CBind x T \grmor \Ctxt, \CBind* x T
\end{align*}

\declrel{Context exhaustion}[$\CExhausted$]
\begin{mathpar}
  \inferrule{}{\CExhausted[\CNil]} \and
  \inferrule{\CExhausted}{\CExhausted[\Ctxt, \CBind* x T]}
\end{mathpar}

\declrel{Context splitting}[$\Ctxt = \CSplit[\Ctxt][\Ctxt]$]
\begin{mathpar}
  \inferrule{}{\cdot = \CSplit[\cdot][\cdot]} \and
  \inferrule{\Ctxt = \CSplit}{\Ctxt, \CBind  x T = \CSplit[(\Ctxt_1 , \CBind  x T)][ \Ctxt_2               ]} \and
  \inferrule{\Ctxt = \CSplit}{\Ctxt, \CBind  x T = \CSplit[ \Ctxt_ 1              ][(\Ctxt_2 , \CBind  x T)]} \and
  \inferrule{\Ctxt = \CSplit}{\Ctxt, \CBind* x T = \CSplit[(\Ctxt_1 , \CBind* x T)][(\Ctxt_2 , \CBind* x T)]}
\end{mathpar}

% \begin{figure}
  \declrel{Typing rules for expressions in the equi-recursive system}[$\typingE{e}{T}$]
  \begin{mathpar}
    \inferrule{ }{\typingE{\EUnit}{\TUnit}}

    \inferrule{
      \CExhausted
    }{
      \typingE[\Ctxt, \CBind x T]{x}{T}
    }

    \inferrule{
      \CExhausted
    }{
      \typingE[\Ctxt, \CBind* x T]{x}{T}
    }

    \inferrule{
      \CExhausted \\
      \typingE[\Ctxt, \CBind* x {\TFun {T_1} {T_2}}]{e}{\TFun {T_1} {T_2}}
    }{
      \typingE{\ERec x e}{\TFun{T_1}{T_2}}
    }

    \inferrule{
      \typingE[\Ctxt, \CBind x {T_1}]{e}{T_2}
    }{
      \typingE{\ELam x e}{\TFun{T_1}{T_2}}
    }

    \inferrule{
      \typingE[\Ctxt_1]{e_1}{\TFun {T_1}{T_2}}
      \\
      \typingE[\Ctxt_2]{e_2}{T_1}
    }{
      \typingE[\CSplit]{e_1 e_2}{T_2}
    }

    \inferrule{
      \typingE[\Ctxt_1]{e_1}{T_1} \\
      \typingE[\Ctxt_2]{e_2}{T_2}
    }{
      \typingE[\CSplit]{
        \ELetU{e_1}{e_2}
      }{ T_2 }
    }

    \inferrule{
      \typingE[\Ctxt_1]{e_1}{T_1} \\
      \typingE[\Ctxt_2]{e_2}{T_2}
    }{
      \typingE[\CSplit]{(e_1,e_2)}{\TPair{T_1}{T_2}}
    }

    \inferrule{
      \typingE[\Ctxt_1]{e_1}{\TPair{T_1}{T_2}} \\
      \typingE[\Ctxt_2, \CBind x {T_1}, \CBind y {T_2}]{e_2}{T_3}
    }{
      \typingE[\CSplit]{
        \ELetP xy {e_1} {e_2}
      }{ T_3 }
    }

    \inferrule{
      \typingE{e}{\TEnd ?}
    }{
      \typingE{\EWait e}{\TUnit}
    }

    \inferrule{
      \typingE{e}{\TEnd !}
    }{
      \typingE{\ETerm e}{\TUnit}
    }

    \inferrule{
      \typingE[\Ctxt_1]{e_1} T  \\
      \typingE[\Ctxt_2]{e_2} {\TOut T S}
    }{
      \typingE[\CSplit]{\ESend{e_1}{e_2}} S
    }

    \inferrule{
      \typingE{e_1} {\TIn T S}
    }{
      \typingE{\ERecv{e_1}}{\TPair T S}
    }

    \inferrule{
      \typingE{e}{\TSelect{l_i : S_i}} \\
      j \in I
    }{
      \typingE{\ESelect{l_j}{e}}{S_j}
    }

    \inferrule{
      \typingE[\Ctxt_1]{e}{\TCase{l_i : S_i}} \\
      \typingE[\Ctxt_2]{e_i}{\TFun{S_i}{T}}
    }{
      \typingE[\CSplit]{
        \ECase e { l_i \rightarrow e_i }
      }{T}
    }

    \inferrule{
      \typingI{e}{\TFun {\TDual S} \TUnit}
    }{
      \typingI{\EFork e}{S}
    }
  \end{mathpar}
%   \label{fig:equi-typing-rules}
%   \caption{Typing rules for expressions in the equi-recursive system}
% \end{figure}

%%% Local Variables:
%%% mode: latex
%%% TeX-master: "main"
%%% End:



\subsubsection{Operational semantics}

\declrel{Structural congruence of processes}[$p \equiv p$]\medskip\\
The structural congruence relation on processes is defined as the smallest
congruence relation that includes the commutative monoidal rules with the
binary operator being parallel process composition $\PPar \_ \_$ and
value~$\EUnit$ as the neutral element, and scope extrusion:
\begin{align*}
  \PPar{\PScope p}{q} \equiv \PScope (\PPar p q)
  &
  \text{if $a,b$ not free in $q$}
  \\
  \PScope p \equiv \PScope[b,a] p
\end{align*}

\declrel{Evaluation contexts}
\begin{align*}
  E \grmdef&
    [] \grmor
    E \; e \grmor
    v \; E \grmor
    \ELetU E e \grmor
    (E,e) \grmor
    (v,E) \grmor
    \ELetP xy E e \grmor
  \\ &
    \ESelect l E \grmor 
    \ECase E { l_i \rightarrow e_i }
\end{align*}

%\begin{figure}
  \declrel{Reduction relation in the equi-recursive system}[$\reduceE{p}{p}$]
  \begin{mathpar}
    \reduceruleE {
      (\ELam x e) v
    }{
      e[ v / a ]
    }

    \reduceruleE {
      (\ERec x e) v
    }{
      e[ \ERec x e / x ] \; v
    }

    \reduceruleE {
      \ELetU \EUnit e
    }{
      e
    }

    \reduceruleE {
      \ELetP {x_1} {x_2} {(v_1,v_2)} e
    }{
      e[v_1 / x_1][v_2 / x_2]
    }

    \reduceruleE[\reduceE{e_1}{e_2}]{
      E[e_1]
    }{
      E[e_2]
    }

    \reduceruleE{
      \PScope (\PPar* {\ESend v a} {\ERecv b} )
    }{
      \PScope (\PPar* {a} {(v,b)})
    }

    \reduceruleE{
      \PScope (\PPar* {\ESelect {l_j} a} {\ECase b { l_i \rightarrow e_i }})
    }{
      \PScope (\PPar* {a} {e_j \; b})
    }

    \reduceruleE{
      \PScope (\PPar* {\ETerm a} {\EWait b})
    }{
      \PPar* \EUnit \EUnit
    }

    \reduceruleE{
      E[\EFork e]
    }{
      \PScope (\PPar {E[a]} {e \; b})
    }

    \reduceruleE[\reduceE{p}{p^\prime}] { \PPar p q } { \PPar {p^\prime} q } \and
    \reduceruleE[\reduceE{p}{p^\prime}] { \PScope p } { \PScope p^\prime }   \and
    \reduceruleE[p \equiv q \\ \reduceE{q}{q^\prime}] { p } { q^\prime }
  \end{mathpar}
  Dual $\PScope[b,a]$ rules for $\EkwSend$/$\EkwRecv$, $\EkwSelect$/$\EkwCase$,
  $\EkwTerm$/$\EkwWait$ are omitted.
%   \label{fig:equi-reduction}
%   \caption{Reduction relation for the equi-recursive system}
% \end{figure}


%%% Local Variables:
%%% mode: latex
%%% TeX-master: "main"
%%% End:

\section{Iso-recursion}
\label{sec:iso-recursion}

The traditional equi-recursive system can---with very few and localized changes---be turned
into an iso-recursive system. The type syntax and dualization of
session types require no adjustments.

We extend the syntax with an extra expression to unroll a recursive
type,
\begin{align*}
    e \grmdef&
               \EUnroll e
\end{align*}
%
process reduction with an extra rule:
%
\begin{mathpar}
    % \reducerule{
    %   \PScope (\PPar* {\ERoll a} {\ERoll b})
    % }{
    %   \PPar* a b
    % }
    %   
    \reducerule{
      \PScope (\PPar* {\EUnroll a} {\EUnroll b})
    }{
      \PScope (\PPar* a b)
    }
\end{mathpar}
%
and we replace the typing rule for conversion by a rule for explicit unrolling:
\begin{mathpar}
  \inferrule{
    \typing{e}{\TRec S}
  }{
    \typing{\EUnroll e}{S[\TRec[X] S / X]}
  }
\end{mathpar}

The resulting system is much more tedious as the example from the
introduction demonstrates. Nevertheless, we can go forth and back
between the equirecursive system and the isorecursive system under
certain mild conditions.



\subsection{Expression typing}

\declrel{Typing contexts}
\begin{align*}
  \Ctxt \grmdef&
    \CNil \grmor \Ctxt, \CBind x T \grmor \Ctxt, \CBind* x T
\end{align*}

\declrel{Context exhaustion}[$\CExhausted$]
\begin{mathpar}
  \inferrule{ }{\CExhausted[\CNil]} \and
  \inferrule{\CExhausted}{\CExhausted[\Ctxt, \CBind* x T]}
\end{mathpar}

\declrel{Context splitting}[$\Ctxt = \CSplit[\Ctxt][\Ctxt]$]
\begin{mathpar}
  \inferrule{}{\cdot = \CSplit[\cdot][\cdot]} \and
  \inferrule{\Ctxt = \CSplit}{\Ctxt, \CBind  x T = \CSplit[(\Ctxt_1 , \CBind  x T)][ \Ctxt_2               ]} \and
  \inferrule{\Ctxt = \CSplit}{\Ctxt, \CBind  x T = \CSplit[ \Ctxt_ 1              ][(\Ctxt_2 , \CBind  x T)]} \and
  \inferrule{\Ctxt = \CSplit}{\Ctxt, \CBind* x T = \CSplit[(\Ctxt_1 , \CBind* x T)][(\Ctxt_2 , \CBind* x T)]}
\end{mathpar}

% \begin{figure}
  \declrel{Typing rules for expressions in the iso-recursive system}[$\typing{e}{T}$]
  \begin{mathpar}
    \inferrule{ 
      \CExhausted
    }{
      \typing{\EUnit}{\TUnit}
    }

    \inferrule{
      \CExhausted
    }{
      \typing[\Ctxt, \CBind x T]{x}{T}
    }

    \inferrule{
      \CExhausted
    }{
      \typing[\Ctxt, \CBind* x T]{x}{T}
    }

    \inferrule{
      \CExhausted \\
      \typing[\Ctxt, \CBind* x {\TFun T U}]{e}{\TFun T U}
    }{
      \typing{\ERec x e}{\TFun{T}{U}}
    }

    \inferrule{
      \typing[\Ctxt, \CBind x T]{e}{U}
    }{
      \typing{\ELam x e}{\TFun T U}
    }

    \inferrule{
      \typing[\Ctxt_1]{e_1}{\TFun T U}
      \\
      \typing[\Ctxt_2]{e_2}{T}
    }{
      \typing[\CSplit]{e_1 e_2}{U}
    }

    \inferrule{
      \typing[\Ctxt_1]{e_1}{\TUnit} \\
      \typing[\Ctxt_2]{e_2}{T}
    }{
      \typing[\CSplit]{
        \ELetU{e_1}{e_2}
      }{ T }
    }

    \inferrule{
      \typing[\Ctxt_1]{e_1}{T} \\
      \typing[\Ctxt_2]{e_2}{U}
    }{
      \typing[\CSplit]{(e_1,e_2)}{\TPair T U}
    }

    \inferrule{
      \typing[\Ctxt_1]{e_1}{\TPair{T_1}{T_2}} \\
      \typing[\Ctxt_2, \CBind x {T_1}, \CBind y {T_2}]{e_2}{U}
    }{
      \typing[\CSplit]{
        \ELetP xy {e_1} {e_2}
      }{ U }
    }

    \inferrule{
      \typing{e}{\TEnd ?}
    }{
      \typing{\EWait e}{\TUnit}
    }

    \inferrule{
      \typing{e}{\TEnd !}
    }{
      \typing{\ETerm e}{\TUnit}
    }

    \inferrule{
      \typing[\Ctxt_1]{e_1} T  \\
      \typing[\Ctxt_2]{e_2} {\TOut T S}
    }{
      \typing[\CSplit]{\ESend{e_1}{e_2}} S
    }

    \inferrule{
      \typing{e_1} {\TIn T S}
    }{
      \typing{\ERecv{e_1}}{\TPair T S}
    }

    \inferrule{
      \typing{e}{\TSelect{l_i : S_i}} \\
      j \in I
    }{
      \typing{\ESelect{l_j}{e}}{S_j}
    }

    \inferrule{
      \typing[\Ctxt_1]{e}{\TCase{l_i : S_i}} \\
      \typing[\Ctxt_2]{e_i}{\TFun{S_i}{T}}
    }{
      \typing[\CSplit]{
        \ECase e { l_i \rightarrow e_i }
      }{T}
    }

    \inferrule{
      \typing{e}{\TFun {\TDual S} \TUnit}
    }{
      \typing{\EFork e}{S}
    }

    \inferrule{
      \typing{e}{S[\TRec[X] S / X]}
    }{
      \typing{\ERoll e}{\TRec S}
    }
  
    \inferrule{
      \typing{e}{\TRec S}
    }{
      \typing{\EUnroll e}{S[\TRec[X] S / X]}
    }
%
%   \inferrule{
%     \typing e T \\ f : T \eqt U : f'
%   }{
%     \typing e U
%   }
  \end{mathpar}
%   \label{fig:iso-typing-rules}
%   \caption{Typing rules for expressions in the iso-recursive system}
% \end{figure}


% Typing of expressions in the iso-recursive system is largely the same as for
% expressions in the equi-recursive system. As the only change, two additional
% typing rules are introduced concerned with $\EkwRoll$ and $\EkwUnroll$.
% \begin{mathpar}
%   \inferrule{
%     \typing{e}{S[\TRec[X] S / X]}
%   }{
%     \typing{\ERoll e}{\TRec S}
%   }
% 
%   \inferrule{
%     \typing{e}{\TRec S}
%   }{
%     \typing{\EUnroll e}{S[\TRec[X] S / X]}
%   }
% \end{mathpar}

% Evaluation contexts for the iso-recursive system have the same shape as the
% contexts in the equi-recursive system. The reduction relation is extended by
% two rules concerned with $\EkwRoll$ and $\EkwUnroll$.
% \begin{mathpar}
%   \reducerule{
%     \PScope (\PPar {E[\ERoll a]} {p})
%   }{
%     \PScope (\PPar {E[a]} {p})
%   }
% 
%   \reducerule{
%     \PScope (\PPar {E[\EUnroll a]} {p})
%   }{
%     \PScope (\PPar {E[a]} {p})
%   }
% \end{mathpar}
% Dual $\PScope[b,a]$ rules omitted.

%%% Local Variables:
%%% mode: latex
%%% TeX-master: "main"
%%% End:


\section{Properties}
\label{sec:properties}

TODO (PJT): what do we mean with $\cong$?
\begin{itemize}
\item for ``normal'' types, it should be contextual equivalence(?)
\item for channels, it might be bisimilarity(?)
\item might be simpler with an LTS semantics, see proposed definition:
\end{itemize}

\begin{definition}
  For $\typing[\CNil]{e_1, e_2} T$ define contextual equivalence $e_1 \cong e_2$
  iff, for all contexts $C : T \to \TUnit$, $C[e_1] \Downarrow$ iff $C[e_2] \Downarrow$. 
\end{definition}

\begin{lemma}[Conversions]~\\[-\baselineskip]
  \begin{enumerate}
    \item If $f : T \eqt T'$, then $f' : T' \eqt T$
      such that $\typing[\CNil]{f}{\TFun T{T'}}$
      and  $\typing[\CNil]{f'}{\TFun {T'}T}$
      and  $f \circ f' \cong \ELam x x$
      and  $f' \circ f \cong \ELam x x$.
    \item If $g : S \eqs S'$, then $g' : S' \eqs S$
      such that $\typing[\CNil]{g}{\TFun {\TPair {S} {\TDual{S'}}} \TUnit}$
      and $\typing[\CNil]{g'}{\TFun {\TPair {S'} {\TDual{S}}} \TUnit}$
      and,\\
      for all channels $c : S$,  $c \cong \ELet{c'}{\EFork{( \ELam {\bar c'} g \; (c, \bar c')) }}{\EFork{( \ELam {\bar c} g' \; (c', \bar c)) }} $.
\end{enumerate}
\end{lemma}

Proving contextual equivalence is daunting. Maybe set up a logical relation? But how to do the session part?
(I don't think it works like this!)

\begin{definition}
  Define $\reduceto e{e'}$ by $\reducemany e{e'}$ and $e'$
  irreducible.
\end{definition}
\begin{align*}
  V (\TUnit) & = \{ (\EUnit, \EUnit) \}
  \\
  V (\TPair TU) & = \{ (v_1, v_2), (w_1, w_2) \mid (v_1, w_1) \in V (T), (v_2, w_2) \in V (U) \}
  \\
  V (\TFun TU) &= \{ (\ELam xe_1, \ELam xe_2) \mid \forall v, w. (v, w) \in V (T) \Rightarrow (e_1[v/x], e_2[w/x]) \in E (U)  \}
  \\
  V (\TEnd!) &= \{ (c_1, c_2) \mid
                       \reduce{\ETerm c_1}{\EUnit}, \reduce{\ETerm
                       c_2}{\EUnit} \}
  \\
  V (\TEnd?) &= \{ (c_1, c_2) \mid
               \reduce{\EWait c_1}{\EUnit}, \reduce{\EWait
                       c_2}{\EUnit} \}
  \\
  V (\TIn TS) &= \{ (c_1, c_2) \mid
                \reduce{\ERecv c_1}{(v, c_1')}, 
                \reduce{\ERecv c_2}{(w, c_2')},
                (v,w) \in V(T),
                (c_1', c_2') \in V (S) \}
  % \\
  %               &\stackrel{?}{=}
  %                 \{ (c_1, c_2) \mid 
  %                 (\ERecv c_1,  \ERecv c_2) \in E (\TPair TS) \}
  \\
  V (\TOut TS) &= \{ (c_1, c_2) \mid \forall v, w. (v,w) \in V(T)
                \Rightarrow
                \reduce{\ESend v c_1}{c_1'}, 
                 \reduce{\ESend w c_2}{c_2'},
                 (c_1', c_2') \in V(S) \}
  % \\
  %               &\stackrel{?}{=}
  %                 \{ (c_1, c_2) \mid \mid \forall v, w. (v,w) \in V(T)
  %               \Rightarrow
  %                 (\ESend v c_1,  \ESend w c_2) \in E (S) \}
  \\
  V (\TSelect{l_i: S_i}) &=  \{ (c_1, c_2) \mid \forall j. j \in I
                           \Rightarrow
                           \reduce{\ESelect{l_j} c_1 }{c_1'}, 
                           \reduce{\ESelect{l_j} c_2 }{c_2'},
                           (c_1', c_2') \in V(S_j)
                           \}
  % \\&\stackrel{?}{=}
  % \{ (c_1, c_2) \mid \forall j. j \in I
  % \Rightarrow
  % (\ESelect{l_j}{c_1}, \ESelect{l_j}{c_1}) \in E (S_j)
  % \}
  \\
  V (\TCase{l_i: S_i})
             &= \{ (c_1, c_2) \mid \forall j. j \in I \Rightarrow
                         c_1 \stackrel{?l_j}{\to} e_1[c_1],
                c_2 \stackrel{?l_j}{\to} e_2[c_2],
                         (e_1, e_2) \in E(S_j) \}
  \\&\stackrel{?}{=}
  \{ (c_1, c_2) \mid
  \reduce{\ECase{c_1}{l_i \to e_{1i}}}{e_{1j}\; c_1'}, 
  \reduce{\ECase{c_2}{l_i \to e_{2i}}}{e_{2j}\; c_2'},
  (c_1', c_2') \in V (S_j)
  \}
  % \\&\stackrel{??}{=}
  % \{ (c_1, c_2) \mid \exists j\in I, (\ECase{c_1}{\dots},
  % \ECase{c_2}{\dots}) \in E (S_j) \}
  \\
  E (T) &= \{ (e_1, e_2) \mid \forall v_1, v_2, E_1, E_2. \reduceto{e_1}{E_1[v_1]}, \reduceto{e_2}{ E_2[v_2]} \Rightarrow (v_1, v_2)\in V (T) \}
\end{align*}

The session part seems wrong. I think we have to talk about processes stuck on a communication.
The following is inspired by a paper by Perez, Toninho, Pfenning ($+$ will need step indexing to deal with recursion): 

\begin{align*}
  L (c : \TEnd!) &= \{ (E[ \ETerm c ], E'[ \ETerm c ]) \mid (E[\EUnit], E'[\EUnit]) \in E (\TUnit)
                   \}
  \\
  L (c : \TOut TS) &= \{ (E[ \ESend v c ], E'[ \ESend w c ]) \mid \forall v, w. (v, w) \in V (T) \Rightarrow (E[c], E'[c]) \in L^* (c : S) \}
  \\
  L (c : \TIn TS) &= \{ (E[ \ERecv c ], E'[ \ERecv c ]) \mid \forall v, w. (v, w) \in V (T) \Rightarrow (E[(v,c)], E'[(w,c)]) \in L^* (c : S) \}
  \\
  L^* (c : S) &= \{ (e_1, e_2) \mid \forall e_1', e_2'. e_1 \to^* e_1', e_2 \to^* e_2' \Rightarrow (e_1', e_2') \in L (c : S) \}
\end{align*}

\subsection{Attempt at Vasco's suggestion}
\label{sec:attempt-at-vascos}

\begin{lemma}
  Suppose $\Ctxt \vdash e_1 \leadsto e'_1 : T$ where $\Ctxt$ contains
  only channel bindings.  If
  $\reduce[\alpha]{e_1}{e_2}$, then $\reducemany[\alpha]{e'_1}{e'_2}$
  and $\Ctxt \vdash e_2 \leadsto e'_2 : T$. \vv{Shouldn't it be $\Ctxt \vdash e_2 \leadsto e'_2 : T$?}\pt{yes!}
\end{lemma}
\begin{proof}
  Suppose the top-level translation rule is
  \begin{mathpar}
    \inferrule{
      \Ctxt \vdash e \leadsto e' : T \\
      f : T \eqt U
    }{
      \Ctxt \vdash e \leadsto f e' : U
    }
  \end{mathpar}
  By induction on the derivation of the translation judgment, we
  obtain
  $\reduce[\alpha]{e_1}{e_3}$ implies $\reducemany[\alpha]{e_1'}{e'_3}$,
  for all $\alpha$, and $\Ctxt \vdash e_1' \leadsto e'_3 : T$.

  As $f$ is a value, we have $\reducemany[\alpha]{f e_1'}{f e'_3}$.

  \vv{If expression reduction is deterministic (is it?), then we take
    $e_2 = e_3$ and $e'_2 = f e'_3$ and we are done. Otherwise expression
    reduction is Church-Rosser and we should be able to complete the proof with
    a little more effort. Does the difficulty lie in \emph{process} (rather than
    expression) reduction?}

  \pt{Agreed, this part works ok. But what follows troubles me.}

  However, if $e_1$ is a value, then so is $e_1'$ (lemma!). 

  Proceed by induction on $f : T \eqt U$?

  Or: prove that if I can observe some $\alpha$ on $e_1$, then I can
  observe the same $\alpha$ on $f e_1'$.

  \begin{itemize}
  \item $\alpha = \ELetU\EHole{e''}$
  \item $\alpha = \ELetP xy\EHole{e''}$
  \item $\alpha = \EHole\; v$
  \item $\alpha = \EWait\EHole$
  \item $\alpha = \ETerm\EHole$
  \item $\alpha = \ESend{v}\EHole$
  \item $\alpha = \ERecv\EHole$
  \item $\alpha = \ESelect l \EHole$
  \item $\alpha = \ECase\EHole{l_i: e_i}$
  \end{itemize}
  
\end{proof}

%%% Local Variables:
%%% mode: latex
%%% TeX-master: "main"
%%% End:

\section{Translating equi-recursion to iso-recursion and back}

\subsection{Forward translation}
\label{sec:forward}


\declrel{Translation from equi-recursive to iso-recursive expressions}[$\dtrans{e}=e$]
\begin{mathpar}
  \dtranslhs{ }{ \typing \EUnit \TUnit } = \EUnit \and
  \dtranslhs{ }{ \typing x T } = x

  \dtranslhs{
    \dtbind[\Ctxt,\CBind x T] e U
  }{
    \typing {\ELam x e} {\TFun T U}
  } = \ELam x \dtvar e

  \dtranslhs{
    \dtbind[\Ctxt,\CBind x T] e U
  }{
    \typing {\ERec x e} {\TFun T U}
  } = \ERec x \dtrans e

  \dtranslhs{
    \dtbind[\Ctxt_1] {e_1} {\TFun T U} \\
    \dtbind[\Ctxt_2] {e_2} T
  }{
    \typing[\CSplit] {e_1 \; e_2} U
  } = \dtvar{e_1} \; \dtvar{e_2}

  \dtranslhs{
    \dtbind[\Ctxt_1] {e_1} \TUnit \\
    \dtbind[\Ctxt_2] {e_2} T
  }{
    \typing[\CSplit] {\ELetU {e_1} {e_2}} T
  } = \ELetU {\dtvar{e_1}} {\dtvar{e_2}}

  \dtranslhs{
    \dtbind[\Ctxt_1] {e_1} T \\
    \dtbind[\Ctxt_2] {e_2} U
  }{
    \typing[\CSplit] {(e_1,e_2)} {\TPair T U}
  } = \left( \dtvar{e_1},\dtvar{e_2} \right)

  \dtranslhs{
    \dtbind[\Ctxt_1] {e_1} {\TPair T_1 T_2} \\
    \dtbind[\Ctxt_2, \CBind x T, \CBind y U] {e_2} U
  }{
    \typing[\CSplit]{\ELetP xy {e_1} {e_2}} U
  } = \ELetP xy {\dtvar{e_1}} {\dtvar{e_2}}

  \dtranslhs{
    \dtbind e {\TEnd ?}
  }{
    \typing {\EWait e} \TUnit
  } = \EWait {\dtvar e}

  \dtranslhs{
    \dtbind e {\TEnd !}
  }{
    \typing {\ETerm e} \TUnit
  } = \ETerm {\dtvar e}

  \dtranslhs{
    \dtbind {e_1} T \\
    \dtbind {e_2} {\TOut T S}
  }{
    \typing[\CSplit] {\ESend {e_1} {e_2}} S
  } = \ESend {\dtvar{e_1}} {\dtvar{e_2}}

  \dtranslhs{
    \dtbind e {\TIn T S}
  }{
    \typing {\ERecv e} {(T,S)}
  } = \ERecv {\dtvar e}

  \dtranslhs{
    \dtbind e {\TSelect{l_i : S_i}} \\
    j \in I
  }{
    \typing{\ESelect{l_j}{e}}{S_j}
  } = \ESelect {l_j} {\dtvar e}

  \dtranslhs{
    \dtbind[\Ctxt_1]{e}{\TCase{l_i : S_i}} \\
    \dtbind[\Ctxt_2]{e_i}{\TFun{S_i}{T}}
  }{
    \typing[\CSplit]{
      \ECase e { l_i \rightarrow e_i }
    }{T}
  } = \ECase {\dtvar e} { l_i \rightarrow \dtvar{e_i} }

  \dtranslhs{
    \dtbind e {\TFun {\TDual S} \TUnit}
  }{
    \typing{\EFork e}{S}
  } = \EFork {\dtvar e}

  \dtranslhs{
    \dtbind e T \\
    f : T \eqt U : f'
  }{
    \typing e U
  } = f \; \dtvar e
\end{mathpar}


The translation judgment $\translation e {e'} T$ in
Figure~\ref{fig:translation} describes the translation of an equi-recursively
typed expression $e$ to an expression $e'$ with the same type in the iso-recursive
system. The translation is mostly just copying. The only interesting part is where the
equi-recursive system makes use of type conversions. At this point, an appropriate
conversion function is inserted on the iso-recursive side. These
conversion expressions are specified by the blue parts in figure~\ref{fig:equi-equivalence}.
\begin{mathpar}
  \inferrule{
      \translation e {e'} T \\
      \convexpr{f} : T \eqt U
    }{
      \translation e {f \; e'} U
    }
\end{mathpar}

To prove that the translation preserves well-typedness, we
need a few properties of the conversion expression $f$.
\begin{lemma}
  \begin{enumerate}
  \item If $f : T \eqt U$, then $\exists f'$ such that  $f' : U \eqt
    T$ and $\typing[\CNil]{\tyIso f}{\TFun TU}$ and $\typing[\CNil]{\tyIso f'}{\TFun UT}$.
  \item If $g : R \eqs S$, then $\exists g'$ such that  $g' : S \eqs
    R$ and $\typing[\CNil]{\tyIso g}{\TFun{\TPair
      R {\TDual S}} \TUnit}$ and $\typing[\CNil]{\tyIso g'}{\TFun{\TPair
      S {\TDual R}} \TUnit}$.
  \end{enumerate}
\end{lemma}
% Technically, the expression $g$ generated by the coinductively defined
% judgment $g : R \eqs S$ can be infinitely big. However, the proof of
% such a judgment is a regular tree so that $g$ can be finitely
% represented by a recursive function. The details of turning regular
% coinduction into induction are well-known
% \cite{DBLP:conf/popl/HengleinN11,DBLP:journals/jfp/GapeyevLP02,DBLP:journals/lmcs/Dagnino21}
% and hence omitted.

\begin{lemma}
  \begin{enumerate}
  \item $\typing{\tyEqui e} T$ iff there is some $e'$ such that
    $\translation e {e'} T$.
  \item If $\translation e {e'} T$, then $\typing{\tyIso e'} T$.
  \end{enumerate}
\end{lemma}

\subsection{Backward translation}
\label{sec:backward-translation}

The reverse translation $\IsoToEqui\cdot$ from iso-recursion to equi-recursion is even
simpler. The translation just removes each occurrence of $\EUnroll\_$
as in $\IsoToEqui{\EUnroll e} = \IsoToEqui e$ and treats all other
expressions homomorphically. Formally, we have this result.
\begin{lemma}
  Suppose that $\typing{\tyIso e}T$ where each recursive type
  occurring in the derivation is contractive.
  Then $\typing{\tyEqui \IsoToEqui{e}}T$.
\end{lemma}
\begin{proof}
  By rule induction on the typing derivation of  $\typing{\tyIso
    e}T$. The only interesting case concerns the $\EUnroll\_$
  expression:
  \begin{mathpar}
    \inferrule{
      \typing{\tyIso e}{\TRec S}
    }{
      \typing{\tyIso \EUnroll e}{S[\TRec[X\!] S / X]}
    }
  \end{mathpar}
  By induction, we have $\typing{\tyEqui \IsoToEqui{e}}{\TRec S}$.
  From reflexivity of $\eqs$ (Lemma~\ref{lemma:congruence}) and rule
  {\EqUnrollL}, we obtain $\TRec S 
  \eqs S[\TRec[X\!] S / X]$ and thus  $\TRec S 
  \eqt S[\TRec[X\!] S / X]$, so that the conversion rule yields
  $\typing{\tyEqui \IsoToEqui{e}}{S[\TRec[X\!] S / X]}$.

  The translation does not change the structure of the types, so that
  contractiveness is preserved.
\end{proof}

\subsection{Wire compatibility by silencing unroll}
\label{sec:wire-compatibility}

Now that we established a route to go forth and back between
isorecursive and equirecursive session types, we wonder if
isorecursive and equirecursive programs can communicate with one
another. On first sight, the answer is ``no'' because the folklore
semantics of $\EUnroll\_$ involves communication, which renders a
program and its translation according to section~\ref{sec:forward}
or~\ref{sec:backward-translation} incompatible.

However, if we impose the restriction that recursive types are
generally contractive, we can modify the semantics of isorecursive
session types so as not to communicate $\EUnroll\_$ messages. In fact,
$\EUnroll\_$ becomes a no-op which just changes the type. In other
words, we downgrade the process reduction of $\EUnroll\_$ to a simple,
silent expression reduction:
\begin{gather*}
  \reduce{\EUnroll a}{a}
\end{gather*}

At this point, we should have that an equirecursive expression
$\translation e{e'} T$ is wire compatible to its translation
$e'$. However, this statement is not straightforward to state and
prove because the translation introduces explicit conversion
expressions. As conversion expressions on session types are
just forwarders of the respective messages, we realize that we need to
lift our point of view to the process level.

The translation from equirecursion to isorecursion extends easily to
the process level written as $\Ptranslation p{p'}$ by applying the
expression translation to each subprocess.

% \begin{definition}
%   For processes $p_1$ and $p_2$ such that  $\Ptyping{p_1}$ and
%   $\Ptyping{ p_2}$ define $p_1 = p_2$ 
%   as the reflexive, transitive, symmetric, congruence closure of all
%   silent reductions, and extensionality.
% \end{definition}

\begin{definition}[Simulation and bisimulation]
  For processes $p$ and $q$ define
  \begin{itemize}
  \item $p \leftsim q$ if, for all $\alpha$ and $p'$ such that
    $\reducemany[\alpha] p{p'}$, then there exists some process $q'$
    such that $\reducemany[\alpha] q{q'}$ and $p' \leftsim q'$;
  \item $p \bisim q$ if $p \leftsim q$ and $q \leftsim p$.
  \end{itemize}
\end{definition}

The first conjecture is needed to prove the second one. It essentially
says that the isomorphisms generated from the equivalence judgments
for types and session types deserve their name.

\begin{conjecture}[Conversion]~
  \begin{enumerate}
  \item
    If $f : T \eqt T'$ and $f' : T' \eqt T$ and $\typing eT$ and
    $\typing {E[e]}\TUnit$, then 
    $\PExp{E[e]} \bisim \PExp{E[f' (f\;e)]}$.
  \item If $g : S \eqs S'$
    and $g' : S' \eqs S$
    and $\Ptyping[\Ctxt, c : S]p$, then\\
    $p \bisim
    \PScope[c', \bar c']\PScope[c_1, \bar c]
    \PPar{\PPar {\PExp{g \; (c, \bar c')}}{\PExp{g' \; (c', \bar c)}}}{p[c_1/c]}$.
  \end{enumerate}
\end{conjecture}

\begin{conjecture}
  Suppose that $\translation {e}{f}T$.
  If $\reduce[\alpha]{e}{e'}$
  and $\translation{e'}{f'}T$,
  then $\exists f''$,  $\reducemany[\alpha]{f}{f''}$
  and $ \reducemany{f'}{f''}$.
\end{conjecture}
\begin{proof}
  Case $\RuleActAppLetL$.
  The translation does not change a non-value into a value, so that
  $\EApp{n'}{e'}$ reduces with the same rule.

  Case $\RuleActAppLetR$. If the translation maps $v$ to a value $v'$,
  then $\EApp{v'}{n'}$ reduces with the same rule.
  However, in general $\translation{v}{\EApp f{e'}}{T}$ if
  $\translation{v}{e'}T$ (where $f$ is a value), so we are looking at reducing $\EApp{\EApp
    f{e'}}{n'}$, a non-value, so that \textsc{\ActAppLetL} applies:
  \begin{gather*}
    \EApp{\EApp f{e'}}{n'}
    \to \ELet y{\EApp f{e'}}{\EApp y{n'}}
    \to^* \ELet y{[f{e'}]}{\EApp y{n'}}
    \to \ELet y{[f{e'}]}{\ELet {y'}{n'}\EApp y{y'}}
  \end{gather*}
  The reduced term translates to $\ELet x {n'} \EApp {\EApp f{e'}} x$
  which reduces as follows:
  \begin{gather*}
    \ELet x {n'} \EApp {\EApp f{e'}} x
    \to^* \ELet x {[n']} \EApp {\EApp f{e'}} x
    \to \ELet x {[n']} \ELet{x'}{\EApp f{e'}}\EApp {x'}x 
  \end{gather*}
\end{proof}

\begin{conjecture}
  Suppose that $\Ptranslation {p}{q}$.
  If $\reduce[\alpha]{p}{p'}$
  and $\Ptranslation{p'}{q'}$,
  then $\exists q''$,  $\reducemany[\alpha]{q}{q''}$
  and $ \reducemany{q'}{ q''}$.
\end{conjecture}


\subsection{Obsolete Stuff}
\label{sec:obsolete-stuff}



\begin{conjecture}[obsolete]
  Suppose that $\translation {e_1}{e_1'} T$.
  If $\reduce[\alpha]{e_1}{e_2}$
  and $\translation{e_2}{e_2''}{T}$,
  then $\reducemany[\alpha]{e_1'}{e_2'}$
  and $e_2' = e_2''$.
\end{conjecture}
If we could establish $\reducemany{e_2''}{e_2'}$ in  the
conclusion, that would result in a reduction correspondence.

The following lemma is a stepping stone for this conjecture. It
establishes that conversions constructed by $\eqt$ behave like
isomorphisms and conversion constructed by $\eqs$ like forwarders.
\begin{conjecture}[Conversions, obsolete]~\\[-\baselineskip]
  \label{lemma:conversion}
  \begin{enumerate}
  \item If $f : T \eqt T'$,
    then $f' : T' \eqt T$
    and $\typing[\CNil]{f}{\TFun T{T'}}$
    and  $\typing[\CNil]{f'}{\TFun {T'}T}$
    and  $f \circ f' = \ELam x x$
    and  $f' \circ f = \ELam x x$.
  \item If $g : S \eqs S'$,
    then $g' : S' \eqs S$
    and $\typing[\CNil]{g}{\TFun {\TPair {S} {\TDual{S'}}} \TUnit}$
    and $\typing[\CNil]{g'}{\TFun {\TPair {S'} {\TDual{S}}} \TUnit}$
    and,\\
    for all channels $c : S$,  $c \cong \ELet{c'}{\EFork{( \ELam {\bar c'} g \; (c, \bar c')) }}{\EFork{( \ELam {\bar c} g' \; (c', \bar c)) }} $.
  \end{enumerate}
\end{conjecture}
For session type $S$ and $c:S$, $c \cong c'$ means that for all
sequences of session operations $o$ described by $S$, $o\; c = o\; c'$.


%%% Local Variables:
%%% mode: latex
%%% TeX-master: "main"
%%% End:


%% to be ignored!

\section{Stuff}
\label{sec:stuff}

A function that returns an isorecursive channel.
\begin{align*}
  f &: \TFun\TInt{\TRec{\TIn \TInt X}}
  \\
  f &= \ELam x \EFork{(\ERec g \ELam {(c: \TRec{\TOut \TInt X})} g\; (\ESend x {(\EUnroll c)}))}
\end{align*}

%%% Local Variables:
%%% mode: latex
%%% TeX-master: "main"
%%% End:


\end{document}

%%% Local Variables:
%%% mode: latex
%%% TeX-master: t
%%% End:
