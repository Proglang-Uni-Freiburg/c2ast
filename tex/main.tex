\documentclass[adraft]{eptcs}

\usepackage[T1]{fontenc}
\usepackage{amsmath}
\usepackage{amssymb}
\usepackage{amsthm}
\usepackage{textcomp}
\usepackage{mathpartir}
\usepackage{mathtools}
\usepackage{microtype}
\usepackage{stmaryrd}
\usepackage{xcolor}
\usepackage{xparse}

\newtheorem{definition}{Definition}
\newtheorem{lemma}{Lemma}
\newtheorem{conjecture}{Conjecture}

\title{A silent semantics for isorecursive session types}
\newcommand\titlerunning{Isorecursive Sessions}
\author{Janek Spaderna \and Peter Thiemann \and Vasco Vasconcelos}
\newcommand\authorrunning{J. Spaderna, P. Thiemann, V. Vasconcelos}

\newcommand\note{}
\newcommand\note[2]{[\textcolor{red}{#1}: #2]}
\newcommand\js{\note{js}}
\newcommand\vv{\note{vv}}
\newcommand\pt{\note{pt}}
\newcommand\highlight[1]{\colorbox{yellow}{#1}}

% Macros to define type and term grammars
\newcommand\grmdef{\; \Coloneqq \;\;}
\newcommand\grmor{\; \mid \;}

% Formatting of type and expression keywords.
\newcommand\Tkw[1]{\mathtt{#1}}
\newcommand\Ekw[1]{\mathtt{#1}}
\newcommand\Econst[1]{\mathtt{#1}}

% Type syntax macros
\newcommand\TInt{\Tkw{Int}}
\newcommand\TUnit{()}
\newcommand\TPair[2]{#1 \otimes #2}
\newcommand\TFun[2]{#1 \rightarrow #2}
\newcommand\TEnd[1]{\Tkw{end #1}}
\newcommand\TIn[2]{{}?#1.#2}
\newcommand\TOut[2]{{}!#1.#2}
\newcommand\TSelect[2][i \in I]{\oplus\{\, #2 \,\}_{#1}}
\newcommand\TCase[2][i \in I]{\&\{\, #2 \,\}_{#1}}
\newcommand\TRec[1][X]{\mu #1.\;}
\newcommand\TDual[1]{\overline{#1}}

% Expression syntax macros
\newcommand\EUnit{()}
\newcommand\ELam[1]{\lambda #1.\;}
\newcommand\ERec[1]{\Ekw{rec}\; #1.\;}
\newcommand\ELet[2]{\Ekw{let}\; #1 = #2 \;\Ekw{in}\;}
\newcommand\ELetU[1]{#1;\;}%{\ELet{\EUnit}{#1}}
\newcommand\ELetP[3]{\ELet{(#1,#2)}{#3}}
\newcommand\EkwWait{\Econst{wait}}
\newcommand\EkwTerm{\Econst{term}}
\newcommand\EkwSend{\Econst{send}}
\newcommand\EkwRecv{\Econst{recv}}
\newcommand\EkwSelect{\Econst{sel}}
\newcommand\EkwRoll{\Econst{roll}}
\newcommand\EkwUnroll{\Econst{unroll}}
\newcommand\EkwFork{\Econst{fork}}
\newcommand\EWait[1]{\EkwWait\; #1}
\newcommand\ETerm[1]{\EkwTerm\; #1}
\newcommand\ESend[2]{\EkwSend\; #1 \; #2}
\newcommand\ERecv[1]{\EkwRecv\; #1}
\newcommand\ESelect[2]{\EkwSelect\; #1 \; #2}
\newcommand\ECase[2]{\Ekw{case}\; #1 \; \left\{\, #2 \,\right\}}
\newcommand\ERoll[1]{\EkwRoll\; #1}
\newcommand\EUnroll[1]{\EkwUnroll\; #1}
\newcommand\EFork[1]{\EkwFork\; #1}

\newcommand\EHole{[\;]}

% Type equivalence names
\newcommand\EqUnit{Eq-Unit}
\newcommand\EqPair{Eq-Pair}
\newcommand\EqFun{Eq-Fun}
\newcommand\EqS{Eq-S}
\newcommand\EqEnd[1]{Eq-End#1}
\newcommand\EqIn{Eq-In}
\newcommand\EqOut{Eq-Out}
\newcommand\EqSelect{Eq-Select}
\newcommand\EqCase{Eq-Case}
\newcommand\EqUnrollL{Eq-Unroll-L}
\newcommand\EqUnrollR{Eq-Unroll-R}
\newcommand\EqUnroll{Eq-Unroll}
\newcommand\EqRoll{Eq-Roll}

% Outputs a header for defining a new relation.
%   {#1} description/name
%   [#2] relation syntax, optional
\NewDocumentCommand \declrel
  { m o }
  {%
    \noindent%
    \emph{#1}%
    \IfValueT{#2}{\hfill\fbox{#2}}%
  }

% Outputs a rule for definitions of session type dualization.
%   {#1} type to be dualized
%   {#2} dualized result
\newcommand\dualdef[2]{\inferrule{\TDual{#1} {}={} #2}{}}

% Outputs a typing context, can be embellished with an index `i` like this
%
%     \Ctxt_i    or    \Ctxt_{longer_index}
\NewDocumentCommand \Ctxt
  { e{_} }
  { \IfNoValueTF{#1}{\Gamma}{\Gamma_{#1}} }

% The empty context.
\newcommand\CNil{\cdot}

% Typesets a single binding in a context
%
%   *       given => reusable binding, not given => linear binding
%   {#1}    binding name
%   {#2}    binding typing
\NewDocumentCommand \CBind
  { s m m }
  { #2 \IfBooleanTF{#1}{:^*}{:} #3 }

% A split context.
%
%   [#1]   1st sub-context, defaults to \Ctxt_1  
%   [#2]   2nd sub-context, defaults to \Ctxt_2  
\NewDocumentCommand \CSplit
  { O{\Ctxt_1} O{\Ctxt_2} }
  { #1 \circ #2 }

% Predicate on contexts: only unrrestricted bindings remaining
\newcommand\CExhausted[1][\Ctxt]{all^*(#1)}

% Macros for typing relations.
%
%   [#1]  typing context, defaults to  \Ctxt
%   {#2}  expression to type
%   {#3}  type
\newcommand\typing[3][\Ctxt]{#1 \vdash #2 : #3}

\newcommand\eqt{\sim^T}
\newcommand\eqs{\sim^S}

% Macros for reductions
\newcommand\reduce[2]{#1 \rightarrow #2}
\newcommand\reducerule[3][]{\inferrule{#1}{\reduce{#2}{#3}}}

\newcommand\PScope[1][a,b]{(\nu #1)}
\NewDocumentCommand \PPar
  { s m m }
  { \IfBooleanTF{#1}
    { \PParImpl {E_1[{#2}]} {E_2[{#3}]} }
    { \PParImpl {#2} {#3} } }
\newcommand\PParImpl[2]{#1 \| #2}

% MACROS FOR DERIVATION TRANSLATION
%
% The translation function in its most general form
\newcommand\dtrans[1]{\left\llbracket #1 \right\rrbracket}

% Set of derivation trees where
%
%   #1 |- #2 : #3
%
% Context defaults to \Ctxt
\newcommand\derivTree[3][\Ctxt]{\mathcal{T}\!\!\left(\typing[#1]{#2}{#3}\right)}

% How we call derivation tree variables.
\newcommand\dvarName[1]{#1^{\mathcal T}}

% Shortcut for calling \dtrans on a \dtreevar
\newcommand\dtvar[1]{\dtrans{\dvarName{#1}}}

% Shortcut for binding a \dtreevar
\newcommand\dtbind[3][\Ctxt]{\dvarName{#2} \in \derivTree[#1]{#2}{#3}}

% Typeset the the LHS of the equi => iso translation
%
%   {#1} premise
%   {#2} conclusion
\newcommand\dtranslhs[2]{\dtrans{\inferrule{#1}{#2}}}

%%% Local Variables:
%%% mode: latex
%%% TeX-master: "main"
%%% End:

\newcommand\RuleActAppLetL{%
\ltsrule \ActAppLetL {\EApp n e} \LLet {\ELet x n \EApp x e}
}
\newcommand\RuleActAppLetR{%
\ltsrule \ActAppLetR {\EApp v n} \LLet {\ELet x n \EApp v x}
}

%%% Local Variables:
%%% mode: latex
%%% TeX-master: "main"
%%% End:


\begin{document}

\maketitle
\begin{abstract}
  Almost all formalizations of recursive session types rely on
  equirecursion because the explicit isomorphism to roll/unroll the
  recursive type is expected to give rise to additional communication and
  synchronization. We examine this statement from first principles and
  find mild conditions under which these communications can be
  elided.
\end{abstract}


\section{Introduction}
\label{sec:introduction}

Recursion in session types is almost always interpreted as
equirecursion. That is, the two session types $\TRec{\TOut \TInt X} $
and $\TRec{\TOut \TInt {\TOut \TInt X}} $ are considered equal because
their unfoldings as infinite trees are equal. That means, servers with
these types are interchangeable without breaking the protocol and
functions that expect arguments of one type also expect arguments of
the other.

Using an isorecursive interpretation the two types are not equal
because unrolling of the recursion is a distinct communication
step. Hence, channels of these two types are no longer
interchangeable.

However, for these particular types we can define a pair of conversion
functions
$f : \TFun{\TRec{\TOut \TInt X}}{\TRec{\TOut \TInt {\TOut \TInt X}}}$
and
$f' : \TFun{\TRec{\TOut \TInt {\TOut \TInt X}}}{\TRec{\TOut \TInt X}}$
that mediate between the two types. For example
\begin{align*}
  f &=       \ELam {c_1} \EFork{ \ELam {c_2} \ERec g {
      \begin{array}[t]{l}
      \ELam{(c_1, c_2)} \\
      \ELet{(x_2,c_2)}{\ERecv{(\EUnroll{c_2})}} \\
      \ELet{c_1}{\ESend{x_2}{(\EUnroll{c_1})}} \\
      \ELet{(x_2,c_2)}{\ERecv{({c_2})}} \\
      \ELet{c_1}{\ESend{x_2}{(\EUnroll{c_1})}} \\
        g \; ({c_1}, {c_2}) }
      \end{array}}
\end{align*}

%%% Local Variables:
%%% mode: latex
%%% TeX-master: "main"
%%% End:

\section{Equi-recursion}
\label{sec:equi-recursion}
%\begin{figure}
  \declrel{Type equivalence}[$e : T \eqt T$]
  \begin{mathpar}
    \inferrule[\EqUnit]{
    }{
      \ELam x x :
      \TUnit \eqt \TUnit
    }

    \inferrule[\EqPair]{
      f_1 : T_1 \eqt T_1^\prime \\
      f_2 : T_2 \eqt T_2^\prime
    }{
      \ELam {(x_1,x_2)} (f_1 \; x_1, f_2 \; x_2) :
      \TPair {T_1} {T_2} \eqt \TPair {T_1^\prime} {T_2^\prime}
    }

    \inferrule[\EqFun]{
      f : U_1 \eqt T_1 \\
      g : T_2 \eqt U_2
    }{
      \ELam h {\ELam x {g(h(f x))}} :
      \TFun {T_1} {T_2} \eqt \TFun {U_1} {U_2}
    }

    \inferrule[\EqFun]{
      f_1 : T_1 \eqt T_1^\prime \\
      f_2 : T_2 \eqt T_2^\prime
    }{
      \ELam x f_2 \circ x \circ f_1^{-1} :
      \TFun {T_1} {T_2} \eqt \TFun {T_1^\prime} {T_2^\prime}
    }

    \inferrule[\EqS]{
      g : S_1 \eqs S_2
    }{
      \ELam {c_1} \EFork{ \ELam {c_2} g \; c_1 \; c_2 } :
      S_1 \eqt S_2
    }
  \end{mathpar}
  % 
  \declrel{Session type equivalence}[$e : S \eqs S$]\medskip\\
  $g : S_1 \eqs S_2$ gives rise to
  $g : \TFun {\TPair {S_1} {\TDual{S_2}}} \TUnit$
  and
  $g^{-1} : \TFun {\TPair {S_2} {\TDual{S_1}}} \TUnit$
  \begin{mathpar}
    \inferrule[\EqEnd !]{
    }{
      \ELam {(c_1, c_2)} \ELetU {\EWait {c_2}} {\ETerm {c_1}} :
      \TEnd ! \eqs \TEnd !
    }

    \inferrule[\EqEnd ?]{
    }{
      \ELam {(c_1, c_2)} \ELetU {\EWait {c_1}} {\ETerm {c_2}} :
      \TEnd ? \eqs \TEnd ?
    }

    \inferrule[\EqOut]{
      f : T \eqt U  \\
      g : R \eqs S
    }{
      \ELam {c_1}{
      \ELam {c_2}
        \ELetP {x} {c_2} {\ERecv{c_2}}
        \ELet {c_1} {\ESend {(f \; x)} {c_1}}
        g\; c_1\; c_2}
      : \TOut {T} {R} \eqs \TOut {U} {S}
    }

    \inferrule[\EqOut]{
      f : T_1 \eqt T_2  \\
      g : S_1 \eqs S_2
    }{
      \ELam {(c_1, c_2)}
        \ELetP {t_1} {c_2} {\ERecv{c_2}}
        \ELet {c_1} {\ESend {(f \; t_1)} {c_1}}
        g \; (c_1, c_2)
      : \TOut {T_1} {S_1} \eqs \TOut {T_2} {S_2}
    }

    \inferrule[\EqIn]{
      f : T_1 \eqt T_2  \\
      g : S_1 \eqs S_2
    }{
      \ELam {(c_1, c_2)}
        \ELetP {t_1} {c_1} {\ERecv{c_1}}
        \ELet {c_2} {\ESend {(f \; t_1)} {c_2}}
        g \; (c_1, c_2)
      : \TIn {T_1} {S_1} \eqs \TIn {T_2} {S_2}
    }

    \inferrule[\EqSelect]{
      g_i : S_i \eqs S_i^\prime
    }{
      \ELam {(c_1, c_2)}
        \ECase {c_2} { l_i \rightarrow \ELam {c_2} g_i \; (\ESelect {l_i} c_1, c_2) }
      : \TSelect { l_i : S_i } \eqs \TSelect { l_i : S_i^\prime }
    }

    \inferrule[\EqCase]{
      g_i : S_i \eqs S_i^\prime
    }{
      \ELam {(c_1, c_2)}
        \ECase {c_1} { l_i \rightarrow \ELam {c_1} g_i \; (c_1, \ESelect {l_i} c_2) }
      : \TCase { l_i : S_i } \eqs \TCase { l_i : S_i^\prime }
    }

    \inferrule[\EqUnroll]{
    }{
      \EkwUnroll : \TRec S \eqs S[\TRec S / X]
    }

    \inferrule[\EqRoll]{
    }{
      \EkwRoll : S[\TRec S / X] \eqs S
    }

    \inferrule[\EqUnrollL]{
      g : S_1[\TRec S_1 / X] \eqs S_2
    }{
      \TRec S_1 \eqs S_2
    }

    \inferrule[\EqUnrollR]{
      g : S_1 \eqs S_2[\TRec S_2 / X]
    }{
      S_1 \eqs \TRec S_2
    }
  \end{mathpar}
%  \label{fig:equi-equivalence}
%  \caption{Type equivalence in the equi-recursive system}
%\end{figure}

%%% Local Variables:
%%% mode: latex
%%% TeX-master: "main"
%%% End:


In the following we define a core language for equirecursive
functional session types inspired by GV
\cite{DBLP:journals/jfp/GayV10}, but much simplified. To ensure that
duality is well-defined,
recursion is restricted to session types and the payload type of a
communication is always closed \cite{DBLP:journals/corr/abs-2004-01322}.

Figure~\ref{fig:equi-equivalence} contains the definition of the type
language and duality along with equivalence for types $T \eqt T$  and session
types $S \eqs S$. The nonterminal $S$ for session types is indexed with a set
$\mathcal X$ of variables that may occur free in $S$. The $\mu$-binder
increases this set for its body. 
Type equivalence is inductive, but session type
equivalence is coinductive and equates 
session types up to unfolding of recursive types. This unfolding is
only well-defined if types are contractive, i.e., they must
not contain a type of the form $\TRec[X_1]\dots\TRec[X_n]X_1$.
The figure also
specifies evidence terms for equivalence (in light blue), which we
ignore for now.

\begin{lemma}\label{lemma:congruence}
 The relations $\eqt$ and $\eqs$ are congruence relations.
\end{lemma}

\declrel{Syntax for constants, values, expressions and processes}
\begin{align*}
  v \grmdef&
    c                        \grmor
    (v, v)                   \grmor
    \ELam x e                \grmor
    \ERec x v                \grmor
  \\
  e \grmdef&
    v                       \grmor
    x                       \grmor
    e \; e                  \grmor
    \ELetU e e              \grmor
    (e, e)                  \grmor
    \ELetP x y e e          \grmor
  %\\ &
    \ESelect l e            \grmor
    \ECase e { l_i \rightarrow e } \grmor
    \EFork e
  \\
  p \grmdef&
    e                       \grmor
    \PPar p p               \grmor
    \PScope p
\end{align*}

The expression $\ELam {(x,y)} e$ abbreviates $\ELam z \ELetP xy z e$
for some variable $z$ not free in $e$.


%%% Local Variables:
%%% mode: latex
%%% TeX-master: "main"
%%% End:


\subsection{Operational semantics}

\declrel{Structural congruence of processes}[$p \equiv p$]\medskip\\
The structural congruence relation on processes is defined as the smallest
congruence relation that includes the commutative monoidal rules with the
binary operator being parallel process composition $\PPar \_ \_$ and
value~$\EUnit$ as the neutral element, scope extrusion, and swapping
of binders:
\begin{align*}
  \PPar{\PScope p}{q} &\equiv \PScope (\PPar p q)
  \quad\text{if $a,b$ not free in $q$}
  \\
  \PScope p &\equiv \PScope[b,a] p
\end{align*}

\declrel{Evaluation contexts}
\begin{align*}
  E \grmdef&
    \EHole \grmor
    E \; e \grmor
    v \; E \grmor
    \ELetU E e \grmor
    (E,e) \grmor
    (v,E) \grmor
    \ELetP xy E e \grmor
  %\\ &
    % \ESelect l E \grmor 
    \ECase E { l_i \rightarrow e_i }
\end{align*}

\declrel{Transition labels}
\begin{align*}
  \sigma \grmdef&
    !v \grmor ?v \grmor !l \grmor ?l \grmor ! \grmor ?
  && \text{label suffixes for session operations} \\
  \lambda \grmdef&
    a\sigma \grmor \beta \grmor \PScope[a,a] \EFork v
  &&\text{labels for expression reduction} \\
  \pi \grmdef&
               a\sigma \grmor \tau \grmor \PScope[a,a] a!a \RPar \pi
               \pi
  &&\text{labels for process reduction}
\end{align*}

\declrel{Free variables of process labels}[$\FV{p}$]
\begin{mathpar}
  \FV{a!v} = \FV{a?v} = \{a\} \cup \FV{v} \and
  \FV{a!l} = \FV{a?l} = \FV{a!} = \FV{a?} = \{a\} \and
  \FV{\tau} = \emptyset \and
  \FV{\PScope[c,d] a!x} = \{a,x\} \setminus \{c,d\} \and
  \FV{\RPar {\pi_1} {\pi_2}} = \FV{\pi_1} \cup \FV{\pi_2}
\end{mathpar}

\declrel{Labeled transition system for expressions}[$\reduce[\lambda] e e$]
\begin{mathpar}
  \ltsrule \ActApp { (\ELam x e) v } \beta { e[v / x] }

  \ltsrule \ActRec { (\ERec x v) v' } \beta { v[\ERec x v / x] v' }

  \ltsrule \ActLet { \ELetP xy {(v_1,v_2)} e } \beta { e[v_1/x, v_2/y] }

  \ltsrule \ActUnit { \ELetU \EUnit e } \beta { e }

  \ltsrule \ActSend { \ESend v a } {a!v} { a }

  \ltsrule \ActRecv { \ERecv a } {a?v} { (v,a) }

  \ltsrule \ActSelect { \ESelect l a } {a!l} { a }

  \ltsrule \ActCase { \ECase a { l_i \rightarrow v_i } } {a?l_j} { v_{\!j} \, a }

  \ltsrule \ActWait { \EWait a } {a?} { \EUnit }

  \ltsrule \ActTerm { \ETerm a } {a!} { \EUnit }

  \ltsrule \ActFork { \EFork v } { \PScope \EFork v } { a }
\end{mathpar}
The standard structural call-by-value rules are omitted.

\declrel{Labeled transition system for processes}[$\reduce[\pi] p p$]
\begin{mathpar}
  \ltsrule \ActSession [
    \reduce[\sigma]{ e }{ e' }
  ]{ \PExp e } \sigma { \PExp {e'} }

  \ltsrule \ActBeta [
    \reduce[\beta]{ e }{ e' }
  ]{ \PExp e } \tau { \PExp {e'} }

  \ltsrule \ActForkP [
    \reduce[ \PScope \EFork v ]{ e }{ e' }
  ]{ \PExp e } \tau { \PScope (\PPar {\PExp{e'}} {\PExp{v\;b}}) }

  \ltsrule \ActJoin [
    \reduce[\pi_1]{p}{p'} \\
    \reduce[\pi_2]{q}{q'} \\
  ]{ \PPar p q } {\RPar{\pi_1\!}{\pi_2}} { \PPar {p'} {q'} }

  \ltsrule \ActMsg [
    \reduce[\RPar{a!v}{b?v}]{p}{p'}
  ]{ \PScope p } \tau { \PScope p' }

  \ltsrule \ActBranch [
    \reduce[\RPar{a!l}{b?l}]{p}{p'}
  ]{ \PScope p } \tau { \PScope p' }

  \ltsrule \ActEnd [
    \reduce[\RPar{a!}{b?}]{p}{p'}
  ]{ \PScope p } \tau { p' }

  \ltsrule \ActScope [
    \reduce[\pi]{p}{p'} \\
    a,b \not\in \FV{\pi}
  ]{ \PScope p } \pi { \PScope p' }

  \ltsrule \ActPar [
    \reduce[\pi]{p}{p'}
  ]{ \PPar p q } \pi { \PPar {p'} q }

 \ltsrule \ActOpen [
   \reduce[a!c]{p}{p'} \\
   a \not\in \{c,d\}
 ]{ \PScope[c,d] p } { \PScope[c,d] a!c } { p' }

 \ltsrule \ActClose [
   \reduce[\RPar{\PScope[c,d] a!c}{b?c}]{p}{p'}
 ]{ \PScope[a,b] p } \tau { \PScope[a,b] \PScope[c,d] p' }

  \ltsrule \ActCong [
    p \equiv q \\
    \reduce[\pi]{q}{q'}
  ]{ p } \pi { q' }
\end{mathpar}
\js{I'm unclear about why/if \textsc{\ActOpen} and \textsc{\ActClose} are needed.}

%%% Local Variables:
%%% mode: latex
%%% TeX-master: "main"
%%% End:


The expression reductions implement a standard call-by-value lambda
calculus. Process reductions implement a synchronous semantics for
session types.

The typing rules are standard (see \dots), but we choose to make
equirecursion explicit by adopting a conversion rule, which is not
syntax directed.
\begin{mathpar}
  \inferrule{
    \typing {\tyEqui e} T \\ T \eqt U
  }{
    \typing {\tyEqui e} U
  }
\end{mathpar}

We omit the standard metatheoretical results.
%  \cite{metatheory} % needed?
%\subsection{Types}

% \declrel{Type syntax}
% \begin{align*}
%   T \grmdef&
%     S^\emptyset               \grmor
%     \TUnit          \grmor
%     \TPair T T      \grmor
%     \TFun  T T      \\
%   S^{\mathcal X} \grmdef&
%     \TEnd !         \grmor
%     \TEnd ?         \grmor
%     \TOut T S^{\mathcal X}       \grmor
%     \TIn  T S^{\mathcal X}       \grmor
%     \TSelect{ l_i: S^{\mathcal X}_i } \grmor
%     \TCase{ l_i: S^{\mathcal X}_i }   \grmor
%     \TRec[X] S^{\mathcal X \cup \{X\}}      \grmor
%     X^{\in \mathcal X}
% \end{align*}
% 
% \declrel{Dualization of session types}[$\TDual S = S$]
% \begin{align*}
%   \TDual X &= X                               &
%   \TDual{\TEnd !} &= \TEnd ?                  &
%   \TDual{\TOut T S} &= \TIn T \TDual{S}       &
%   \TDual{\TSelect{ l_i: S_i }} &=
%     \TCase{ l_i: \TDual{S_i} }                \\
%   \TDual{\TRec S} &= \TRec{\TDual S}          &
%   \TDual{\TEnd ?} &= \TEnd !                  &
%   \TDual{\TIn T S} &= \TOut T \TDual{S}       &
%   \TDual{\TCase{ l_i: S_i }} &=
%     \TSelect{ l_i: \TDual{S_i} }
% \end{align*}

%%\begin{figure}
  \declrel{Type equivalence}[$e : T \eqt T$]
  \begin{mathpar}
    \inferrule[\EqUnit]{
    }{
      \ELam x x :
      \TUnit \eqt \TUnit
    }

    \inferrule[\EqPair]{
      f_1 : T_1 \eqt T_1^\prime \\
      f_2 : T_2 \eqt T_2^\prime
    }{
      \ELam {(x_1,x_2)} (f_1 \; x_1, f_2 \; x_2) :
      \TPair {T_1} {T_2} \eqt \TPair {T_1^\prime} {T_2^\prime}
    }

    \inferrule[\EqFun]{
      f : U_1 \eqt T_1 \\
      g : T_2 \eqt U_2
    }{
      \ELam h {\ELam x {g(h(f x))}} :
      \TFun {T_1} {T_2} \eqt \TFun {U_1} {U_2}
    }

    \inferrule[\EqFun]{
      f_1 : T_1 \eqt T_1^\prime \\
      f_2 : T_2 \eqt T_2^\prime
    }{
      \ELam x f_2 \circ x \circ f_1^{-1} :
      \TFun {T_1} {T_2} \eqt \TFun {T_1^\prime} {T_2^\prime}
    }

    \inferrule[\EqS]{
      g : S_1 \eqs S_2
    }{
      \ELam {c_1} \EFork{ \ELam {c_2} g \; c_1 \; c_2 } :
      S_1 \eqt S_2
    }
  \end{mathpar}
  % 
  \declrel{Session type equivalence}[$e : S \eqs S$]\medskip\\
  $g : S_1 \eqs S_2$ gives rise to
  $g : \TFun {\TPair {S_1} {\TDual{S_2}}} \TUnit$
  and
  $g^{-1} : \TFun {\TPair {S_2} {\TDual{S_1}}} \TUnit$
  \begin{mathpar}
    \inferrule[\EqEnd !]{
    }{
      \ELam {(c_1, c_2)} \ELetU {\EWait {c_2}} {\ETerm {c_1}} :
      \TEnd ! \eqs \TEnd !
    }

    \inferrule[\EqEnd ?]{
    }{
      \ELam {(c_1, c_2)} \ELetU {\EWait {c_1}} {\ETerm {c_2}} :
      \TEnd ? \eqs \TEnd ?
    }

    \inferrule[\EqOut]{
      f : T \eqt U  \\
      g : R \eqs S
    }{
      \ELam {c_1}{
      \ELam {c_2}
        \ELetP {x} {c_2} {\ERecv{c_2}}
        \ELet {c_1} {\ESend {(f \; x)} {c_1}}
        g\; c_1\; c_2}
      : \TOut {T} {R} \eqs \TOut {U} {S}
    }

    \inferrule[\EqOut]{
      f : T_1 \eqt T_2  \\
      g : S_1 \eqs S_2
    }{
      \ELam {(c_1, c_2)}
        \ELetP {t_1} {c_2} {\ERecv{c_2}}
        \ELet {c_1} {\ESend {(f \; t_1)} {c_1}}
        g \; (c_1, c_2)
      : \TOut {T_1} {S_1} \eqs \TOut {T_2} {S_2}
    }

    \inferrule[\EqIn]{
      f : T_1 \eqt T_2  \\
      g : S_1 \eqs S_2
    }{
      \ELam {(c_1, c_2)}
        \ELetP {t_1} {c_1} {\ERecv{c_1}}
        \ELet {c_2} {\ESend {(f \; t_1)} {c_2}}
        g \; (c_1, c_2)
      : \TIn {T_1} {S_1} \eqs \TIn {T_2} {S_2}
    }

    \inferrule[\EqSelect]{
      g_i : S_i \eqs S_i^\prime
    }{
      \ELam {(c_1, c_2)}
        \ECase {c_2} { l_i \rightarrow \ELam {c_2} g_i \; (\ESelect {l_i} c_1, c_2) }
      : \TSelect { l_i : S_i } \eqs \TSelect { l_i : S_i^\prime }
    }

    \inferrule[\EqCase]{
      g_i : S_i \eqs S_i^\prime
    }{
      \ELam {(c_1, c_2)}
        \ECase {c_1} { l_i \rightarrow \ELam {c_1} g_i \; (c_1, \ESelect {l_i} c_2) }
      : \TCase { l_i : S_i } \eqs \TCase { l_i : S_i^\prime }
    }

    \inferrule[\EqUnroll]{
    }{
      \EkwUnroll : \TRec S \eqs S[\TRec S / X]
    }

    \inferrule[\EqRoll]{
    }{
      \EkwRoll : S[\TRec S / X] \eqs S
    }

    \inferrule[\EqUnrollL]{
      g : S_1[\TRec S_1 / X] \eqs S_2
    }{
      \TRec S_1 \eqs S_2
    }

    \inferrule[\EqUnrollR]{
      g : S_1 \eqs S_2[\TRec S_2 / X]
    }{
      S_1 \eqs \TRec S_2
    }
  \end{mathpar}
%  \label{fig:equi-equivalence}
%  \caption{Type equivalence in the equi-recursive system}
%\end{figure}

%%% Local Variables:
%%% mode: latex
%%% TeX-master: "main"
%%% End:


%\subsection{Expressions}

% \declrel{Syntax for values, expressions and processes}
% \begin{align*}
%   c \grmdef&
%     \EkwSend \grmor \EkwRecv \grmor
%     \EkwTerm \grmor \EkwWait \grmor
%     \EkwRoll \grmor \EkwUnroll
%   \\
%   v \grmdef&
%     c                        \grmor
%     \EUnit                   \grmor
%     (v, v)                   \grmor
%     \ELam x e                \grmor
%     \ERec x e                \grmor
%   \\
%   e \grmdef&
%     v                       \grmor
%     x                       \grmor
%     e \; e                  \grmor
%     \ELetU e e              \grmor
%     (e, e)                  \grmor
%     \ELetP x y e e          \grmor
%   %\\ &
%     \ESelect l e            \grmor
%     \ECase e { l_i \rightarrow e } \grmor
%     \EFork e
%   \\
%   p \grmdef&
%     e                       \grmor
%     \PPar p p               \grmor
%     \PScope p
% \end{align*}


%\subsubsection{Expression typing}

% \declrel{Typing contexts}
% \begin{align*}
%   \Ctxt \grmdef&
%     \CNil \grmor \Ctxt, \CBind x T \grmor \Ctxt, \CBind* x T
% \end{align*}
% 
% \declrel{Context exhaustion}[$\CExhausted$]
% \begin{mathpar}
%   \inferrule{ }{\CExhausted[\CNil]} \and
%   \inferrule{\CExhausted}{\CExhausted[\Ctxt, \CBind* x T]}
% \end{mathpar}
% 
% \declrel{Context splitting}[$\Ctxt = \CSplit[\Ctxt][\Ctxt]$]
% \begin{mathpar}
%   \inferrule{}{\cdot = \CSplit[\cdot][\cdot]} \and
%   \inferrule{\Ctxt = \CSplit}{\Ctxt, \CBind  x T = \CSplit[(\Ctxt_1 , \CBind  x T)][ \Ctxt_2               ]} \and
%   \inferrule{\Ctxt = \CSplit}{\Ctxt, \CBind  x T = \CSplit[ \Ctxt_ 1              ][(\Ctxt_2 , \CBind  x T)]} \and
%   \inferrule{\Ctxt = \CSplit}{\Ctxt, \CBind* x T = \CSplit[(\Ctxt_1 , \CBind* x T)][(\Ctxt_2 , \CBind* x T)]}
% \end{mathpar}

%% \begin{figure}
  \declrel{Typing rules for expressions in the equi-recursive system}[$\typingE{e}{T}$]
  \begin{mathpar}
    \inferrule{ }{\typingE{\EUnit}{\TUnit}}

    \inferrule{
      \CExhausted
    }{
      \typingE[\Ctxt, \CBind x T]{x}{T}
    }

    \inferrule{
      \CExhausted
    }{
      \typingE[\Ctxt, \CBind* x T]{x}{T}
    }

    \inferrule{
      \CExhausted \\
      \typingE[\Ctxt, \CBind* x {\TFun {T_1} {T_2}}]{e}{\TFun {T_1} {T_2}}
    }{
      \typingE{\ERec x e}{\TFun{T_1}{T_2}}
    }

    \inferrule{
      \typingE[\Ctxt, \CBind x {T_1}]{e}{T_2}
    }{
      \typingE{\ELam x e}{\TFun{T_1}{T_2}}
    }

    \inferrule{
      \typingE[\Ctxt_1]{e_1}{\TFun {T_1}{T_2}}
      \\
      \typingE[\Ctxt_2]{e_2}{T_1}
    }{
      \typingE[\CSplit]{e_1 e_2}{T_2}
    }

    \inferrule{
      \typingE[\Ctxt_1]{e_1}{T_1} \\
      \typingE[\Ctxt_2]{e_2}{T_2}
    }{
      \typingE[\CSplit]{
        \ELetU{e_1}{e_2}
      }{ T_2 }
    }

    \inferrule{
      \typingE[\Ctxt_1]{e_1}{T_1} \\
      \typingE[\Ctxt_2]{e_2}{T_2}
    }{
      \typingE[\CSplit]{(e_1,e_2)}{\TPair{T_1}{T_2}}
    }

    \inferrule{
      \typingE[\Ctxt_1]{e_1}{\TPair{T_1}{T_2}} \\
      \typingE[\Ctxt_2, \CBind x {T_1}, \CBind y {T_2}]{e_2}{T_3}
    }{
      \typingE[\CSplit]{
        \ELetP xy {e_1} {e_2}
      }{ T_3 }
    }

    \inferrule{
      \typingE{e}{\TEnd ?}
    }{
      \typingE{\EWait e}{\TUnit}
    }

    \inferrule{
      \typingE{e}{\TEnd !}
    }{
      \typingE{\ETerm e}{\TUnit}
    }

    \inferrule{
      \typingE[\Ctxt_1]{e_1} T  \\
      \typingE[\Ctxt_2]{e_2} {\TOut T S}
    }{
      \typingE[\CSplit]{\ESend{e_1}{e_2}} S
    }

    \inferrule{
      \typingE{e_1} {\TIn T S}
    }{
      \typingE{\ERecv{e_1}}{\TPair T S}
    }

    \inferrule{
      \typingE{e}{\TSelect{l_i : S_i}} \\
      j \in I
    }{
      \typingE{\ESelect{l_j}{e}}{S_j}
    }

    \inferrule{
      \typingE[\Ctxt_1]{e}{\TCase{l_i : S_i}} \\
      \typingE[\Ctxt_2]{e_i}{\TFun{S_i}{T}}
    }{
      \typingE[\CSplit]{
        \ECase e { l_i \rightarrow e_i }
      }{T}
    }

    \inferrule{
      \typingE{e}{\TFun {\TDual S} \TUnit}
    }{
      \typingE{\EFork e}{S}
    }
  \end{mathpar}
%   \label{fig:equi-typing-rules}
%   \caption{Typing rules for expressions in the equi-recursive system}
% \end{figure}

%%% Local Variables:
%%% mode: latex
%%% TeX-master: "main"
%%% End:



%\subsubsection{Operational semantics}

% \declrel{Structural congruence of processes}[$p \equiv p$]\medskip\\
% The structural congruence relation on processes is defined as the smallest
% congruence relation that includes the commutative monoidal rules with the
% binary operator being parallel process composition $\PPar \_ \_$ and
% value~$\EUnit$ as the neutral element, and scope extrusion:
% \begin{align*}
%   \PPar{\PScope p}{q} &\equiv \PScope (\PPar p q)
%   \quad\text{if $a,b$ not free in $q$}
%   \\
%   \PScope p &\equiv \PScope[b,a] p
% \end{align*}
% 
% \declrel{Evaluation contexts}
% \begin{align*}
%   E \grmdef&
%     \EHole \grmor
%     E \; e \grmor
%     v \; E \grmor
%     \ELetU E e \grmor
%     (E,e) \grmor
%     (v,E) \grmor
%     \ELetP xy E e \grmor
%   %\\ &
%     \ESelect l E \grmor 
%     \ECase E { l_i \rightarrow e_i }
% \end{align*}
% 
% %\begin{figure}
  \declrel{Reduction relation in the equi-recursive system}[$\reduceE{p}{p}$]
  \begin{mathpar}
    \reduceruleE {
      (\ELam x e) v
    }{
      e[ v / a ]
    }

    \reduceruleE {
      (\ERec x e) v
    }{
      e[ \ERec x e / x ] \; v
    }

    \reduceruleE {
      \ELetU \EUnit e
    }{
      e
    }

    \reduceruleE {
      \ELetP {x_1} {x_2} {(v_1,v_2)} e
    }{
      e[v_1 / x_1][v_2 / x_2]
    }

    \reduceruleE[\reduceE{e_1}{e_2}]{
      E[e_1]
    }{
      E[e_2]
    }

    \reduceruleE{
      \PScope (\PPar* {\ESend v a} {\ERecv b} )
    }{
      \PScope (\PPar* {a} {(v,b)})
    }

    \reduceruleE{
      \PScope (\PPar* {\ESelect {l_j} a} {\ECase b { l_i \rightarrow e_i }})
    }{
      \PScope (\PPar* {a} {e_j \; b})
    }

    \reduceruleE{
      \PScope (\PPar* {\ETerm a} {\EWait b})
    }{
      \PPar* \EUnit \EUnit
    }

    \reduceruleE{
      E[\EFork e]
    }{
      \PScope (\PPar {E[a]} {e \; b})
    }

    \reduceruleE[\reduceE{p}{p^\prime}] { \PPar p q } { \PPar {p^\prime} q } \and
    \reduceruleE[\reduceE{p}{p^\prime}] { \PScope p } { \PScope p^\prime }   \and
    \reduceruleE[p \equiv q \\ \reduceE{q}{q^\prime}] { p } { q^\prime }
  \end{mathpar}
  Dual $\PScope[b,a]$ rules for $\EkwSend$/$\EkwRecv$, $\EkwSelect$/$\EkwCase$,
  $\EkwTerm$/$\EkwWait$ are omitted.
%   \label{fig:equi-reduction}
%   \caption{Reduction relation for the equi-recursive system}
% \end{figure}


%%% Local Variables:
%%% mode: latex
%%% TeX-master: "main"
%%% End:

\section{Iso-recursion}
\label{sec:iso-recursion}

The traditional equi-recursive system can---with very few and localized changes---be turned
into an iso-recursive system. The type syntax and dualization of
session types require no adjustments.

We extend the syntax with an extra expression to unroll a recursive
type:
\begin{align*}
    e \grmdef&
               \EUnroll e
\end{align*}

Extra process reduction rules
\begin{mathpar}
    % \reducerule{
    %   \PScope (\PPar* {\ERoll a} {\ERoll b})
    % }{
    %   \PPar* a b
    % }
    %   
    \reducerule{
      \PScope (\PPar* {\EUnroll a} {\EUnroll b})
    }{
      \PPar* a b
    }
\end{mathpar}

We replace the typing rule for conversion by a rule for explicit
unrolling.
\begin{mathpar}
    \inferrule{
      \typing{e}{\TRec S}
    }{
      \typing{\EUnroll e}{S[\TRec[X] S / X]}
    }
\end{mathpar}

The equi-recursive nature of this adjusted system handles any necessary
(un)rolling of recursive session types. This behavior is captured in the type
equivalence relation defined in Figure~\ref{fig:equi-equivalence}. Equivalence
$f : T \eqt T' : f'$ gives rise to two functions $f : \TFun {T} {T'}$, $f' :
\TFun {T'} {T}$. Equivalence of session types $g : S \eqs S' : g'$ gives rise
to two functions $g : \TFun {\TPair {S} {\TDual{S'}}} \TUnit$ and $g' :
\TFun {\TPair {S'} {\TDual{S}}} \TUnit$. All of these functions are
expressions in the iso-recursive system.

At the same time, the equi-recursive nature obviates the need for $\EkwRoll$
and $\EkwUnroll$ operations. These two values, their two typing rules and
reductions are removed. Instead, a rule is added which uses the equi-recursive
type equivalence.

\subsection{Types}


\declrel{Dualization of session types}[$\TDual S = S$]
\begin{align*}
  \TDual X &= X                               &
  \TDual{\TEnd !} &= \TEnd ?                  &
  \TDual{\TOut T S} &= \TIn T \TDual{S}       &
  \TDual{\TSelect{ l_i: S_i }} &=
    \TCase{ l_i: \TDual{S_i} }                \\
  \TDual{\TRec S} &= \TRec{\TDual S}          &
  \TDual{\TEnd ?} &= \TEnd !                  &
  \TDual{\TIn T S} &= \TOut T \TDual{S}       &
  \TDual{\TCase{ l_i: S_i }} &=
    \TSelect{ l_i: \TDual{S_i} }
\end{align*}

% The type syntax for the iso-recursive system is the same as in the
% equi-recursive system, including the notion of session type dualization.
% 
% Type equivalence for the iso-recursive system is largely the same as in the
% equi-recursive system except for the rules \textsc{\EqUnrollL} and
% \textsc{\EqUnrollR}, which do not exist.

\vv{
  \begin{itemize}
  \item The dual function for recursive types is a bit more complicated; pls have
    a look at ``the final cut''.
    \pt{restricted payload type to closed types}
  \item Iso-recursive expressions $e$ must appear before type equivalence.
    \js{fixed this by moving sections around}
  \item IMO, $T,U$ reads better than $T_1, T_2$.
    \pt{sorry, messed this up!}
    \js{changed it}
  \item The expression for session types isomorphism may be curried. Then we
    seek these results, right? (do they hold?)

    \pt{Yes, see below in Sec~\ref{sec:properties}}

    Lemma. If $e : T \eqt U$, then $\typing{e}{\TFun{T}{U}}$.

    Lemma. If $e : R \eqs S$, then $\typing{e}{\TFun{R}{\TFun{\TDual S}{\TUnit}}}$.
  \item $f^{-1}$ is confusing; why not a simple $g$?
    \pt{Fixed, differently}
\item Added an easier to read, IMO, rule for \EqFun{} and (curried)
  for \EqOut
  \pt{Messed this up. Sorry again}
\item \EqPair{} uses an abbreviation, right? $\lambda p. \text{let}(x,y) = p
  \text{ in } \dots$
  \js{added note about this abbreviation next to grammar}
\item Added suggestion for roll and unroll; these constants may have function
  types. (types don't work :(
  \end{itemize}
}


\subsection{Expressions}

% The term syntax of the iso-recursive system is the same as for the
% equi-recursive system with two additional values.
% \begin{align*}
%   v \grmdef& \dots \grmor \EkwRoll \grmor \EkwUnroll
% \end{align*}


\subsubsection{Expression typing}

\declrel{Typing contexts}
\begin{align*}
  \Ctxt \grmdef&
    \CNil \grmor \Ctxt, \CBind x T \grmor \Ctxt, \CBind* x T
\end{align*}

\declrel{Context exhaustion}[$\CExhausted$]
\begin{mathpar}
  \inferrule{ }{\CExhausted[\CNil]} \and
  \inferrule{\CExhausted}{\CExhausted[\Ctxt, \CBind* x T]}
\end{mathpar}

\declrel{Context splitting}[$\Ctxt = \CSplit[\Ctxt][\Ctxt]$]
\begin{mathpar}
  \inferrule{}{\cdot = \CSplit[\cdot][\cdot]} \and
  \inferrule{\Ctxt = \CSplit}{\Ctxt, \CBind  x T = \CSplit[(\Ctxt_1 , \CBind  x T)][ \Ctxt_2               ]} \and
  \inferrule{\Ctxt = \CSplit}{\Ctxt, \CBind  x T = \CSplit[ \Ctxt_ 1              ][(\Ctxt_2 , \CBind  x T)]} \and
  \inferrule{\Ctxt = \CSplit}{\Ctxt, \CBind* x T = \CSplit[(\Ctxt_1 , \CBind* x T)][(\Ctxt_2 , \CBind* x T)]}
\end{mathpar}

% \begin{figure}
  \declrel{Typing rules for expressions in the iso-recursive system}[$\typing{e}{T}$]
  \begin{mathpar}
    \inferrule{}{
      \typing[\Ctxt^*]{\EUnit}{\TUnit}
    }

    \inferrule{}{
      \typing[\Ctxt^*, \CBind x T]{x}{T}
    }

    \inferrule{}{
      \typing[\Ctxt^*, \CBind* x T]{x}{T}
    }

    \inferrule{
      % \CExhausted \\
      \typing[\Ctxt^*, \CBind* x {\TFun T U}]{v}{\TFun T U}
    }{
      \typing[\Ctxt^*]{\ERec x v}{\TFun{T}{U}}
    }

    \inferrule{
      \typing[\Ctxt, \CBind x T]{e}{U}
    }{
      \typing{\ELam x e}{\TFun T U}
    }

    \inferrule{
      \typing[\Ctxt_1]{e_1}{\TFun T U}
      \\
      \typing[\Ctxt_2]{e_2}{T}
    }{
      \typing[\CSplit]{e_1 e_2}{U}
    }

    \inferrule{
      \typing[\Ctxt_1]{e_1}{\TUnit} \\
      \typing[\Ctxt_2]{e_2}{T}
    }{
      \typing[\CSplit]{
        \ELetU{e_1}{e_2}
      }{ T }
    }

    \inferrule{
      \typing[\Ctxt_1]{e_1}{T} \\
      \typing[\Ctxt_2]{e_2}{U}
    }{
      \typing[\CSplit]{(e_1,e_2)}{\TPair T U}
    }

    \inferrule{
      \typing[\Ctxt_1]{e_1}{\TPair{T_1}{T_2}} \\
      \typing[\Ctxt_2, \CBind x {T_1}, \CBind y {T_2}]{e_2}{U}
    }{
      \typing[\CSplit]{
        \ELetP xy {e_1} {e_2}
      }{ U }
    }

    \inferrule{
      \typing{e}{\TEnd ?}
    }{
      \typing{\EWait e}{\TUnit}
    }

    \inferrule{
      \typing{e}{\TEnd !}
    }{
      \typing{\ETerm e}{\TUnit}
    }

    \inferrule{
      \typing[\Ctxt_1]{e_1} T  \\
      \typing[\Ctxt_2]{e_2} {\TOut T S}
    }{
      \typing[\CSplit]{\ESend{e_1}{e_2}} S
    }

    \inferrule{
      \typing{e_1} {\TIn T S}
    }{
      \typing{\ERecv{e_1}}{\TPair T S}
    }

    \inferrule{
      \typing{e}{\TSelect{l_i : S_i}} \\
      j \in I
    }{
      \typing{\ESelect{l_j}{e}}{S_j}
    }

    \inferrule{
      \typing[\Ctxt_1]{e}{\TCase{l_i : S_i}} \\
      \typing[\Ctxt_2]{e_i}{\TFun{S_i}{T}}
    }{
      \typing[\CSplit]{
        \ECase e { l_i \rightarrow e_i }
      }{T}
    }

    \inferrule{
      \typing{e}{\TFun {\TDual S} \TUnit}
    }{
      \typing{\EFork e}{S}
    }

    \inferrule{
      \typing{e}{S[\TRec[X] S / X]}
    }{
      \typing{\ERoll e}{\TRec S}
    }
  
    \inferrule{
      \typing{e}{\TRec S}
    }{
      \typing{\EUnroll e}{S[\TRec[X] S / X]}
    }
%
%   \inferrule{
%     \typing e T \\ f : T \eqt U : f'
%   }{
%     \typing e U
%   }
  \end{mathpar}
%   \label{fig:iso-typing-rules}
%   \caption{Typing rules for expressions in the iso-recursive system}
% \end{figure}

%%% Local Variables:
%%% mode: latex
%%% TeX-master: "main"
%%% End:


% Typing of expressions in the iso-recursive system is largely the same as for
% expressions in the equi-recursive system. As the only change, two additional
% typing rules are introduced concerned with $\EkwRoll$ and $\EkwUnroll$.
% \begin{mathpar}
%   \inferrule{
%     \typing{e}{S[\TRec[X] S / X]}
%   }{
%     \typing{\ERoll e}{\TRec S}
%   }
% 
%   \inferrule{
%     \typing{e}{\TRec S}
%   }{
%     \typing{\EUnroll e}{S[\TRec[X] S / X]}
%   }
% \end{mathpar}

% Evaluation contexts for the iso-recursive system have the same shape as the
% contexts in the equi-recursive system. The reduction relation is extended by
% two rules concerned with $\EkwRoll$ and $\EkwUnroll$.
% \begin{mathpar}
%   \reducerule{
%     \PScope (\PPar {E[\ERoll a]} {p})
%   }{
%     \PScope (\PPar {E[a]} {p})
%   }
% 
%   \reducerule{
%     \PScope (\PPar {E[\EUnroll a]} {p})
%   }{
%     \PScope (\PPar {E[a]} {p})
%   }
% \end{mathpar}
% Dual $\PScope[b,a]$ rules omitted.

%%% Local Variables:
%%% mode: latex
%%% TeX-master: "main"
%%% End:

\section{Translating equi-recursion to iso-recursion}

The set of syntactically valid expressions in the equi-recursive system is a
subset of the syntactically valid expressions in the iso-recursive system. A
simple one-to-one mapping, however, does not preserve well-typedness because
the target type system is more strict.

\declrel{Translation from equi-recursive to iso-recursive expressions}[$\dtrans{e}=e$]
\begin{mathpar}
  \dtranslhs{ \vdots }{ \typing \EUnit \TUnit } = \EUnit \and
  \dtranslhs{ \vdots }{ \typing x T } = x

  \dtranslhs{
    \dtbind[\Ctxt,\CBind x T] e U
  }{
    \typing {\ELam x e} {\TFun T U}
  } = \ELam x \dtvar e

  \dtranslhs{
    \dtbind[\Ctxt,\CBind* x {\TFun T U}] v {\TFun T U}
  }{
    \typing {\ERec x v} {\TFun T U}
  } = \ERec x \dtrans v

  \dtranslhs{
    \dtbind[\Ctxt_1] {e_1} {\TFun T U} \\
    \dtbind[\Ctxt_2] {e_2} T
  }{
    \typing[\CSplit] {e_1 e_2} U
  } = \dtvar{e_1} \dtvar{e_2}

  \dtranslhs{
    \dtbind[\Ctxt_1] {e_1} \TUnit \\
    \dtbind[\Ctxt_2] {e_2} T
  }{
    \typing[\CSplit] {\ELetU {e_1} {e_2}} T
  } = \ELetU {\dtvar{e_1}} {\dtvar{e_2}}

  \dtranslhs{
    \dtbind[\Ctxt_1] {e_1} T \\
    \dtbind[\Ctxt_2] {e_2} U
  }{
    \typing[\CSplit] {(e_1,e_2)} {\TPair T U}
  } = \left( \dtvar{e_1},\dtvar{e_2} \right)

  \dtranslhs{
    \dtbind[\Ctxt_1] {e_1} {\TPair T_1 T_2} \\
    \dtbind[\Ctxt_2, \CBind x T, \CBind y U] {e_2} U
  }{
    \typing[\CSplit]{\ELetP xy {e_1} {e_2}} U
  } = \ELetP xy {\dtvar{e_1}} {\dtvar{e_2}}

  \dtranslhs{
    \dtbind e {\TEnd ?}
  }{
    \typing {\EWait e} \TUnit
  } = \EWait {\dtvar e}

  \dtranslhs{
    \dtbind e {\TEnd !}
  }{
    \typing {\ETerm e} \TUnit
  } = \ETerm {\dtvar e}

  \dtranslhs{
    \dtbind {e_1} T \\
    \dtbind {e_2} {\TOut T S}
  }{
    \typing[\CSplit] {\ESend {e_1} {e_2}} S
  } = \ESend {\dtvar{e_1}} {\dtvar{e_2}}

  \dtranslhs{
    \dtbind e {\TIn T S}
  }{
    \typing {\ERecv e} {(T,S)}
  } = \ERecv {\dtvar e}

  \dtranslhs{
    \dtbind e {\TSelect{l_i : S_i}} \\
    j \in I
  }{
    \typing{\ESelect{l_j}{e}}{S_j}
  } = \ESelect {l_j} {\dtvar e}

  \dtranslhs{
    \dtbind[\Ctxt_1]{e}{\TCase{l_i : S_i}} \\
    \dtbind[\Ctxt_2]{e_i}{\TFun{S_i}{T}}
  }{
    \typing[\CSplit]{
      \ECase e { l_i \rightarrow e_i }
    }{T}
  } = \ECase {\dtvar e} { l_i \rightarrow \dtvar{e_i} }

  \dtranslhs{
    \dtbind e {\TFun {\TDual S} \TUnit}
  }{
    \typing{\EFork e}{S}
  } = \EFork {\dtvar e}

  \dtranslhs{
    \dtbind e T \\
    f : T \eqt U : f'
  }{
    \typing e U
  } = f \; \dtvar e
\end{mathpar}

%%% Local Variables:
%%% mode: latex
%%% TeX-master: "main"
%%% End:


Instead, the translation judgment $\translation e {e'} T$ in
Figure~\ref{fig:translation} describes the translation of equi-recursively
typed expression $e$ to an equivalent expression $e'$ in the iso-recursive
system. The interesting part of the translation process is where the
equi-recursive system makes use of type conversions. Here, an appriopriate
conversion function is inserted on the iso-recursive side.

The translation preserves well-typedness and the original and the
resulting process are \highlight{``wire-compatible''}. \js{what is the correct
word?} \pt{I've used that word, but the actual technical content is more tricky
than I imagined.}

% \begin{align*}
%   \dtransDef[\TUnit] \EUnit &= \EUnit &
%   \dtransDef x &= x \\
%   \dtransDef {\ELam x e} &= \ELam x \dtrans e &
%   \dtransDef {\ERec x e} &= \ERec x \dtrans e \\
%   \dtransDef {e_1 \; e_2} &= \dtrans{e_1} \; \dtrans{e_2} &
%   \dtransDef {\ELetU {e_1} {e_2}} &= \ELetU {\dtrans{e_1}} {\dtrans{e_2}} \\
%   \dtransDef {(e_1,e_2)} &= (\dtrans{e_1},\dtrans{e_2}) &
%   \dtransDef {\ELetP xy {e_1} {e_2}} &= \ELetP xy {\dtrans{e_1}} \dtrans{e_2} \\
% \end{align*}
% 
% \declrel{Typing derivation translation}
% \begin{mathpar}
%   \derivTransRule{ }{\typing \EUnit \TUnit}{\EUnit} \and
%   \derivTransRule{ }{\typing x T}{x} \and
%   
%   \derivTransRule{\DT e}{
%     \typing{\ELam x e}{U}
%   }{
%     \ELam x \derivTrans e
%   }
% 
%   \derivTransRule{\DT e}{
%     \typing{\ERec x e}{U}
%   }{
%     \ERec x \derivTrans e
%   }
% 
%   \derivTransRule{\DT {e_1} \\ \DT {e_2}}{
%     \typing[\CSplit]{e_1 e_2}{T_2}
%   }{
%     \derivTrans{e_1} \derivTrans{e_2}
%   }
% 
%   \derivTransRule{\DT {e_1} \\ \DT {e_2}}{
%     \typing{ \ELetU{e_1}{e_2} }{ T_2 }
%   }{
%     \ELetU {\derivTrans {e_1}} {\derivTrans {e_2}}
%   }
% 
%   \derivTransRule{\DT e \\ f : T \eqt U : f'}{
%     \typing e U
%   }{
%     f \; \derivTrans e
%   }
% \end{mathpar}

%%% Local Variables:
%%% mode: latex
%%% TeX-master: "main"
%%% End:


\section{Conclusion}
\label{sec:conclusion}

This work started from the question how to embed regular equirecursive
session types into algebraic session types
\cite{DBLP:journals/corr/abs-2304-03764}. As a first step we
considered the translation of equirecursive sessions into isorecursive
ones and noticed that they were neglected by the literature (talking
bananas being an exception \cite{DBLP:conf/icfp/LindleyM16}). One
reason for the neglegience is the folklore statement ``The
equirecursive interpretation of a session type guarantees that no 
message is required for the unfolding of a recursive type.''
\cite{DBLP:journals/pacmpl/BalzerP17}. We believe we dispelled this
myth by showing that isorecursive session types do not need messages
for unfolding as long as they are contractive.

In future work we will consider composing the results of this work
with a translation from isorecursive session types to algebraic
session types. There may also be scope to resolve subtyping by generating a conversion term
(different to Horne and Padovani's investigation
\cite{DBLP:journals/corr/abs-2304-06398}, which is based on fixed point logic) to isorecursion. 

% \begin{itemize}
% \item define mapping isorecursive $\leftrightarrow$ algebraic session types
% \item silent transitions for algebraic session types
% \item composition yields wire compatibility between equirecursive + subtyping and algebraic session types
% \end{itemize}

%%% Local Variables:
%%% mode: latex
%%% TeX-master: "main"
%%% End:


\bibliographystyle{eptcs}
\bibliography{biblio}

\clearpage
\appendix
\section{Expression typing}

\declrel{Typing contexts}
\begin{align*}
  \Ctxt \grmdef&
    \CNil \grmor \Ctxt, \CBind x T \grmor \Ctxt, \CBind* x T
\end{align*}

\declrel{Context exhaustion}[$\CExhausted$]
\begin{mathpar}
  \inferrule{ }{\CExhausted[\CNil]} \and
  \inferrule{\CExhausted}{\CExhausted[\Ctxt, \CBind* x T]}
\end{mathpar}

\declrel{Context splitting}[$\Ctxt = \CSplit[\Ctxt][\Ctxt]$]
\begin{mathpar}
  \inferrule{}{\cdot = \CSplit[\cdot][\cdot]} \and
  \inferrule{\Ctxt = \CSplit}{\Ctxt, \CBind  x T = \CSplit[(\Ctxt_1 , \CBind  x T)][ \Ctxt_2               ]} \and
  \inferrule{\Ctxt = \CSplit}{\Ctxt, \CBind  x T = \CSplit[ \Ctxt_ 1              ][(\Ctxt_2 , \CBind  x T)]} \and
  \inferrule{\Ctxt = \CSplit}{\Ctxt, \CBind* x T = \CSplit[(\Ctxt_1 , \CBind* x T)][(\Ctxt_2 , \CBind* x T)]}
\end{mathpar}

% \begin{figure}
  \declrel{Typing rules for expressions in the iso-recursive system}[$\typing{e}{T}$]
  \begin{mathpar}
    \inferrule{}{
      \typing[\Ctxt^*]{\EUnit}{\TUnit}
    }

    \inferrule{}{
      \typing[\Ctxt^*, \CBind x T]{x}{T}
    }

    \inferrule{}{
      \typing[\Ctxt^*, \CBind* x T]{x}{T}
    }

    \inferrule{
      % \CExhausted \\
      \typing[\Ctxt^*, \CBind* x {\TFun T U}]{v}{\TFun T U}
    }{
      \typing[\Ctxt^*]{\ERec x v}{\TFun{T}{U}}
    }

    \inferrule{
      \typing[\Ctxt, \CBind x T]{e}{U}
    }{
      \typing{\ELam x e}{\TFun T U}
    }

    \inferrule{
      \typing[\Ctxt_1]{e_1}{\TFun T U}
      \\
      \typing[\Ctxt_2]{e_2}{T}
    }{
      \typing[\CSplit]{e_1 e_2}{U}
    }

    \inferrule{
      \typing[\Ctxt_1]{e_1}{\TUnit} \\
      \typing[\Ctxt_2]{e_2}{T}
    }{
      \typing[\CSplit]{
        \ELetU{e_1}{e_2}
      }{ T }
    }

    \inferrule{
      \typing[\Ctxt_1]{e_1}{T} \\
      \typing[\Ctxt_2]{e_2}{U}
    }{
      \typing[\CSplit]{(e_1,e_2)}{\TPair T U}
    }

    \inferrule{
      \typing[\Ctxt_1]{e_1}{\TPair{T_1}{T_2}} \\
      \typing[\Ctxt_2, \CBind x {T_1}, \CBind y {T_2}]{e_2}{U}
    }{
      \typing[\CSplit]{
        \ELetP xy {e_1} {e_2}
      }{ U }
    }

    \inferrule{
      \typing{e}{\TEnd ?}
    }{
      \typing{\EWait e}{\TUnit}
    }

    \inferrule{
      \typing{e}{\TEnd !}
    }{
      \typing{\ETerm e}{\TUnit}
    }

    \inferrule{
      \typing[\Ctxt_1]{e_1} T  \\
      \typing[\Ctxt_2]{e_2} {\TOut T S}
    }{
      \typing[\CSplit]{\ESend{e_1}{e_2}} S
    }

    \inferrule{
      \typing{e_1} {\TIn T S}
    }{
      \typing{\ERecv{e_1}}{\TPair T S}
    }

    \inferrule{
      \typing{e}{\TSelect{l_i : S_i}} \\
      j \in I
    }{
      \typing{\ESelect{l_j}{e}}{S_j}
    }

    \inferrule{
      \typing[\Ctxt_1]{e}{\TCase{l_i : S_i}} \\
      \typing[\Ctxt_2]{e_i}{\TFun{S_i}{T}}
    }{
      \typing[\CSplit]{
        \ECase e { l_i \rightarrow e_i }
      }{T}
    }

    \inferrule{
      \typing{e}{\TFun {\TDual S} \TUnit}
    }{
      \typing{\EFork e}{S}
    }

    \inferrule{
      \typing{e}{S[\TRec[X] S / X]}
    }{
      \typing{\ERoll e}{\TRec S}
    }
  
    \inferrule{
      \typing{e}{\TRec S}
    }{
      \typing{\EUnroll e}{S[\TRec[X] S / X]}
    }
%
%   \inferrule{
%     \typing e T \\ f : T \eqt U : f'
%   }{
%     \typing e U
%   }
  \end{mathpar}
%   \label{fig:iso-typing-rules}
%   \caption{Typing rules for expressions in the iso-recursive system}
% \end{figure}

%%% Local Variables:
%%% mode: latex
%%% TeX-master: "main"
%%% End:


% Typing of expressions in the iso-recursive system is largely the same as for
% expressions in the equi-recursive system. As the only change, two additional
% typing rules are introduced concerned with $\EkwRoll$ and $\EkwUnroll$.
% \begin{mathpar}
%   \inferrule{
%     \typing{e}{S[\TRec[X] S / X]}
%   }{
%     \typing{\ERoll e}{\TRec S}
%   }
% 
%   \inferrule{
%     \typing{e}{\TRec S}
%   }{
%     \typing{\EUnroll e}{S[\TRec[X] S / X]}
%   }
% \end{mathpar}

% Evaluation contexts for the iso-recursive system have the same shape as the
% contexts in the equi-recursive system. The reduction relation is extended by
% two rules concerned with $\EkwRoll$ and $\EkwUnroll$.
% \begin{mathpar}
%   \reducerule{
%     \PScope (\PPar {E[\ERoll a]} {p})
%   }{
%     \PScope (\PPar {E[a]} {p})
%   }
% 
%   \reducerule{
%     \PScope (\PPar {E[\EUnroll a]} {p})
%   }{
%     \PScope (\PPar {E[a]} {p})
%   }
% \end{mathpar}
% Dual $\PScope[b,a]$ rules omitted.

%%% Local Variables:
%%% mode: latex
%%% TeX-master: "main"
%%% End:

\section{Process typing}
\label{sec:process-typing}


\declrel{Typing rules for processes}[$\Ptyping p$]
\begin{mathpar}
  \inferrule{\typing e \TUnit}{\Ptyping{\PExp e}}

  \inferrule{
    \Ptyping[\Ctxt_1]{p_1} \\
    \Ptyping[\Ctxt_2]{p_2}
  }{\Ptyping[\CSplit]{\PPar{p_1}{p_2}}}

  \inferrule{\Ptyping[\Ctxt, \CBind a S, \CBind b {\TDual S}] p
  }{\Ptyping{\PScope p}}
\end{mathpar}


%%% Local Variables:
%%% mode: latex
%%% TeX-master: "main"
%%% End:


\section{Properties}
\label{sec:properties}

% The type syntax for the iso-recursive system is the same as in the
% equi-recursive system, including the notion of session type dualization.
% 
% Type equivalence for the iso-recursive system is largely the same as in the
% equi-recursive system except for the rules \textsc{\EqUnrollL} and
% \textsc{\EqUnrollR}, which do not exist.

\vv{
  \begin{itemize}
  \item The dual function for recursive types is a bit more complicated; pls have
    a look at ``the final cut''.
    \pt{restricted payload type to closed types}
  \item Iso-recursive expressions $e$ must appear before type equivalence.
    \js{fixed this by moving sections around}
  \item IMO, $T,U$ reads better than $T_1, T_2$.
    \pt{sorry, messed this up!}
    \js{changed it}
  \item The expression for session types isomorphism may be curried. Then we
    seek these results, right? (do they hold?)

    \pt{Yes, see below in Sec~\ref{sec:properties}}

    Lemma. If $e : T \eqt U$, then $\typing{e}{\TFun{T}{U}}$.

    Lemma. If $e : R \eqs S$, then $\typing{e}{\TFun{R}{\TFun{\TDual S}{\TUnit}}}$.
  \item $f^{-1}$ is confusing; why not a simple $g$?
    \pt{Fixed, differently}
\item Added an easier to read, IMO, rule for \EqFun{} and (curried)
  for \EqOut
  \pt{Messed this up. Sorry again}
\item \EqPair{} uses an abbreviation, right? $\lambda p. \text{let}(x,y) = p
  \text{ in } \dots$
  \js{added note about this abbreviation next to grammar}
\item Added suggestion for roll and unroll; these constants may have function
  types. (types don't work :(
  \end{itemize}
}



TODO (PJT): what do we mean with $\cong$?
\begin{itemize}
\item for ``normal'' types, it should be contextual equivalence(?)
\item for channels, it might be bisimilarity(?)
\item might be simpler with an LTS semantics, see proposed definition:
\end{itemize}

\begin{definition}
  For $\typing[\CNil]{e_1, e_2} T$ define contextual equivalence $e_1 \cong e_2$
  iff, for all contexts $C : T \to \TUnit$, $C[e_1] \Downarrow$ iff $C[e_2] \Downarrow$. 
\end{definition}

\begin{lemma}[Conversions]~\\[-\baselineskip]
  \begin{enumerate}
    \item If $f : T \eqt T'$, then $f' : T' \eqt T$
      such that $\typing[\CNil]{f}{\TFun T{T'}}$
      and  $\typing[\CNil]{f'}{\TFun {T'}T}$
      and  $f \circ f' \cong \ELam x x$
      and  $f' \circ f \cong \ELam x x$.
    \item If $g : S \eqs S'$, then $g' : S' \eqs S$
      such that $\typing[\CNil]{g}{\TFun {\TPair {S} {\TDual{S'}}} \TUnit}$
      and $\typing[\CNil]{g'}{\TFun {\TPair {S'} {\TDual{S}}} \TUnit}$
      and,\\
      for all channels $c : S$,  $c \cong \ELet{c'}{\EFork{( \ELam {\bar c'} g \; (c, \bar c')) }}{\EFork{( \ELam {\bar c} g' \; (c', \bar c)) }} $.
\end{enumerate}
\end{lemma}

Proving contextual equivalence is daunting. Maybe set up a logical relation? But how to do the session part?
(I don't think it works like this!)

\begin{definition}
  Define $\reduceto e{e'}$ by $\reducemany e{e'}$ and $e'$
  irreducible.
\end{definition}
\begin{align*}
  V (\TUnit) & = \{ (\EUnit, \EUnit) \}
  \\
  V (\TPair TU) & = \{ (v_1, v_2), (w_1, w_2) \mid (v_1, w_1) \in V (T), (v_2, w_2) \in V (U) \}
  \\
  V (\TFun TU) &= \{ (\ELam xe_1, \ELam xe_2) \mid \forall v, w. (v, w) \in V (T) \Rightarrow (e_1[v/x], e_2[w/x]) \in E (U)  \}
  \\
  V (\TEnd!) &= \{ (c_1, c_2) \mid
                       \reduce{\ETerm c_1}{\EUnit}, \reduce{\ETerm
                       c_2}{\EUnit} \}
  \\
  V (\TEnd?) &= \{ (c_1, c_2) \mid
               \reduce{\EWait c_1}{\EUnit}, \reduce{\EWait
                       c_2}{\EUnit} \}
  \\
  V (\TIn TS) &= \{ (c_1, c_2) \mid
                \reduce{\ERecv c_1}{(v, c_1')}, 
                \reduce{\ERecv c_2}{(w, c_2')},
                (v,w) \in V(T),
                (c_1', c_2') \in V (S) \}
  % \\
  %               &\stackrel{?}{=}
  %                 \{ (c_1, c_2) \mid 
  %                 (\ERecv c_1,  \ERecv c_2) \in E (\TPair TS) \}
  \\
  V (\TOut TS) &= \{ (c_1, c_2) \mid \forall v, w. (v,w) \in V(T)
                \Rightarrow
                \reduce{\ESend v c_1}{c_1'}, 
                 \reduce{\ESend w c_2}{c_2'},
                 (c_1', c_2') \in V(S) \}
  % \\
  %               &\stackrel{?}{=}
  %                 \{ (c_1, c_2) \mid \mid \forall v, w. (v,w) \in V(T)
  %               \Rightarrow
  %                 (\ESend v c_1,  \ESend w c_2) \in E (S) \}
  \\
  V (\TSelect{l_i: S_i}) &=  \{ (c_1, c_2) \mid \forall j. j \in I
                           \Rightarrow
                           \reduce{\ESelect{l_j} c_1 }{c_1'}, 
                           \reduce{\ESelect{l_j} c_2 }{c_2'},
                           (c_1', c_2') \in V(S_j)
                           \}
  % \\&\stackrel{?}{=}
  % \{ (c_1, c_2) \mid \forall j. j \in I
  % \Rightarrow
  % (\ESelect{l_j}{c_1}, \ESelect{l_j}{c_1}) \in E (S_j)
  % \}
  \\
  V (\TCase{l_i: S_i})
             &= \{ (c_1, c_2) \mid \forall j. j \in I \Rightarrow
                         c_1 \stackrel{?l_j}{\to} e_1[c_1],
                c_2 \stackrel{?l_j}{\to} e_2[c_2],
                         (e_1, e_2) \in E(S_j) \}
  \\&\stackrel{?}{=}
  \{ (c_1, c_2) \mid
  \reduce{\ECase{c_1}{l_i \to e_{1i}}}{e_{1j}\; c_1'}, 
  \reduce{\ECase{c_2}{l_i \to e_{2i}}}{e_{2j}\; c_2'},
  (c_1', c_2') \in V (S_j)
  \}
  % \\&\stackrel{??}{=}
  % \{ (c_1, c_2) \mid \exists j\in I, (\ECase{c_1}{\dots},
  % \ECase{c_2}{\dots}) \in E (S_j) \}
  \\
  E (T) &= \{ (e_1, e_2) \mid \forall v_1, v_2, E_1, E_2. \reduceto{e_1}{E_1[v_1]}, \reduceto{e_2}{ E_2[v_2]} \Rightarrow (v_1, v_2)\in V (T) \}
\end{align*}

The session part seems wrong. I think we have to talk about processes stuck on a communication.
The following is inspired by a paper by Perez, Toninho, Pfenning ($+$ will need step indexing to deal with recursion): 

\begin{align*}
  L (c : \TEnd!) &= \{ (E[ \ETerm c ], E'[ \ETerm c ]) \mid (E[\EUnit], E'[\EUnit]) \in E (\TUnit)
                   \}
  \\
  L (c : \TOut TS) &= \{ (E[ \ESend v c ], E'[ \ESend w c ]) \mid \forall v, w. (v, w) \in V (T) \Rightarrow (E[c], E'[c]) \in L^* (c : S) \}
  \\
  L (c : \TIn TS) &= \{ (E[ \ERecv c ], E'[ \ERecv c ]) \mid \forall v, w. (v, w) \in V (T) \Rightarrow (E[(v,c)], E'[(w,c)]) \in L^* (c : S) \}
  \\
  L^* (c : S) &= \{ (e_1, e_2) \mid \forall e_1', e_2'. e_1 \to^* e_1', e_2 \to^* e_2' \Rightarrow (e_1', e_2') \in L (c : S) \}
\end{align*}

\subsection{Attempt at Vasco's suggestion}
\label{sec:attempt-at-vascos}

\begin{lemma}[Equational correspondence for expressions]
  Suppose $\Ctxt \vdash e_1 \leadsto e'_1 : T$ where $\Ctxt$ contains only
  channel bindings.
  \begin{enumerate}
  \item
    If $\reduce[\alpha]{e_1}{e_2}$
    and $\Ctxt \vdash e_2 \leadsto e'_2 : T$, then
    $\reducemany[\alpha]{e'_1}{e'_3}$ and $e_3' = e_2'$.
  % \item If $\reduce[\alpha]{e'_1}{e'_3}$, then $\reduce[\alpha]{e_1}{e_2}$ and
  %   $\reducemany[\alpha]{e'_3}{e'_2}$ and $\Ctxt \vdash e_2 \leadsto e'_2 : T$.
  \end{enumerate}
\end{lemma}

Define $e_1 = e_2$: reflexive, transitive, congruence, symmetric
closure of all expression reductions, and fork. As well as process
reduction on restricted channels. 
\begin{mathpar}
  \reduce[a!v]{\ESend v a}{a}

  \dots

  \inferrule{
    \reduce[\alpha]{e}{e'}
  }{
    \reduce[\alpha]{E[e]}{E[e']}
  }
  \\
  \text{process reductions}
  \\
  \inferrule{
    \reduce[a!v]{p_1}{p_1'} \\
    \reduce[b?v]{p_2}{p_2'}
  }{
    \reduce[a!v, b?v]{\PPar{p_1}{p_1'} }{\PPar{p_2}{p_2'} }
  }

  \inferrule{
    \reduce[a!v, b?v]{p }{p' }
  }{
    \reduce[\tau]{\PScope p}{\PScope {p'}}
  }
\end{mathpar}

\newpage
\begin{lemma}[Operational correspondence for expressions]
  Suppose $\Ctxt \vdash e_1 \leadsto e'_1 : T$ where $\Ctxt$ contains only
  channel bindings. \begin{enumerate}
  \item If $\reduce[\alpha]{e_1}{e_2}$, then
    $\reducemany[\alpha]{e'_1}{e'_2}$ and $\Ctxt \vdash e_2 \leadsto e'_2 : T$.
  \item If $\reduce[\alpha]{e'_1}{e'_3}$, then $\reduce[\alpha]{e_1}{e_2}$ and
    $\reducemany[\alpha]{e'_3}{e'_2}$ and $\Ctxt \vdash e_2 \leadsto e'_2 : T$.
  \end{enumerate}
\end{lemma}
\begin{proof} Item 1.
  The proof is by rule induction on $\Ctxt \vdash e_1 \leadsto e'_1 : T$.
  %
  Case the top-level translation rule is
  \begin{mathpar}
    \inferrule{
      \Ctxt \vdash e \leadsto e' : T \\
      f : T \eqt U
    }{
      \Ctxt \vdash e \leadsto f e' : U
    }
  \end{mathpar}

  Then, we know (premises) that $\Ctxt \vdash e_1 \leadsto e'_1 : T$ and $f : T
  \eqt U$.
  %
  We need to show that $\reducemany[\alpha]{f e_1'}{e'_2}$ and $\Ctxt \vdash e_2 \leadsto e'_2 : T$.
  %
  At this point I suggest proceeding by rule induction on
  $\reduce[\alpha]{e_1}{e_2}$. Details to be seen.

  \pt{Agreed, this part works ok. But what follows troubles me.}

  However, if $e_1$ is a value, then so is $e_1'$ (lemma!). 

  Proceed by induction on $f : T \eqt U$?

  Or: prove that if I can observe some $\alpha$ on $e_1$, then I can
  observe the same $\alpha$ on $f e_1'$.

  \begin{itemize}
  \item $\alpha = \ELetU\EHole{e''}$
  \item $\alpha = \ELetP xy\EHole{e''}$
  \item $\alpha = \EHole\; v$
  \item $\alpha = \EWait\EHole$
  \item $\alpha = \ETerm\EHole$
  \item $\alpha = \ESend{v}\EHole$
  \item $\alpha = \ERecv\EHole$
  \item $\alpha = \ESelect l \EHole$
  \item $\alpha = \ECase\EHole{l_i: e_i}$
  \end{itemize}
  
\end{proof}

%%% Local Variables:
%%% mode: latex
%%% TeX-master: "main"
%%% End:


\end{document}

%%% Local Variables:
%%% mode: latex
%%% TeX-master: t
%%% End:
