\section{Equi-recursion}
\label{sec:equi-recursion}
\begin{figure}[tp]
  \declrel{Type syntax}
\begin{align*}
  T,U \grmdef&
    S^\emptyset               \grmor
    \TUnit          \grmor
    \TPair TU      \grmor
    \TFun  T U      \\
  R^{\mathcal X}, S^{\mathcal X} \grmdef&
    \TEnd !         \grmor
    \TEnd ?         \grmor
    \TOut T S^{\mathcal X}       \grmor
    \TIn  T S^{\mathcal X}       \grmor
    \TSelect{ l_i: S^{\mathcal X}_i } \grmor
    \TCase{ l_i: S^{\mathcal X}_i }   \grmor
    \TRec[X] S^{\mathcal X \cup \{X\}}      \grmor
    X^{\in \mathcal X}
\end{align*}

\declrel{Dualization of session types}[$\TDual S = S$]
\begin{align*}
  \TDual X &= X                               &
  \TDual{\TEnd !} &= \TEnd ?                  &
  \TDual{\TOut T S} &= \TIn T \TDual{S}       &
  \TDual{\TSelect{ l_i: S_i }} &=
    \TCase{ l_i: \TDual{S_i} }                \\
  \TDual{\TRec S} &= \TRec{\TDual S}          &
  \TDual{\TEnd ?} &= \TEnd !                  &
  \TDual{\TIn T S} &= \TOut T \TDual{S}       &
  \TDual{\TCase{ l_i: S_i }} &=
    \TSelect{ l_i: \TDual{S_i} }
\end{align*}

%%% Local Variables:
%%% mode: latex
%%% TeX-master: "main"
%%% End:

  \declrel{Type equivalence}[$\convexpr{f} : T \eqt U$]
  \begin{mathpar}
    \inferrule[\EqUnit]{
    }{
      \convexpr{\ELam x x} :
      \TUnit \eqt \TUnit
      % : \ELam x x
    }

    \inferrule[\EqPair]{
      \convexpr{f_T} : T \eqt T' \\ %: f'_T \\
      \convexpr{f_U} : U \eqt U' %: f'_U
    }{
      \convexpr{\ELam {(x_1,x_2)} (f_T \; x_1, f_U \; x_2)} :
      \TPair T U \eqt \TPair {T'} {U'}
      % : \ELam {(x_1,x_2)} (f'_T \; x_1, f'_U \; x_2)
    }

    \inferrule[\EqFun]{
      \convexpr{f_{T'}} : T' \eqt T \\ %: f'_T \\
      \convexpr {f_U} : U \eqt U' %: f'_U
    }{
      \convexpr{\ELam x f_U \circ x \circ f_{T'}} :
      \TFun {T} {U} \eqt \TFun {T'} {U'}
      % : \ELam x f'_U \circ x \circ f_T
    }

    \inferrule[\EqS]{
      \convexpr{g} : R \eqs S
    }{
      \convexpr{\ELam {c_1} \EFork{ \ELam {c_2} g \; (c_1, c_2) }} :
      R \eqt S
      %: \ELam {c_1} \EFork{ \ELam {c_2} g' \; (c_1, c_2) }
    }
  \end{mathpar}
  \declrel{Session type equivalence}[$\eqsrel{g}{R}{S}$]
  \begin{mathpar}
    \inferrule[\EqAssump]{
      \convexpr{x} : \TRec R \eqs S \in \eqsctxt
    }{
      \eqsrel{x}{\TRec R}{S}
    }

    \inferrule[\EqUnrollL]{
      \eqsrel[\eqsctxt , \convexpr{x} : \TRec R \eqs S]
        {g}{R[\TRec R / X]}{S} \\
      \convexpr{x'} : \TRec R \eqs S \notin \eqsctxt
    }{
      \eqsrel{
        \ERec x
        \ELam{(c_1, c_2)}
        g \; (\EUnroll c_1, c_2)
      }{\TRec R}{S}
    }

    \inferrule[\EqUnrollR]{
      \eqsrel
      {g}{R}{S[\TRec S / X]}
      \\
      R \ne \TRec R'
    }{
      \eqsrel{
        \ELam{(c_1, c_2)}
        g \; (c_1, \EUnroll c_2)
      }{R}{\TRec S}
    }

    % \inferrule[\EqUnrollL]{
    %   \eqsrel[\eqsctxt , \convexpr{x} : \TRec S \eqs S']
    %     {g}{S[\TRec S / X]}{S'}
    % }{
    %   \eqsrel{
    %     \ERec x
    %     \ELam{(c_1, c_2)}
    %     g \; (\EUnroll c_1, c_2)
    %   }{\TRec S}{S'}
    % }
    %   
    % \inferrule[\EqUnrollR]{
    %   \eqsrel[\eqsctxt , \convexpr{x} : S \eqs \TRec S']
    %     {g}{S}{S'[\TRec S' / X]}
    % }{
    %   \eqsrel{
    %     \ERec x
    %     \ELam{(c_1, c_2)}
    %     g \; (c_1, \EUnroll c_2)
    %   }{S}{\TRec S'}
    % }
    %   
    \inferrule[\EqEnd !]{
    }{
      \eqsrel{
        \ELam {(c_1, c_2)} \ELetU {\EWait {c_2}} {\ETerm {c_1}}
      }{\TEnd !}{\TEnd !}
    }

    \inferrule[\EqEnd ?]{
    }{
      \eqsrel{
        \ELam {(c_1, c_2)} \ELetU {\EWait {c_1}} {\ETerm {c_2}}
      }{\TEnd ?}{\TEnd ?}
    }

    \inferrule[\EqOut]{
      \convexpr{f} : U \eqt T \\
      \eqsrel{g}{R}{S}
    }{
      \colorlet{outer}{.}
      {\begin{array}{r@{}l}
        \eqsctxt \vdash
            \convcolor \ELam {(c_1, c_2)}
          & \convcolor \ELetP {u} {c_2} {\ERecv{c_2}} \\
          & \convcolor \hspace{-2em} \ELet {c_1} {\ESend {f \; u} {c_1}} \\
          & \convcolor \hspace{-2em} g \; (c_1, c_2)
            \color{outer} \hspace{4em}
              : \TOut {T} {R} \eqs \TOut {U} {S}
      \end{array}}
    }

    \inferrule[\EqSelect]{
      \eqsrel{g_i}{R_i}{S_i}
    }{
      \eqsctxt \vdash \convexpr{
        \ELam {(c_1, c_2)}
        \ECase {c_2} { l_i \rightarrow \ELam {c_2} g_i \; (\ESelect {l_i} c_1, c_2) } \\ {}
      } \\\\
      : \TSelect { l_i : R_i } \eqs \TSelect { l_i : S_i }
    }


    \inferrule[\EqIn]{
      \convexpr{f} : T \eqt U \\
      \eqsrel{g}{R}{S}
    }{
      \colorlet{outer}{.}
      {\begin{array}{r@{}l}
        \eqsctxt \vdash
            \convcolor \ELam {(c_1, c_2)}
          & \convcolor \ELetP {t} {c_1} {\ERecv{c_1}} \\
          & \convcolor \hspace{-2em} \ELet {c_2} {\ESend {f \; t} {c_2}} \\
          & \convcolor \hspace{-2em} g \; (c_1, c_2)
            \color{outer} \hspace{4em}
              : \TIn {T} {R} \eqs \TIn {U} {S}
      \end{array}}
    }

    \inferrule[\EqCase]{
      \eqsrel{g_i}{R_i}{S_i}
    }{
      \eqsctxt \vdash \convexpr{
        \ELam {(c_1, c_2)}
        \ECase {c_1} { l_i \rightarrow \ELam {c_1} g_i \; (c_1, \ESelect {l_i} c_2) }
      } \\\\
      : \TCase { l_i : R_i } \eqs \TCase { l_i : S_i }
    }
  \end{mathpar}
  \caption{Equi-recursive system: types, duality, inductive type and session type equivalence}
  \label{fig:equi-equivalence}
\end{figure}

%%% Local Variables:
%%% mode: latex
%%% TeX-master: "main"
%%% End:


In the following we define a standard language for recursive
functional session types inspired by GV \cite{GV}. To avoid problems
with duality, the interplay of
recursion and session types is such that the payload type of a
communication is always closed \cite{final-cut}.

Figure~\ref{fig:equi-equivalence} contains the definition of the type
language along with equivalence for types $T \eqt T$  and session
types $S \eqs S$. The latter definition is coinductive and equates
session types up to unfolding of recursive types. The figure also
specifies evidence terms for equivalence (in light blue), which we ignore for now.

\declrel{Syntax for constants, values, expressions and processes}
\begin{align*}
  v \grmdef&
    c                        \grmor
    (v, v)                   \grmor
    \ELam x e                \grmor
    \ERec x v                \grmor
  \\
  e \grmdef&
    v                       \grmor
    x                       \grmor
    e \; e                  \grmor
    \ELetU e e              \grmor
    (e, e)                  \grmor
    \ELetP x y e e          \grmor
  %\\ &
    \ESelect l e            \grmor
    \ECase e { l_i \rightarrow e } \grmor
    \EFork e
  \\
  p \grmdef&
    e                       \grmor
    \PPar p p               \grmor
    \PScope p
\end{align*}

The expression $\ELam {(x,y)} e$ abbreviates $\ELam z \ELetP xy z e$
for some variable $z$ not free in $e$.


%%% Local Variables:
%%% mode: latex
%%% TeX-master: "main"
%%% End:


\subsection{Operational semantics}

\declrel{Structural congruence of processes}[$p \equiv p$]\medskip\\
The structural congruence relation on processes is defined as the smallest
congruence relation that includes the commutative monoidal rules with the
binary operator being parallel process composition $\PPar \_ \_$ and
value~$\EUnit$ as the neutral element, scope extrusion, and swapping
of binders:
\begin{align*}
  \PPar{\PScope p}{q} &\equiv \PScope (\PPar p q)
  \quad\text{if $a,b$ not free in $q$}
  \\
  \PScope p &\equiv \PScope[b,a] p
\end{align*}

\declrel{Evaluation contexts}
\begin{align*}
  E \grmdef&
    \EHole \grmor
    E \; e \grmor
    v \; E \grmor
    \ELetU E e \grmor
    (E,e) \grmor
    (v,E) \grmor
    \ELetP xy E e \grmor
  %\\ &
    \ESelect l E \grmor 
    \ECase E { l_i \rightarrow e_i }
\end{align*}

\declrel{Transition labels}
\begin{align*}
  \sigma \grmdef&
    !v \grmor ?v \grmor !l \grmor ?l \grmor ! \grmor ?
  && \text{label suffixes for session operations} \\
  \lambda \grmdef&
    a\sigma \grmor \beta \grmor  \LLet \grmor \PScope \EFork v
  &&\text{labels for expression reduction} \\
  \pi \grmdef&
               a\sigma
               \grmor \tau
               % \grmor \PScope c!a
               \grmor \RPar \pi \pi
  &&\text{labels for process reduction}
\end{align*}

\declrel{Duality for label suffixes}[$\TDual\sigma$]
\begin{align*}
  \TDual{!v} &= ?v
  &\TDual{ ?v} &= !v
  &\TDual{ !l} &= ?l
  &\TDual{ ?l} &= !l
  & \TDual ! &= ?
  &\TDual ? &= !
\end{align*}

\declrel{Free variables of process labels}[$\FV{p}$]
\begin{mathpar}
  \FV{a!v} = \FV{a?v} = \{a\} \cup \FV{v} \and
  \FV{a!l} = \FV{a?l} = \FV{a!} = \FV{a?} = \{a\} \and
  \FV{\tau} = \emptyset \and
  \FV{\PScope[c,d] a!x} = \{a,x\} \setminus \{c,d\} \and
  \FV{\RPar {\pi_1} {\pi_2}} = \FV{\pi_1} \cup \FV{\pi_2}
\end{mathpar}

\declrel{Labeled transition system for expressions (where $n$ ranges over
  non-value expressions)}[$\reduce[\lambda] e e$]
\begin{mathpar}
  \ltsrule \ActAppLetL {\EApp n e} \LLet {\ELet x n \EApp x e}

  \ltsrule \ActAppLetR {\EApp v n} \LLet {\ELet x n \EApp v x}

  \ltsrule \ActPairLetL {\EPair n e} \LLet {\ELet x n \EPair x e}

  \ltsrule \ActPairLetR {\EPair v n} \LLet {\ELet x n \EPair v x}

  \ltsrule \ActLetLet {\ELet {t_2} {(\ELet {t_1}{e_1}{e_2} )}{e_3}}
  \LLet {\ELet{t_1}{e_1}{(\ELet{t_2}{e_2}{e_3})}}

  \ltsrule \ActLetEta {\ELet t e t} \LLet e

  \ltsrule \ActApp { (\ELam x e) v } \beta { e[v / x] }

  \ltsrule \ActRec { (\ERec x v) v' } \beta { v[\ERec x v / x] v' }

  \ltsrule \ActLetBeta { \ELet x {v} e } \beta { e[v/x] }

  \ltsrule \ActLetPair { \ELetP xy {(v_1,v_2)} e } \beta { e[v_1/x, v_2/y] }

  \ltsrule \ActUnit { \ELetU \EUnit e } \beta { e }

  \ltsrule \ActSend { \ESend v a } {a!v} { a }

  \ltsrule \ActRecv { \ERecv a } {a?v} { (v,a) }

  \ltsrule \ActSelect { \ESelect l a } {a!l} { a }

  \ltsrule \ActCase { \ECase a { l_i \rightarrow v_i } } {a?l_j} { v_{\!j} \, a }

  \ltsrule \ActWait { \EWait a } {a?} { \EUnit }

  \ltsrule \ActTerm { \ETerm a } {a!} { \EUnit }

  \ltsrule \ActFork { \EFork v } { \PScope \EFork v } { a }
\end{mathpar}
The standard structural call-by-value rules are omitted.

\declrel{Labeled transition system for processes}[$\reduce[\pi] p p$]
\begin{mathpar}
  \ltsrule \ActSession [
    \reduce[a\sigma]{ e }{ e' }
  ]{ \PExp e } {a\sigma} { \PExp {e'} }

  \ltsrule \ActBeta [
    \reduce[\beta]{ e }{ e' }
  ]{ \PExp e } \tau { \PExp {e'} }

  \ltsrule \ActLet [
    \reduce[\LLet]{ e }{ e' }
  ]{ \PExp e } \tau { \PExp {e'} }

  \ltsrule \ActForkP [
    \reduce[ \PScope \EFork v ]{ e }{ e' }
  ]{ \PExp e } \tau { \PScope (\PPar {\PExp{e'}} {\PExp{v\;b}}) }

  \ltsrule \ActJoin [
    \reduce[\pi_1]{p}{p'} \\
    \reduce[\pi_2]{q}{q'} \\
  ]{ \PPar p q } {\RPar{\pi_1\!}{\pi_2}} { \PPar {p'} {q'} }

  \ltsrule \ActSync [
    \reduce[\RPar{a\sigma}{b\TDual\sigma}]{p}{p'}
  ]{ \PScope p } \tau { \PScope p' }

  % \ltsrule \ActMsg [
  %   \reduce[\RPar{a!v}{b?v}]{p}{p'}
  % ]{ \PScope p } \tau { \PScope p' }
  % 
  % \ltsrule \ActBranch [
  %   \reduce[\RPar{a!l}{b?l}]{p}{p'}
  % ]{ \PScope p } \tau { \PScope p' }
  % 
  % \ltsrule \ActEnd [
  %   \reduce[\RPar{a!}{b?}]{p}{p'}
  % ]{ \PScope p } \tau { p' }
  % 
  \ltsrule \ActScope [
    \reduce[\pi]{p}{p'} \\
    a,b \not\in \FV{\pi}
  ]{ \PScope p } \pi { \PScope p' }

  \ltsrule \ActPar [
    \reduce[\pi]{p}{p'}
  ]{ \PPar p q } \pi { \PPar {p'} q }

  \ltsrule \ActCong [
    p \equiv q \\
    \reduce[\pi]{q}{q'}
  ]{ p } \pi { q' }
\end{mathpar}

We define multistep labeled reductions in the usual way by
concatenating the labels using $\_\cdot\_$ as the concatenation
operator. We treat the silent label $\tau$ as a 
neutral element in the concatenation, i.e., $\tau \cdot \overline\pi = \overline\pi
\cdot \tau = \overline\pi$, where $\overline\pi$ is a sequence of labels. 
\begin{mathpar}
  \ltsrulemany \ActRefl {p} \tau p

  \ltsrulemany \ActOne[
  \reduce[\pi] p {p'} \\
  \reducemany[\overline\pi] {p'} {p''}
  ] p {\pi \cdot \overline\pi } {p''}
\end{mathpar}
We write $\reducemany[\alpha] p {p'}$ to indicate that the
$\alpha$-labeled transition comes from the final $\textsc{\ActOne}$ step in the
reduction sequence and all preceding transitions are $\tau$ transitions.

%%% Local Variables:
%%% mode: latex
%%% TeX-master: "main"
%%% End:


The typing rules are standard (see \dots), but we choose to make
equirecursion explicit by adopting a conversion rule, which is not
syntax directed.
\begin{mathpar}
  \inferrule{
    \typing e T \\ T \eqt U
  }{
    \typing e U
  }
\end{mathpar}

We omit the standard metatheoretial results \cite{metatheory}.
%\subsection{Types}

% \declrel{Type syntax}
% \begin{align*}
%   T \grmdef&
%     S^\emptyset               \grmor
%     \TUnit          \grmor
%     \TPair T T      \grmor
%     \TFun  T T      \\
%   S^{\mathcal X} \grmdef&
%     \TEnd !         \grmor
%     \TEnd ?         \grmor
%     \TOut T S^{\mathcal X}       \grmor
%     \TIn  T S^{\mathcal X}       \grmor
%     \TSelect{ l_i: S^{\mathcal X}_i } \grmor
%     \TCase{ l_i: S^{\mathcal X}_i }   \grmor
%     \TRec[X] S^{\mathcal X \cup \{X\}}      \grmor
%     X^{\in \mathcal X}
% \end{align*}
% 
% \declrel{Dualization of session types}[$\TDual S = S$]
% \begin{align*}
%   \TDual X &= X                               &
%   \TDual{\TEnd !} &= \TEnd ?                  &
%   \TDual{\TOut T S} &= \TIn T \TDual{S}       &
%   \TDual{\TSelect{ l_i: S_i }} &=
%     \TCase{ l_i: \TDual{S_i} }                \\
%   \TDual{\TRec S} &= \TRec{\TDual S}          &
%   \TDual{\TEnd ?} &= \TEnd !                  &
%   \TDual{\TIn T S} &= \TOut T \TDual{S}       &
%   \TDual{\TCase{ l_i: S_i }} &=
%     \TSelect{ l_i: \TDual{S_i} }
% \end{align*}

%\begin{figure}[tp]
  \declrel{Type syntax}
\begin{align*}
  T,U \grmdef&
    S^\emptyset               \grmor
    \TUnit          \grmor
    \TPair TU      \grmor
    \TFun  T U      \\
  R^{\mathcal X}, S^{\mathcal X} \grmdef&
    \TEnd !         \grmor
    \TEnd ?         \grmor
    \TOut T S^{\mathcal X}       \grmor
    \TIn  T S^{\mathcal X}       \grmor
    \TSelect{ l_i: S^{\mathcal X}_i } \grmor
    \TCase{ l_i: S^{\mathcal X}_i }   \grmor
    \TRec[X] S^{\mathcal X \cup \{X\}}      \grmor
    X^{\in \mathcal X}
\end{align*}

\declrel{Dualization of session types}[$\TDual S = S$]
\begin{align*}
  \TDual X &= X                               &
  \TDual{\TEnd !} &= \TEnd ?                  &
  \TDual{\TOut T S} &= \TIn T \TDual{S}       &
  \TDual{\TSelect{ l_i: S_i }} &=
    \TCase{ l_i: \TDual{S_i} }                \\
  \TDual{\TRec S} &= \TRec{\TDual S}          &
  \TDual{\TEnd ?} &= \TEnd !                  &
  \TDual{\TIn T S} &= \TOut T \TDual{S}       &
  \TDual{\TCase{ l_i: S_i }} &=
    \TSelect{ l_i: \TDual{S_i} }
\end{align*}

%%% Local Variables:
%%% mode: latex
%%% TeX-master: "main"
%%% End:

  \declrel{Type equivalence}[$\convexpr{f} : T \eqt U$]
  \begin{mathpar}
    \inferrule[\EqUnit]{
    }{
      \convexpr{\ELam x x} :
      \TUnit \eqt \TUnit
      % : \ELam x x
    }

    \inferrule[\EqPair]{
      \convexpr{f_T} : T \eqt T' \\ %: f'_T \\
      \convexpr{f_U} : U \eqt U' %: f'_U
    }{
      \convexpr{\ELam {(x_1,x_2)} (f_T \; x_1, f_U \; x_2)} :
      \TPair T U \eqt \TPair {T'} {U'}
      % : \ELam {(x_1,x_2)} (f'_T \; x_1, f'_U \; x_2)
    }

    \inferrule[\EqFun]{
      \convexpr{f_{T'}} : T' \eqt T \\ %: f'_T \\
      \convexpr {f_U} : U \eqt U' %: f'_U
    }{
      \convexpr{\ELam x f_U \circ x \circ f_{T'}} :
      \TFun {T} {U} \eqt \TFun {T'} {U'}
      % : \ELam x f'_U \circ x \circ f_T
    }

    \inferrule[\EqS]{
      \convexpr{g} : R \eqs S
    }{
      \convexpr{\ELam {c_1} \EFork{ \ELam {c_2} g \; (c_1, c_2) }} :
      R \eqt S
      %: \ELam {c_1} \EFork{ \ELam {c_2} g' \; (c_1, c_2) }
    }
  \end{mathpar}
  \declrel{Session type equivalence}[$\eqsrel{g}{R}{S}$]
  \begin{mathpar}
    \inferrule[\EqAssump]{
      \convexpr{x} : \TRec R \eqs S \in \eqsctxt
    }{
      \eqsrel{x}{\TRec R}{S}
    }

    \inferrule[\EqUnrollL]{
      \eqsrel[\eqsctxt , \convexpr{x} : \TRec R \eqs S]
        {g}{R[\TRec R / X]}{S} \\
      \convexpr{x'} : \TRec R \eqs S \notin \eqsctxt
    }{
      \eqsrel{
        \ERec x
        \ELam{(c_1, c_2)}
        g \; (\EUnroll c_1, c_2)
      }{\TRec R}{S}
    }

    \inferrule[\EqUnrollR]{
      \eqsrel
      {g}{R}{S[\TRec S / X]}
      \\
      R \ne \TRec R'
    }{
      \eqsrel{
        \ELam{(c_1, c_2)}
        g \; (c_1, \EUnroll c_2)
      }{R}{\TRec S}
    }

    % \inferrule[\EqUnrollL]{
    %   \eqsrel[\eqsctxt , \convexpr{x} : \TRec S \eqs S']
    %     {g}{S[\TRec S / X]}{S'}
    % }{
    %   \eqsrel{
    %     \ERec x
    %     \ELam{(c_1, c_2)}
    %     g \; (\EUnroll c_1, c_2)
    %   }{\TRec S}{S'}
    % }
    %   
    % \inferrule[\EqUnrollR]{
    %   \eqsrel[\eqsctxt , \convexpr{x} : S \eqs \TRec S']
    %     {g}{S}{S'[\TRec S' / X]}
    % }{
    %   \eqsrel{
    %     \ERec x
    %     \ELam{(c_1, c_2)}
    %     g \; (c_1, \EUnroll c_2)
    %   }{S}{\TRec S'}
    % }
    %   
    \inferrule[\EqEnd !]{
    }{
      \eqsrel{
        \ELam {(c_1, c_2)} \ELetU {\EWait {c_2}} {\ETerm {c_1}}
      }{\TEnd !}{\TEnd !}
    }

    \inferrule[\EqEnd ?]{
    }{
      \eqsrel{
        \ELam {(c_1, c_2)} \ELetU {\EWait {c_1}} {\ETerm {c_2}}
      }{\TEnd ?}{\TEnd ?}
    }

    \inferrule[\EqOut]{
      \convexpr{f} : U \eqt T \\
      \eqsrel{g}{R}{S}
    }{
      \colorlet{outer}{.}
      {\begin{array}{r@{}l}
        \eqsctxt \vdash
            \convcolor \ELam {(c_1, c_2)}
          & \convcolor \ELetP {u} {c_2} {\ERecv{c_2}} \\
          & \convcolor \hspace{-2em} \ELet {c_1} {\ESend {f \; u} {c_1}} \\
          & \convcolor \hspace{-2em} g \; (c_1, c_2)
            \color{outer} \hspace{4em}
              : \TOut {T} {R} \eqs \TOut {U} {S}
      \end{array}}
    }

    \inferrule[\EqSelect]{
      \eqsrel{g_i}{R_i}{S_i}
    }{
      \eqsctxt \vdash \convexpr{
        \ELam {(c_1, c_2)}
        \ECase {c_2} { l_i \rightarrow \ELam {c_2} g_i \; (\ESelect {l_i} c_1, c_2) } \\ {}
      } \\\\
      : \TSelect { l_i : R_i } \eqs \TSelect { l_i : S_i }
    }


    \inferrule[\EqIn]{
      \convexpr{f} : T \eqt U \\
      \eqsrel{g}{R}{S}
    }{
      \colorlet{outer}{.}
      {\begin{array}{r@{}l}
        \eqsctxt \vdash
            \convcolor \ELam {(c_1, c_2)}
          & \convcolor \ELetP {t} {c_1} {\ERecv{c_1}} \\
          & \convcolor \hspace{-2em} \ELet {c_2} {\ESend {f \; t} {c_2}} \\
          & \convcolor \hspace{-2em} g \; (c_1, c_2)
            \color{outer} \hspace{4em}
              : \TIn {T} {R} \eqs \TIn {U} {S}
      \end{array}}
    }

    \inferrule[\EqCase]{
      \eqsrel{g_i}{R_i}{S_i}
    }{
      \eqsctxt \vdash \convexpr{
        \ELam {(c_1, c_2)}
        \ECase {c_1} { l_i \rightarrow \ELam {c_1} g_i \; (c_1, \ESelect {l_i} c_2) }
      } \\\\
      : \TCase { l_i : R_i } \eqs \TCase { l_i : S_i }
    }
  \end{mathpar}
  \caption{Equi-recursive system: types, duality, inductive type and session type equivalence}
  \label{fig:equi-equivalence}
\end{figure}

%%% Local Variables:
%%% mode: latex
%%% TeX-master: "main"
%%% End:


%\subsection{Expressions}

% \declrel{Syntax for values, expressions and processes}
% \begin{align*}
%   c \grmdef&
%     \EkwSend \grmor \EkwRecv \grmor
%     \EkwTerm \grmor \EkwWait \grmor
%     \EkwRoll \grmor \EkwUnroll
%   \\
%   v \grmdef&
%     c                        \grmor
%     \EUnit                   \grmor
%     (v, v)                   \grmor
%     \ELam x e                \grmor
%     \ERec x e                \grmor
%   \\
%   e \grmdef&
%     v                       \grmor
%     x                       \grmor
%     e \; e                  \grmor
%     \ELetU e e              \grmor
%     (e, e)                  \grmor
%     \ELetP x y e e          \grmor
%   %\\ &
%     \ESelect l e            \grmor
%     \ECase e { l_i \rightarrow e } \grmor
%     \EFork e
%   \\
%   p \grmdef&
%     e                       \grmor
%     \PPar p p               \grmor
%     \PScope p
% \end{align*}


%\subsubsection{Expression typing}

% \declrel{Typing contexts}
% \begin{align*}
%   \Ctxt \grmdef&
%     \CNil \grmor \Ctxt, \CBind x T \grmor \Ctxt, \CBind* x T
% \end{align*}
% 
% \declrel{Context exhaustion}[$\CExhausted$]
% \begin{mathpar}
%   \inferrule{ }{\CExhausted[\CNil]} \and
%   \inferrule{\CExhausted}{\CExhausted[\Ctxt, \CBind* x T]}
% \end{mathpar}
% 
% \declrel{Context splitting}[$\Ctxt = \CSplit[\Ctxt][\Ctxt]$]
% \begin{mathpar}
%   \inferrule{}{\cdot = \CSplit[\cdot][\cdot]} \and
%   \inferrule{\Ctxt = \CSplit}{\Ctxt, \CBind  x T = \CSplit[(\Ctxt_1 , \CBind  x T)][ \Ctxt_2               ]} \and
%   \inferrule{\Ctxt = \CSplit}{\Ctxt, \CBind  x T = \CSplit[ \Ctxt_ 1              ][(\Ctxt_2 , \CBind  x T)]} \and
%   \inferrule{\Ctxt = \CSplit}{\Ctxt, \CBind* x T = \CSplit[(\Ctxt_1 , \CBind* x T)][(\Ctxt_2 , \CBind* x T)]}
% \end{mathpar}

%% \begin{figure}
  \declrel{Typing rules for expressions in the equi-recursive system}[$\typingE{e}{T}$]
  \begin{mathpar}
    \inferrule{ }{\typingE{\EUnit}{\TUnit}}

    \inferrule{
      \CExhausted
    }{
      \typingE[\Ctxt, \CBind x T]{x}{T}
    }

    \inferrule{
      \CExhausted
    }{
      \typingE[\Ctxt, \CBind* x T]{x}{T}
    }

    \inferrule{
      \CExhausted \\
      \typingE[\Ctxt, \CBind* x {\TFun {T_1} {T_2}}]{e}{\TFun {T_1} {T_2}}
    }{
      \typingE{\ERec x e}{\TFun{T_1}{T_2}}
    }

    \inferrule{
      \typingE[\Ctxt, \CBind x {T_1}]{e}{T_2}
    }{
      \typingE{\ELam x e}{\TFun{T_1}{T_2}}
    }

    \inferrule{
      \typingE[\Ctxt_1]{e_1}{\TFun {T_1}{T_2}}
      \\
      \typingE[\Ctxt_2]{e_2}{T_1}
    }{
      \typingE[\CSplit]{e_1 e_2}{T_2}
    }

    \inferrule{
      \typingE[\Ctxt_1]{e_1}{T_1} \\
      \typingE[\Ctxt_2]{e_2}{T_2}
    }{
      \typingE[\CSplit]{
        \ELetU{e_1}{e_2}
      }{ T_2 }
    }

    \inferrule{
      \typingE[\Ctxt_1]{e_1}{T_1} \\
      \typingE[\Ctxt_2]{e_2}{T_2}
    }{
      \typingE[\CSplit]{(e_1,e_2)}{\TPair{T_1}{T_2}}
    }

    \inferrule{
      \typingE[\Ctxt_1]{e_1}{\TPair{T_1}{T_2}} \\
      \typingE[\Ctxt_2, \CBind x {T_1}, \CBind y {T_2}]{e_2}{T_3}
    }{
      \typingE[\CSplit]{
        \ELetP xy {e_1} {e_2}
      }{ T_3 }
    }

    \inferrule{
      \typingE{e}{\TEnd ?}
    }{
      \typingE{\EWait e}{\TUnit}
    }

    \inferrule{
      \typingE{e}{\TEnd !}
    }{
      \typingE{\ETerm e}{\TUnit}
    }

    \inferrule{
      \typingE[\Ctxt_1]{e_1} T  \\
      \typingE[\Ctxt_2]{e_2} {\TOut T S}
    }{
      \typingE[\CSplit]{\ESend{e_1}{e_2}} S
    }

    \inferrule{
      \typingE{e_1} {\TIn T S}
    }{
      \typingE{\ERecv{e_1}}{\TPair T S}
    }

    \inferrule{
      \typingE{e}{\TSelect{l_i : S_i}} \\
      j \in I
    }{
      \typingE{\ESelect{l_j}{e}}{S_j}
    }

    \inferrule{
      \typingE[\Ctxt_1]{e}{\TCase{l_i : S_i}} \\
      \typingE[\Ctxt_2]{e_i}{\TFun{S_i}{T}}
    }{
      \typingE[\CSplit]{
        \ECase e { l_i \rightarrow e_i }
      }{T}
    }

    \inferrule{
      \typingI{e}{\TFun {\TDual S} \TUnit}
    }{
      \typingI{\EFork e}{S}
    }
  \end{mathpar}
%   \label{fig:equi-typing-rules}
%   \caption{Typing rules for expressions in the equi-recursive system}
% \end{figure}

%%% Local Variables:
%%% mode: latex
%%% TeX-master: "main"
%%% End:



%\subsubsection{Operational semantics}

% \declrel{Structural congruence of processes}[$p \equiv p$]\medskip\\
% The structural congruence relation on processes is defined as the smallest
% congruence relation that includes the commutative monoidal rules with the
% binary operator being parallel process composition $\PPar \_ \_$ and
% value~$\EUnit$ as the neutral element, and scope extrusion:
% \begin{align*}
%   \PPar{\PScope p}{q} &\equiv \PScope (\PPar p q)
%   \quad\text{if $a,b$ not free in $q$}
%   \\
%   \PScope p &\equiv \PScope[b,a] p
% \end{align*}
% 
% \declrel{Evaluation contexts}
% \begin{align*}
%   E \grmdef&
%     \EHole \grmor
%     E \; e \grmor
%     v \; E \grmor
%     \ELetU E e \grmor
%     (E,e) \grmor
%     (v,E) \grmor
%     \ELetP xy E e \grmor
%   %\\ &
%     \ESelect l E \grmor 
%     \ECase E { l_i \rightarrow e_i }
% \end{align*}
% 
% %\begin{figure}
  \declrel{Reduction relation in the equi-recursive system}[$\reduceE{p}{p}$]
  \begin{mathpar}
    \reduceruleE {
      (\ELam x e) v
    }{
      e[ v / a ]
    }

    \reduceruleE {
      (\ERec x e) v
    }{
      e[ \ERec x e / x ] \; v
    }

    \reduceruleE {
      \ELetU \EUnit e
    }{
      e
    }

    \reduceruleE {
      \ELetP {x_1} {x_2} {(v_1,v_2)} e
    }{
      e[v_1 / x_1][v_2 / x_2]
    }

    \reduceruleE[\reduceE{e_1}{e_2}]{
      E[e_1]
    }{
      E[e_2]
    }

    \reduceruleE{
      \PScope (\PPar* {\ESend v a} {\ERecv b} )
    }{
      \PScope (\PPar* {a} {(v,b)})
    }

    \reduceruleE{
      \PScope (\PPar* {\ESelect {l_j} a} {\ECase b { l_i \rightarrow e_i }})
    }{
      \PScope (\PPar* {a} {e_j \; b})
    }

    \reduceruleE{
      \PScope (\PPar* {\ETerm a} {\EWait b})
    }{
      \PPar* \EUnit \EUnit
    }

    \reduceruleE{
      E[\EFork e]
    }{
      \PScope (\PPar {E[a]} {e \; b})
    }

    \reduceruleE[\reduceE{p}{p^\prime}] { \PPar p q } { \PPar {p^\prime} q } \and
    \reduceruleE[\reduceE{p}{p^\prime}] { \PScope p } { \PScope p^\prime }   \and
    \reduceruleE[p \equiv q \\ \reduceE{q}{q^\prime}] { p } { q^\prime }
  \end{mathpar}
  Dual $\PScope[b,a]$ rules for $\EkwSend$/$\EkwRecv$, $\EkwSelect$/$\EkwCase$,
  $\EkwTerm$/$\EkwWait$ are omitted.
%   \label{fig:equi-reduction}
%   \caption{Reduction relation for the equi-recursive system}
% \end{figure}


%%% Local Variables:
%%% mode: latex
%%% TeX-master: "main"
%%% End:
