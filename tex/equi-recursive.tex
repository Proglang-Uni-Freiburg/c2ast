\section{Equi-recursion}

The iso-recursive system can---with very few and localized changes---be turned
into a traditional equi-recursive system. The type syntax and dualization of
session types require no adjustments.

\begin{figure}[tp]
  \declrel{Type syntax}
\begin{align*}
  T,U \grmdef&
    S^\emptyset               \grmor
    \TUnit          \grmor
    \TPair TU      \grmor
    \TFun  T U      \\
  R^{\mathcal X}, S^{\mathcal X} \grmdef&
    \TEnd !         \grmor
    \TEnd ?         \grmor
    \TOut T S^{\mathcal X}       \grmor
    \TIn  T S^{\mathcal X}       \grmor
    \TSelect{ l_i: S^{\mathcal X}_i } \grmor
    \TCase{ l_i: S^{\mathcal X}_i }   \grmor
    \TRec[X] S^{\mathcal X \cup \{X\}}      \grmor
    X^{\in \mathcal X}
\end{align*}

\declrel{Dualization of session types}[$\TDual S = S$]
\begin{align*}
  \TDual X &= X                               &
  \TDual{\TEnd !} &= \TEnd ?                  &
  \TDual{\TOut T S} &= \TIn T \TDual{S}       &
  \TDual{\TSelect{ l_i: S_i }} &=
    \TCase{ l_i: \TDual{S_i} }                \\
  \TDual{\TRec S} &= \TRec{\TDual S}          &
  \TDual{\TEnd ?} &= \TEnd !                  &
  \TDual{\TIn T S} &= \TOut T \TDual{S}       &
  \TDual{\TCase{ l_i: S_i }} &=
    \TSelect{ l_i: \TDual{S_i} }
\end{align*}

%%% Local Variables:
%%% mode: latex
%%% TeX-master: "main"
%%% End:

  \declrel{Type equivalence}[$\convexpr{f} : T \eqt U$]
  \begin{mathpar}
    \inferrule[\EqUnit]{
    }{
      \convexpr{\ELam x x} :
      \TUnit \eqt \TUnit
      % : \ELam x x
    }

    \inferrule[\EqPair]{
      \convexpr{f_T} : T \eqt T' \\ %: f'_T \\
      \convexpr{f_U} : U \eqt U' %: f'_U
    }{
      \convexpr{\ELam {(x_1,x_2)} (f_T \; x_1, f_U \; x_2)} :
      \TPair T U \eqt \TPair {T'} {U'}
      % : \ELam {(x_1,x_2)} (f'_T \; x_1, f'_U \; x_2)
    }

    \inferrule[\EqFun]{
      \convexpr{f_{T'}} : T' \eqt T \\ %: f'_T \\
      \convexpr {f_U} : U \eqt U' %: f'_U
    }{
      \convexpr{\ELam x f_U \circ x \circ f_{T'}} :
      \TFun {T} {U} \eqt \TFun {T'} {U'}
      % : \ELam x f'_U \circ x \circ f_T
    }

    \inferrule[\EqS]{
      \convexpr{g} : R \eqs S
    }{
      \convexpr{\ELam {c_1} \EFork{ \ELam {c_2} g \; (c_1, c_2) }} :
      R \eqt S
      %: \ELam {c_1} \EFork{ \ELam {c_2} g' \; (c_1, c_2) }
    }
  \end{mathpar}
  \declrel{Session type equivalence}[$\eqsrel{g}{R}{S}$]
  \begin{mathpar}
    \inferrule[\EqAssump]{
      \convexpr{x} : \TRec R \eqs S \in \eqsctxt
    }{
      \eqsrel{x}{\TRec R}{S}
    }

    \inferrule[\EqUnrollL]{
      \eqsrel[\eqsctxt , \convexpr{x} : \TRec R \eqs S]
        {g}{R[\TRec R / X]}{S} \\
      \convexpr{x'} : \TRec R \eqs S \notin \eqsctxt
    }{
      \eqsrel{
        \ERec x
        \ELam{(c_1, c_2)}
        g \; (\EUnroll c_1, c_2)
      }{\TRec R}{S}
    }

    \inferrule[\EqUnrollR]{
      \eqsrel
      {g}{R}{S[\TRec S / X]}
      \\
      R \ne \TRec R'
    }{
      \eqsrel{
        \ELam{(c_1, c_2)}
        g \; (c_1, \EUnroll c_2)
      }{R}{\TRec S}
    }

    % \inferrule[\EqUnrollL]{
    %   \eqsrel[\eqsctxt , \convexpr{x} : \TRec S \eqs S']
    %     {g}{S[\TRec S / X]}{S'}
    % }{
    %   \eqsrel{
    %     \ERec x
    %     \ELam{(c_1, c_2)}
    %     g \; (\EUnroll c_1, c_2)
    %   }{\TRec S}{S'}
    % }
    %   
    % \inferrule[\EqUnrollR]{
    %   \eqsrel[\eqsctxt , \convexpr{x} : S \eqs \TRec S']
    %     {g}{S}{S'[\TRec S' / X]}
    % }{
    %   \eqsrel{
    %     \ERec x
    %     \ELam{(c_1, c_2)}
    %     g \; (c_1, \EUnroll c_2)
    %   }{S}{\TRec S'}
    % }
    %   
    \inferrule[\EqEnd !]{
    }{
      \eqsrel{
        \ELam {(c_1, c_2)} \ELetU {\EWait {c_2}} {\ETerm {c_1}}
      }{\TEnd !}{\TEnd !}
    }

    \inferrule[\EqEnd ?]{
    }{
      \eqsrel{
        \ELam {(c_1, c_2)} \ELetU {\EWait {c_1}} {\ETerm {c_2}}
      }{\TEnd ?}{\TEnd ?}
    }

    \inferrule[\EqOut]{
      \convexpr{f} : U \eqt T \\
      \eqsrel{g}{R}{S}
    }{
      \colorlet{outer}{.}
      {\begin{array}{r@{}l}
        \eqsctxt \vdash
            \convcolor \ELam {(c_1, c_2)}
          & \convcolor \ELetP {u} {c_2} {\ERecv{c_2}} \\
          & \convcolor \hspace{-2em} \ELet {c_1} {\ESend {f \; u} {c_1}} \\
          & \convcolor \hspace{-2em} g \; (c_1, c_2)
            \color{outer} \hspace{4em}
              : \TOut {T} {R} \eqs \TOut {U} {S}
      \end{array}}
    }

    \inferrule[\EqSelect]{
      \eqsrel{g_i}{R_i}{S_i}
    }{
      \eqsctxt \vdash \convexpr{
        \ELam {(c_1, c_2)}
        \ECase {c_2} { l_i \rightarrow \ELam {c_2} g_i \; (\ESelect {l_i} c_1, c_2) } \\ {}
      } \\\\
      : \TSelect { l_i : R_i } \eqs \TSelect { l_i : S_i }
    }


    \inferrule[\EqIn]{
      \convexpr{f} : T \eqt U \\
      \eqsrel{g}{R}{S}
    }{
      \colorlet{outer}{.}
      {\begin{array}{r@{}l}
        \eqsctxt \vdash
            \convcolor \ELam {(c_1, c_2)}
          & \convcolor \ELetP {t} {c_1} {\ERecv{c_1}} \\
          & \convcolor \hspace{-2em} \ELet {c_2} {\ESend {f \; t} {c_2}} \\
          & \convcolor \hspace{-2em} g \; (c_1, c_2)
            \color{outer} \hspace{4em}
              : \TIn {T} {R} \eqs \TIn {U} {S}
      \end{array}}
    }

    \inferrule[\EqCase]{
      \eqsrel{g_i}{R_i}{S_i}
    }{
      \eqsctxt \vdash \convexpr{
        \ELam {(c_1, c_2)}
        \ECase {c_1} { l_i \rightarrow \ELam {c_1} g_i \; (c_1, \ESelect {l_i} c_2) }
      } \\\\
      : \TCase { l_i : R_i } \eqs \TCase { l_i : S_i }
    }
  \end{mathpar}
  \caption{Equi-recursive system: types, duality, inductive type and session type equivalence}
  \label{fig:equi-equivalence}
\end{figure}

%%% Local Variables:
%%% mode: latex
%%% TeX-master: "main"
%%% End:


The equi-recursive nature of this adjusted system handles any necessary
(un)rolling of recursive session types. This behavior is captured in the type
equivalence relation defined in Figure~\ref{fig:equi-equivalence}. Equivalence
$f : T \eqt T' : f'$ gives rise to two functions $f : \TFun {T} {T'}$, $f' :
\TFun {T'} {T}$. Equivalence of session types $g : S \eqs S' : g'$ gives rise
to two functions $g : \TFun {\TPair {S_1} {\TDual{S_2}}} \TUnit$ and $g' :
\TFun {\TPair {S_2} {\TDual{S_1}}} \TUnit$. All of these functions are
expressions in the iso-recursive system.

At the same time, the equi-recursive nature obviates the need for $\EkwRoll$
and $\EkwUnroll$ operations. These two values, their two typing rules and
reductions are removed. Instead, a rule is added which uses the equi-recursive
type equivalence.
\begin{mathpar}
  \inferrule{
    \typingE e T \\ f : T \eqt U : f'
  }{
    \typingE e U
  }
\end{mathpar}

%\subsection{Types}

% \declrel{Type syntax}
% \begin{align*}
%   T \grmdef&
%     S^\emptyset               \grmor
%     \TUnit          \grmor
%     \TPair T T      \grmor
%     \TFun  T T      \\
%   S^{\mathcal X} \grmdef&
%     \TEnd !         \grmor
%     \TEnd ?         \grmor
%     \TOut T S^{\mathcal X}       \grmor
%     \TIn  T S^{\mathcal X}       \grmor
%     \TSelect{ l_i: S^{\mathcal X}_i } \grmor
%     \TCase{ l_i: S^{\mathcal X}_i }   \grmor
%     \TRec[X] S^{\mathcal X \cup \{X\}}      \grmor
%     X^{\in \mathcal X}
% \end{align*}
% 
% \declrel{Dualization of session types}[$\TDual S = S$]
% \begin{align*}
%   \TDual X &= X                               &
%   \TDual{\TEnd !} &= \TEnd ?                  &
%   \TDual{\TOut T S} &= \TIn T \TDual{S}       &
%   \TDual{\TSelect{ l_i: S_i }} &=
%     \TCase{ l_i: \TDual{S_i} }                \\
%   \TDual{\TRec S} &= \TRec{\TDual S}          &
%   \TDual{\TEnd ?} &= \TEnd !                  &
%   \TDual{\TIn T S} &= \TOut T \TDual{S}       &
%   \TDual{\TCase{ l_i: S_i }} &=
%     \TSelect{ l_i: \TDual{S_i} }
% \end{align*}

\vv{
  \begin{itemize}
  \item The dual function for recursive types is a bit more complicated; pls have
    a look at ``the final cut''.
    \pt{restricted payload type to closed types}
  \item Iso-recursive expressions $e$ must appear before type equivalence.
  \item IMO, $T,U$ reads better than $T_1, T_2$.
    \pt{sorry, messed this up!}
  \item The expression for session types isomorphism may be curried. Then we
    seek these results, right? (do they hold?)

    \pt{Yes, see below in Sec~\ref{sec:properties}}

    Lemma. If $e : T \eqt U$, then $\typingE{e}{\TFun{T}{U}}$.

    Lemma. If $e : R \eqs S$, then $\typingE{e}{\TFun{R}{\TFun{\TDual S}{\TUnit}}}$.
  \item $f^{-1}$ is confusing; why not a simple $g$?
    \pt{Fixed, differently}
\item Added an easier to read, IMO, rule for \EqFun{} and (curried)
  for \EqOut
  \pt{Messed this up. Sorry again}
\item \EqPair{} uses an abbreviation, right? $\lambda p. \text{let}(x,y) = p
  \text{ in } \dots$
\item Added suggestion for roll and unroll; these constants may have function
  types. (types don't work :(
  \end{itemize}
}

%\begin{figure}[tp]
  \declrel{Type syntax}
\begin{align*}
  T,U \grmdef&
    S^\emptyset               \grmor
    \TUnit          \grmor
    \TPair TU      \grmor
    \TFun  T U      \\
  R^{\mathcal X}, S^{\mathcal X} \grmdef&
    \TEnd !         \grmor
    \TEnd ?         \grmor
    \TOut T S^{\mathcal X}       \grmor
    \TIn  T S^{\mathcal X}       \grmor
    \TSelect{ l_i: S^{\mathcal X}_i } \grmor
    \TCase{ l_i: S^{\mathcal X}_i }   \grmor
    \TRec[X] S^{\mathcal X \cup \{X\}}      \grmor
    X^{\in \mathcal X}
\end{align*}

\declrel{Dualization of session types}[$\TDual S = S$]
\begin{align*}
  \TDual X &= X                               &
  \TDual{\TEnd !} &= \TEnd ?                  &
  \TDual{\TOut T S} &= \TIn T \TDual{S}       &
  \TDual{\TSelect{ l_i: S_i }} &=
    \TCase{ l_i: \TDual{S_i} }                \\
  \TDual{\TRec S} &= \TRec{\TDual S}          &
  \TDual{\TEnd ?} &= \TEnd !                  &
  \TDual{\TIn T S} &= \TOut T \TDual{S}       &
  \TDual{\TCase{ l_i: S_i }} &=
    \TSelect{ l_i: \TDual{S_i} }
\end{align*}

%%% Local Variables:
%%% mode: latex
%%% TeX-master: "main"
%%% End:

  \declrel{Type equivalence}[$\convexpr{f} : T \eqt U$]
  \begin{mathpar}
    \inferrule[\EqUnit]{
    }{
      \convexpr{\ELam x x} :
      \TUnit \eqt \TUnit
      % : \ELam x x
    }

    \inferrule[\EqPair]{
      \convexpr{f_T} : T \eqt T' \\ %: f'_T \\
      \convexpr{f_U} : U \eqt U' %: f'_U
    }{
      \convexpr{\ELam {(x_1,x_2)} (f_T \; x_1, f_U \; x_2)} :
      \TPair T U \eqt \TPair {T'} {U'}
      % : \ELam {(x_1,x_2)} (f'_T \; x_1, f'_U \; x_2)
    }

    \inferrule[\EqFun]{
      \convexpr{f_{T'}} : T' \eqt T \\ %: f'_T \\
      \convexpr {f_U} : U \eqt U' %: f'_U
    }{
      \convexpr{\ELam x f_U \circ x \circ f_{T'}} :
      \TFun {T} {U} \eqt \TFun {T'} {U'}
      % : \ELam x f'_U \circ x \circ f_T
    }

    \inferrule[\EqS]{
      \convexpr{g} : R \eqs S
    }{
      \convexpr{\ELam {c_1} \EFork{ \ELam {c_2} g \; (c_1, c_2) }} :
      R \eqt S
      %: \ELam {c_1} \EFork{ \ELam {c_2} g' \; (c_1, c_2) }
    }
  \end{mathpar}
  \declrel{Session type equivalence}[$\eqsrel{g}{R}{S}$]
  \begin{mathpar}
    \inferrule[\EqAssump]{
      \convexpr{x} : \TRec R \eqs S \in \eqsctxt
    }{
      \eqsrel{x}{\TRec R}{S}
    }

    \inferrule[\EqUnrollL]{
      \eqsrel[\eqsctxt , \convexpr{x} : \TRec R \eqs S]
        {g}{R[\TRec R / X]}{S} \\
      \convexpr{x'} : \TRec R \eqs S \notin \eqsctxt
    }{
      \eqsrel{
        \ERec x
        \ELam{(c_1, c_2)}
        g \; (\EUnroll c_1, c_2)
      }{\TRec R}{S}
    }

    \inferrule[\EqUnrollR]{
      \eqsrel
      {g}{R}{S[\TRec S / X]}
      \\
      R \ne \TRec R'
    }{
      \eqsrel{
        \ELam{(c_1, c_2)}
        g \; (c_1, \EUnroll c_2)
      }{R}{\TRec S}
    }

    % \inferrule[\EqUnrollL]{
    %   \eqsrel[\eqsctxt , \convexpr{x} : \TRec S \eqs S']
    %     {g}{S[\TRec S / X]}{S'}
    % }{
    %   \eqsrel{
    %     \ERec x
    %     \ELam{(c_1, c_2)}
    %     g \; (\EUnroll c_1, c_2)
    %   }{\TRec S}{S'}
    % }
    %   
    % \inferrule[\EqUnrollR]{
    %   \eqsrel[\eqsctxt , \convexpr{x} : S \eqs \TRec S']
    %     {g}{S}{S'[\TRec S' / X]}
    % }{
    %   \eqsrel{
    %     \ERec x
    %     \ELam{(c_1, c_2)}
    %     g \; (c_1, \EUnroll c_2)
    %   }{S}{\TRec S'}
    % }
    %   
    \inferrule[\EqEnd !]{
    }{
      \eqsrel{
        \ELam {(c_1, c_2)} \ELetU {\EWait {c_2}} {\ETerm {c_1}}
      }{\TEnd !}{\TEnd !}
    }

    \inferrule[\EqEnd ?]{
    }{
      \eqsrel{
        \ELam {(c_1, c_2)} \ELetU {\EWait {c_1}} {\ETerm {c_2}}
      }{\TEnd ?}{\TEnd ?}
    }

    \inferrule[\EqOut]{
      \convexpr{f} : U \eqt T \\
      \eqsrel{g}{R}{S}
    }{
      \colorlet{outer}{.}
      {\begin{array}{r@{}l}
        \eqsctxt \vdash
            \convcolor \ELam {(c_1, c_2)}
          & \convcolor \ELetP {u} {c_2} {\ERecv{c_2}} \\
          & \convcolor \hspace{-2em} \ELet {c_1} {\ESend {f \; u} {c_1}} \\
          & \convcolor \hspace{-2em} g \; (c_1, c_2)
            \color{outer} \hspace{4em}
              : \TOut {T} {R} \eqs \TOut {U} {S}
      \end{array}}
    }

    \inferrule[\EqSelect]{
      \eqsrel{g_i}{R_i}{S_i}
    }{
      \eqsctxt \vdash \convexpr{
        \ELam {(c_1, c_2)}
        \ECase {c_2} { l_i \rightarrow \ELam {c_2} g_i \; (\ESelect {l_i} c_1, c_2) } \\ {}
      } \\\\
      : \TSelect { l_i : R_i } \eqs \TSelect { l_i : S_i }
    }


    \inferrule[\EqIn]{
      \convexpr{f} : T \eqt U \\
      \eqsrel{g}{R}{S}
    }{
      \colorlet{outer}{.}
      {\begin{array}{r@{}l}
        \eqsctxt \vdash
            \convcolor \ELam {(c_1, c_2)}
          & \convcolor \ELetP {t} {c_1} {\ERecv{c_1}} \\
          & \convcolor \hspace{-2em} \ELet {c_2} {\ESend {f \; t} {c_2}} \\
          & \convcolor \hspace{-2em} g \; (c_1, c_2)
            \color{outer} \hspace{4em}
              : \TIn {T} {R} \eqs \TIn {U} {S}
      \end{array}}
    }

    \inferrule[\EqCase]{
      \eqsrel{g_i}{R_i}{S_i}
    }{
      \eqsctxt \vdash \convexpr{
        \ELam {(c_1, c_2)}
        \ECase {c_1} { l_i \rightarrow \ELam {c_1} g_i \; (c_1, \ESelect {l_i} c_2) }
      } \\\\
      : \TCase { l_i : R_i } \eqs \TCase { l_i : S_i }
    }
  \end{mathpar}
  \caption{Equi-recursive system: types, duality, inductive type and session type equivalence}
  \label{fig:equi-equivalence}
\end{figure}

%%% Local Variables:
%%% mode: latex
%%% TeX-master: "main"
%%% End:


%\subsection{Expressions}

% \declrel{Syntax for values, expressions and processes}
% \begin{align*}
%   c \grmdef&
%     \EkwSend \grmor \EkwRecv \grmor
%     \EkwTerm \grmor \EkwWait \grmor
%     \EkwRoll \grmor \EkwUnroll
%   \\
%   v \grmdef&
%     c                        \grmor
%     \EUnit                   \grmor
%     (v, v)                   \grmor
%     \ELam x e                \grmor
%     \ERec x e                \grmor
%   \\
%   e \grmdef&
%     v                       \grmor
%     x                       \grmor
%     e \; e                  \grmor
%     \ELetU e e              \grmor
%     (e, e)                  \grmor
%     \ELetP x y e e          \grmor
%   %\\ &
%     \ESelect l e            \grmor
%     \ECase e { l_i \rightarrow e } \grmor
%     \EFork e
%   \\
%   p \grmdef&
%     e                       \grmor
%     \PPar p p               \grmor
%     \PScope p
% \end{align*}


%\subsubsection{Expression typing}

% \declrel{Typing contexts}
% \begin{align*}
%   \Ctxt \grmdef&
%     \CNil \grmor \Ctxt, \CBind x T \grmor \Ctxt, \CBind* x T
% \end{align*}
% 
% \declrel{Context exhaustion}[$\CExhausted$]
% \begin{mathpar}
%   \inferrule{ }{\CExhausted[\CNil]} \and
%   \inferrule{\CExhausted}{\CExhausted[\Ctxt, \CBind* x T]}
% \end{mathpar}
% 
% \declrel{Context splitting}[$\Ctxt = \CSplit[\Ctxt][\Ctxt]$]
% \begin{mathpar}
%   \inferrule{}{\cdot = \CSplit[\cdot][\cdot]} \and
%   \inferrule{\Ctxt = \CSplit}{\Ctxt, \CBind  x T = \CSplit[(\Ctxt_1 , \CBind  x T)][ \Ctxt_2               ]} \and
%   \inferrule{\Ctxt = \CSplit}{\Ctxt, \CBind  x T = \CSplit[ \Ctxt_ 1              ][(\Ctxt_2 , \CBind  x T)]} \and
%   \inferrule{\Ctxt = \CSplit}{\Ctxt, \CBind* x T = \CSplit[(\Ctxt_1 , \CBind* x T)][(\Ctxt_2 , \CBind* x T)]}
% \end{mathpar}

%% \begin{figure}
  \declrel{Typing rules for expressions in the equi-recursive system}[$\typingE{e}{T}$]
  \begin{mathpar}
    \inferrule{ }{\typingE{\EUnit}{\TUnit}}

    \inferrule{
      \CExhausted
    }{
      \typingE[\Ctxt, \CBind x T]{x}{T}
    }

    \inferrule{
      \CExhausted
    }{
      \typingE[\Ctxt, \CBind* x T]{x}{T}
    }

    \inferrule{
      \CExhausted \\
      \typingE[\Ctxt, \CBind* x {\TFun {T_1} {T_2}}]{e}{\TFun {T_1} {T_2}}
    }{
      \typingE{\ERec x e}{\TFun{T_1}{T_2}}
    }

    \inferrule{
      \typingE[\Ctxt, \CBind x {T_1}]{e}{T_2}
    }{
      \typingE{\ELam x e}{\TFun{T_1}{T_2}}
    }

    \inferrule{
      \typingE[\Ctxt_1]{e_1}{\TFun {T_1}{T_2}}
      \\
      \typingE[\Ctxt_2]{e_2}{T_1}
    }{
      \typingE[\CSplit]{e_1 e_2}{T_2}
    }

    \inferrule{
      \typingE[\Ctxt_1]{e_1}{T_1} \\
      \typingE[\Ctxt_2]{e_2}{T_2}
    }{
      \typingE[\CSplit]{
        \ELetU{e_1}{e_2}
      }{ T_2 }
    }

    \inferrule{
      \typingE[\Ctxt_1]{e_1}{T_1} \\
      \typingE[\Ctxt_2]{e_2}{T_2}
    }{
      \typingE[\CSplit]{(e_1,e_2)}{\TPair{T_1}{T_2}}
    }

    \inferrule{
      \typingE[\Ctxt_1]{e_1}{\TPair{T_1}{T_2}} \\
      \typingE[\Ctxt_2, \CBind x {T_1}, \CBind y {T_2}]{e_2}{T_3}
    }{
      \typingE[\CSplit]{
        \ELetP xy {e_1} {e_2}
      }{ T_3 }
    }

    \inferrule{
      \typingE{e}{\TEnd ?}
    }{
      \typingE{\EWait e}{\TUnit}
    }

    \inferrule{
      \typingE{e}{\TEnd !}
    }{
      \typingE{\ETerm e}{\TUnit}
    }

    \inferrule{
      \typingE[\Ctxt_1]{e_1} T  \\
      \typingE[\Ctxt_2]{e_2} {\TOut T S}
    }{
      \typingE[\CSplit]{\ESend{e_1}{e_2}} S
    }

    \inferrule{
      \typingE{e_1} {\TIn T S}
    }{
      \typingE{\ERecv{e_1}}{\TPair T S}
    }

    \inferrule{
      \typingE{e}{\TSelect{l_i : S_i}} \\
      j \in I
    }{
      \typingE{\ESelect{l_j}{e}}{S_j}
    }

    \inferrule{
      \typingE[\Ctxt_1]{e}{\TCase{l_i : S_i}} \\
      \typingE[\Ctxt_2]{e_i}{\TFun{S_i}{T}}
    }{
      \typingE[\CSplit]{
        \ECase e { l_i \rightarrow e_i }
      }{T}
    }

    \inferrule{
      \typingI{e}{\TFun {\TDual S} \TUnit}
    }{
      \typingI{\EFork e}{S}
    }
  \end{mathpar}
%   \label{fig:equi-typing-rules}
%   \caption{Typing rules for expressions in the equi-recursive system}
% \end{figure}

%%% Local Variables:
%%% mode: latex
%%% TeX-master: "main"
%%% End:



%\subsubsection{Operational semantics}

% \declrel{Structural congruence of processes}[$p \equiv p$]\medskip\\
% The structural congruence relation on processes is defined as the smallest
% congruence relation that includes the commutative monoidal rules with the
% binary operator being parallel process composition $\PPar \_ \_$ and
% value~$\EUnit$ as the neutral element, and scope extrusion:
% \begin{align*}
%   \PPar{\PScope p}{q} &\equiv \PScope (\PPar p q)
%   \quad\text{if $a,b$ not free in $q$}
%   \\
%   \PScope p &\equiv \PScope[b,a] p
% \end{align*}
% 
% \declrel{Evaluation contexts}
% \begin{align*}
%   E \grmdef&
%     \EHole \grmor
%     E \; e \grmor
%     v \; E \grmor
%     \ELetU E e \grmor
%     (E,e) \grmor
%     (v,E) \grmor
%     \ELetP xy E e \grmor
%   %\\ &
%     \ESelect l E \grmor 
%     \ECase E { l_i \rightarrow e_i }
% \end{align*}
% 
% %\begin{figure}
  \declrel{Reduction relation in the equi-recursive system}[$\reduceE{p}{p}$]
  \begin{mathpar}
    \reduceruleE {
      (\ELam x e) v
    }{
      e[ v / a ]
    }

    \reduceruleE {
      (\ERec x e) v
    }{
      e[ \ERec x e / x ] \; v
    }

    \reduceruleE {
      \ELetU \EUnit e
    }{
      e
    }

    \reduceruleE {
      \ELetP {x_1} {x_2} {(v_1,v_2)} e
    }{
      e[v_1 / x_1][v_2 / x_2]
    }

    \reduceruleE[\reduceE{e_1}{e_2}]{
      E[e_1]
    }{
      E[e_2]
    }

    \reduceruleE{
      \PScope (\PPar* {\ESend v a} {\ERecv b} )
    }{
      \PScope (\PPar* {a} {(v,b)})
    }

    \reduceruleE{
      \PScope (\PPar* {\ESelect {l_j} a} {\ECase b { l_i \rightarrow e_i }})
    }{
      \PScope (\PPar* {a} {e_j \; b})
    }

    \reduceruleE{
      \PScope (\PPar* {\ETerm a} {\EWait b})
    }{
      \PPar* \EUnit \EUnit
    }

    \reduceruleE{
      E[\EFork e]
    }{
      \PScope (\PPar {E[a]} {e \; b})
    }

    \reduceruleE[\reduceE{p}{p^\prime}] { \PPar p q } { \PPar {p^\prime} q } \and
    \reduceruleE[\reduceE{p}{p^\prime}] { \PScope p } { \PScope p^\prime }   \and
    \reduceruleE[p \equiv q \\ \reduceE{q}{q^\prime}] { p } { q^\prime }
  \end{mathpar}
  Dual $\PScope[b,a]$ rules for $\EkwSend$/$\EkwRecv$, $\EkwSelect$/$\EkwCase$,
  $\EkwTerm$/$\EkwWait$ are omitted.
%   \label{fig:equi-reduction}
%   \caption{Reduction relation for the equi-recursive system}
% \end{figure}


%%% Local Variables:
%%% mode: latex
%%% TeX-master: "main"
%%% End:
