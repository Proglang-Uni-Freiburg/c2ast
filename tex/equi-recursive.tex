\section{Equi-recursion}
\label{sec:equi-recursion}
%\begin{figure}
  \declrel{Type equivalence}[$e : T \eqt T$]
  \begin{mathpar}
    \inferrule[\EqUnit]{
    }{
      \ELam x x :
      \TUnit \eqt \TUnit
    }

    \inferrule[\EqPair]{
      f_1 : T_1 \eqt T_1^\prime \\
      f_2 : T_2 \eqt T_2^\prime
    }{
      \ELam {(x_1,x_2)} (f_1 \; x_1, f_2 \; x_2) :
      \TPair {T_1} {T_2} \eqt \TPair {T_1^\prime} {T_2^\prime}
    }

    \inferrule[\EqFun]{
      f : U_1 \eqt T_1 \\
      g : T_2 \eqt U_2
    }{
      \ELam h {\ELam x {g(h(f x))}} :
      \TFun {T_1} {T_2} \eqt \TFun {U_1} {U_2}
    }

    \inferrule[\EqFun]{
      f_1 : T_1 \eqt T_1^\prime \\
      f_2 : T_2 \eqt T_2^\prime
    }{
      \ELam x f_2 \circ x \circ f_1^{-1} :
      \TFun {T_1} {T_2} \eqt \TFun {T_1^\prime} {T_2^\prime}
    }

    \inferrule[\EqS]{
      g : S_1 \eqs S_2
    }{
      \ELam {c_1} \EFork{ \ELam {c_2} g \; c_1 \; c_2 } :
      S_1 \eqt S_2
    }
  \end{mathpar}
  % 
  \declrel{Session type equivalence}[$e : S \eqs S$]\medskip\\
  $g : S_1 \eqs S_2$ gives rise to
  $g : \TFun {\TPair {S_1} {\TDual{S_2}}} \TUnit$
  and
  $g^{-1} : \TFun {\TPair {S_2} {\TDual{S_1}}} \TUnit$
  \begin{mathpar}
    \inferrule[\EqEnd !]{
    }{
      \ELam {(c_1, c_2)} \ELetU {\EWait {c_2}} {\ETerm {c_1}} :
      \TEnd ! \eqs \TEnd !
    }

    \inferrule[\EqEnd ?]{
    }{
      \ELam {(c_1, c_2)} \ELetU {\EWait {c_1}} {\ETerm {c_2}} :
      \TEnd ? \eqs \TEnd ?
    }

    \inferrule[\EqOut]{
      f : T \eqt U  \\
      g : R \eqs S
    }{
      \ELam {c_1}{
      \ELam {c_2}
        \ELetP {x} {c_2} {\ERecv{c_2}}
        \ELet {c_1} {\ESend {(f \; x)} {c_1}}
        g\; c_1\; c_2}
      : \TOut {T} {R} \eqs \TOut {U} {S}
    }

    \inferrule[\EqOut]{
      f : T_1 \eqt T_2  \\
      g : S_1 \eqs S_2
    }{
      \ELam {(c_1, c_2)}
        \ELetP {t_1} {c_2} {\ERecv{c_2}}
        \ELet {c_1} {\ESend {(f \; t_1)} {c_1}}
        g \; (c_1, c_2)
      : \TOut {T_1} {S_1} \eqs \TOut {T_2} {S_2}
    }

    \inferrule[\EqIn]{
      f : T_1 \eqt T_2  \\
      g : S_1 \eqs S_2
    }{
      \ELam {(c_1, c_2)}
        \ELetP {t_1} {c_1} {\ERecv{c_1}}
        \ELet {c_2} {\ESend {(f \; t_1)} {c_2}}
        g \; (c_1, c_2)
      : \TIn {T_1} {S_1} \eqs \TIn {T_2} {S_2}
    }

    \inferrule[\EqSelect]{
      g_i : S_i \eqs S_i^\prime
    }{
      \ELam {(c_1, c_2)}
        \ECase {c_2} { l_i \rightarrow \ELam {c_2} g_i \; (\ESelect {l_i} c_1, c_2) }
      : \TSelect { l_i : S_i } \eqs \TSelect { l_i : S_i^\prime }
    }

    \inferrule[\EqCase]{
      g_i : S_i \eqs S_i^\prime
    }{
      \ELam {(c_1, c_2)}
        \ECase {c_1} { l_i \rightarrow \ELam {c_1} g_i \; (c_1, \ESelect {l_i} c_2) }
      : \TCase { l_i : S_i } \eqs \TCase { l_i : S_i^\prime }
    }

    \inferrule[\EqUnroll]{
    }{
      \EkwUnroll : \TRec S \eqs S[\TRec S / X]
    }

    \inferrule[\EqRoll]{
    }{
      \EkwRoll : S[\TRec S / X] \eqs S
    }

    \inferrule[\EqUnrollL]{
      g : S_1[\TRec S_1 / X] \eqs S_2
    }{
      \TRec S_1 \eqs S_2
    }

    \inferrule[\EqUnrollR]{
      g : S_1 \eqs S_2[\TRec S_2 / X]
    }{
      S_1 \eqs \TRec S_2
    }
  \end{mathpar}
%  \label{fig:equi-equivalence}
%  \caption{Type equivalence in the equi-recursive system}
%\end{figure}

%%% Local Variables:
%%% mode: latex
%%% TeX-master: "main"
%%% End:


In the following we define a core language for equirecursive
functional session types inspired by GV
\cite{DBLP:journals/jfp/GayV10}, but much simplified. To ensure that
duality is well-defined,
recursion is restricted to session types and the payload type of a
communication is always closed \cite{DBLP:journals/corr/abs-2004-01322}.

Figure~\ref{fig:equi-equivalence} contains the definition of the type language
and duality along with equivalence for types and session types. The nonterminal
$S$ for session types is indexed with a set $\mathcal X$ of variables that may
occur free in $S$. The $\mu$-binder increases this set for its body. We assume
that recursive types are contractive, that is they must not contain a type of
the form $\TRec[X_1]\dots\TRec[X_n]X_1$.

The same figure also contains inductive definitions of type equivalence $T \eqt
T$ and session type equivalence $\eqsctxt \vdash S \eqs S$. Equality of session
types works similar to Gay and Hole's algorithmic subtyping rules for session
types~\cite{DBLP:journals/acta/GayH05}. $\eqsctxt$ contains a sequence of
assumed equalities, which is used by \EqAssump{} and extended by \EqUnrollL{}
and \EqUnrollR{}. The judgements are not fully syntax directed. In order to
guarantee termination \EqAssump{} must be used whenever applicable. To ensure
deterministic results we arbitrarily stipulate that \EqUnrollL{} is to be
preferred over \EqUnrollR{} should both be applicable. If
$\eqsrel[\emptyset]{g}{S}{S'}$ is derivable we write $\eqsrel[]{g}{S}{S'}$ or
just $\convexpr{g} : S \eqs S'$. The light blue parts specify equivalence
evidence terms, which we ignore for now.

\begin{lemma}\label{lemma:congruence}
 The relations $\eqt$ and $\eqs$ are congruence relations.
\end{lemma}

\declrel{Syntax for constants, values, expressions and processes}
\begin{align*}
  v \grmdef&
    c                        \grmor
    (v, v)                   \grmor
    \ELam x e                \grmor
    \ERec x v                \grmor
  \\
  e \grmdef&
    v                       \grmor
    x                       \grmor
    e \; e                  \grmor
    \ELetU e e              \grmor
    (e, e)                  \grmor
    \ELetP x y e e          \grmor
  %\\ &
    \ESelect l e            \grmor
    \ECase e { l_i \rightarrow e } \grmor
    \EFork e
  \\
  p \grmdef&
    e                       \grmor
    \PPar p p               \grmor
    \PScope p
\end{align*}

The expression $\ELam {(x,y)} e$ abbreviates $\ELam z \ELetP xy z e$
for some variable $z$ not free in $e$.


%%% Local Variables:
%%% mode: latex
%%% TeX-master: "main"
%%% End:


\subsection{Operational semantics}

\declrel{Structural congruence of processes}[$p \equiv p$]\medskip\\
The structural congruence relation on processes is defined as the smallest
congruence relation that includes the commutative monoidal rules with the
binary operator being parallel process composition $\PPar \_ \_$ and
value~$\EUnit$ as the neutral element, scope extrusion, and swapping
of binders:
\begin{align*}
  \PPar{\PScope p}{q} &\equiv \PScope (\PPar p q)
  \quad\text{if $a,b$ not free in $q$}
  \\
  \PScope p &\equiv \PScope[b,a] p
\end{align*}

\declrel{Evaluation contexts}
\begin{align*}
  E \grmdef&
    \EHole \grmor
    % E \; e \grmor
    % v \; E \grmor
    \ELet t E e \grmor
    % (E,e) \grmor
    % (v,E) \grmor
    % \ELetP xy E e \grmor
  %\\ &
    % \ESelect l E \grmor 
    \ECase E { l_i \rightarrow e_i }
\end{align*}

\declrel{Transition labels}
\begin{align*}
  \sigma \grmdef&
    !v \grmor ?v \grmor !l \grmor ?l \grmor ! \grmor ?
  && \text{label suffixes for session operations} \\
  \lambda \grmdef&
    a\sigma \grmor \beta \grmor \PScope[a,a] \EFork v
  &&\text{labels for expression reduction} \\
  \pi \grmdef&
               a\sigma \grmor \tau \grmor \PScope[a,a] a!a \RPar \pi
               \pi
  &&\text{labels for process reduction}
\end{align*}

\declrel{Free variables of process labels}[$\FV{p}$]
\begin{mathpar}
  \FV{a!v} = \FV{a?v} = \{a\} \cup \FV{v} \and
  \FV{a!l} = \FV{a?l} = \FV{a!} = \FV{a?} = \{a\} \and
  \FV{\tau} = \emptyset \and
  \FV{\PScope[c,d] a!x} = \{a,x\} \setminus \{c,d\} \and
  \FV{\RPar {\pi_1} {\pi_2}} = \FV{\pi_1} \cup \FV{\pi_2}
\end{mathpar}

\declrel{Labeled transition system for expressions}[$\reduce[\lambda] e e$]
\begin{mathpar}
  \ltsrule \ActApp { (\ELam x e) v } \beta { e[v / x] }

  \ltsrule \ActRec { (\ERec x v) v' } \beta { v[\ERec x v / x] v' }

  \ltsrule \ActLet { \ELetP xy {(v_1,v_2)} e } \beta { e[v_1/x, v_2/y] }

  \ltsrule \ActUnit { \ELetU \EUnit e } \beta { e }

  \ltsrule \ActSend { \ESend v a } {a!v} { a }

  \ltsrule \ActRecv { \ERecv a } {a?v} { (v,a) }

  \ltsrule \ActSelect { \ESelect l a } {a!l} { a }

  \ltsrule \ActCase { \ECase a { l_i \rightarrow v_i } } {a?l_j} { v_{\!j} \, a }

  \ltsrule \ActWait { \EWait a } {a?} { \EUnit }

  \ltsrule \ActTerm { \ETerm a } {a!} { \EUnit }

  \ltsrule \ActFork { \EFork v } { \PScope \EFork v } { a }
\end{mathpar}
The standard structural call-by-value rules are omitted.

\declrel{Labeled transition system for processes}[$\reduce[\pi] p p$]
\begin{mathpar}
  \ltsrule \ActSession [
    \reduce[\sigma]{ e }{ e' }
  ]{ \PExp e } \sigma { \PExp {e'} }

  \ltsrule \ActBeta [
    \reduce[\beta]{ e }{ e' }
  ]{ \PExp e } \tau { \PExp {e'} }

  \ltsrule \ActForkP [
    \reduce[ \PScope \EFork v ]{ e }{ e' }
  ]{ \PExp e } \tau { \PScope (\PPar {\PExp{e'}} {\PExp{v\;b}}) }

  \ltsrule \ActJoin [
    \reduce[\pi_1]{p}{p'} \\
    \reduce[\pi_2]{q}{q'} \\
  ]{ \PPar p q } {\RPar{\pi_1\!}{\pi_2}} { \PPar {p'} {q'} }

  \ltsrule \ActMsg [
    \reduce[\RPar{a!v}{b?v}]{p}{p'}
  ]{ \PScope p } \tau { \PScope p' }

  \ltsrule \ActBranch [
    \reduce[\RPar{a!l}{b?l}]{p}{p'}
  ]{ \PScope p } \tau { \PScope p' }

  \ltsrule \ActEnd [
    \reduce[\RPar{a!}{b?}]{p}{p'}
  ]{ \PScope p } \tau { p' }

  \ltsrule \ActScope [
    \reduce[\pi]{p}{p'} \\
    a,b \not\in \FV{\pi}
  ]{ \PScope p } \pi { \PScope p' }

  \ltsrule \ActPar [
    \reduce[\pi]{p}{p'}
  ]{ \PPar p q } \pi { \PPar {p'} q }

 \ltsrule \ActOpen [
   \reduce[a!c]{p}{p'} \\
   a \not\in \{c,d\}
 ]{ \PScope[c,d] p } { \PScope[c,d] a!c } { p' }

 \ltsrule \ActClose [
   \reduce[\RPar{\PScope[c,d] a!c}{b?c}]{p}{p'}
 ]{ \PScope[a,b] p } \tau { \PScope[a,b] \PScope[c,d] p' }

  \ltsrule \ActCong [
    p \equiv q \\
    \reduce[\pi]{q}{q'}
  ]{ p } \pi { q' }
\end{mathpar}
\js{I'm unclear about why/if \textsc{\ActOpen} and \textsc{\ActClose} are needed.}

%%% Local Variables:
%%% mode: latex
%%% TeX-master: "main"
%%% End:


The expression reductions implement a standard call-by-value lambda
calculus with extra reductions from Moggi's computational lambda
calculus \cite{DBLP:conf/lics/Moggi89}. Process reductions implement a synchronous semantics for
session types.

The typing rules are standard (see appendix~\ref{sec:expression-typing}), but we choose to make
equirecursion explicit by adopting a conversion rule, which is not
syntax directed.
\begin{mathpar}
  \inferrule{
    \typing {\tyEqui e} T \\ T \eqt U
  }{
    \typing {\tyEqui e} U
  }
\end{mathpar}

We omit the standard metatheoretical results.
%  \cite{metatheory} % needed?
%\subsection{Types}

% \declrel{Type syntax}
% \begin{align*}
%   T \grmdef&
%     S^\emptyset               \grmor
%     \TUnit          \grmor
%     \TPair T T      \grmor
%     \TFun  T T      \\
%   S^{\mathcal X} \grmdef&
%     \TEnd !         \grmor
%     \TEnd ?         \grmor
%     \TOut T S^{\mathcal X}       \grmor
%     \TIn  T S^{\mathcal X}       \grmor
%     \TSelect{ l_i: S^{\mathcal X}_i } \grmor
%     \TCase{ l_i: S^{\mathcal X}_i }   \grmor
%     \TRec[X] S^{\mathcal X \cup \{X\}}      \grmor
%     X^{\in \mathcal X}
% \end{align*}
% 
% \declrel{Dualization of session types}[$\TDual S = S$]
% \begin{align*}
%   \TDual X &= X                               &
%   \TDual{\TEnd !} &= \TEnd ?                  &
%   \TDual{\TOut T S} &= \TIn T \TDual{S}       &
%   \TDual{\TSelect{ l_i: S_i }} &=
%     \TCase{ l_i: \TDual{S_i} }                \\
%   \TDual{\TRec S} &= \TRec{\TDual S}          &
%   \TDual{\TEnd ?} &= \TEnd !                  &
%   \TDual{\TIn T S} &= \TOut T \TDual{S}       &
%   \TDual{\TCase{ l_i: S_i }} &=
%     \TSelect{ l_i: \TDual{S_i} }
% \end{align*}

%%\begin{figure}
  \declrel{Type equivalence}[$e : T \eqt T$]
  \begin{mathpar}
    \inferrule[\EqUnit]{
    }{
      \ELam x x :
      \TUnit \eqt \TUnit
    }

    \inferrule[\EqPair]{
      f_1 : T_1 \eqt T_1^\prime \\
      f_2 : T_2 \eqt T_2^\prime
    }{
      \ELam {(x_1,x_2)} (f_1 \; x_1, f_2 \; x_2) :
      \TPair {T_1} {T_2} \eqt \TPair {T_1^\prime} {T_2^\prime}
    }

    \inferrule[\EqFun]{
      f : U_1 \eqt T_1 \\
      g : T_2 \eqt U_2
    }{
      \ELam h {\ELam x {g(h(f x))}} :
      \TFun {T_1} {T_2} \eqt \TFun {U_1} {U_2}
    }

    \inferrule[\EqFun]{
      f_1 : T_1 \eqt T_1^\prime \\
      f_2 : T_2 \eqt T_2^\prime
    }{
      \ELam x f_2 \circ x \circ f_1^{-1} :
      \TFun {T_1} {T_2} \eqt \TFun {T_1^\prime} {T_2^\prime}
    }

    \inferrule[\EqS]{
      g : S_1 \eqs S_2
    }{
      \ELam {c_1} \EFork{ \ELam {c_2} g \; c_1 \; c_2 } :
      S_1 \eqt S_2
    }
  \end{mathpar}
  % 
  \declrel{Session type equivalence}[$e : S \eqs S$]\medskip\\
  $g : S_1 \eqs S_2$ gives rise to
  $g : \TFun {\TPair {S_1} {\TDual{S_2}}} \TUnit$
  and
  $g^{-1} : \TFun {\TPair {S_2} {\TDual{S_1}}} \TUnit$
  \begin{mathpar}
    \inferrule[\EqEnd !]{
    }{
      \ELam {(c_1, c_2)} \ELetU {\EWait {c_2}} {\ETerm {c_1}} :
      \TEnd ! \eqs \TEnd !
    }

    \inferrule[\EqEnd ?]{
    }{
      \ELam {(c_1, c_2)} \ELetU {\EWait {c_1}} {\ETerm {c_2}} :
      \TEnd ? \eqs \TEnd ?
    }

    \inferrule[\EqOut]{
      f : T \eqt U  \\
      g : R \eqs S
    }{
      \ELam {c_1}{
      \ELam {c_2}
        \ELetP {x} {c_2} {\ERecv{c_2}}
        \ELet {c_1} {\ESend {(f \; x)} {c_1}}
        g\; c_1\; c_2}
      : \TOut {T} {R} \eqs \TOut {U} {S}
    }

    \inferrule[\EqOut]{
      f : T_1 \eqt T_2  \\
      g : S_1 \eqs S_2
    }{
      \ELam {(c_1, c_2)}
        \ELetP {t_1} {c_2} {\ERecv{c_2}}
        \ELet {c_1} {\ESend {(f \; t_1)} {c_1}}
        g \; (c_1, c_2)
      : \TOut {T_1} {S_1} \eqs \TOut {T_2} {S_2}
    }

    \inferrule[\EqIn]{
      f : T_1 \eqt T_2  \\
      g : S_1 \eqs S_2
    }{
      \ELam {(c_1, c_2)}
        \ELetP {t_1} {c_1} {\ERecv{c_1}}
        \ELet {c_2} {\ESend {(f \; t_1)} {c_2}}
        g \; (c_1, c_2)
      : \TIn {T_1} {S_1} \eqs \TIn {T_2} {S_2}
    }

    \inferrule[\EqSelect]{
      g_i : S_i \eqs S_i^\prime
    }{
      \ELam {(c_1, c_2)}
        \ECase {c_2} { l_i \rightarrow \ELam {c_2} g_i \; (\ESelect {l_i} c_1, c_2) }
      : \TSelect { l_i : S_i } \eqs \TSelect { l_i : S_i^\prime }
    }

    \inferrule[\EqCase]{
      g_i : S_i \eqs S_i^\prime
    }{
      \ELam {(c_1, c_2)}
        \ECase {c_1} { l_i \rightarrow \ELam {c_1} g_i \; (c_1, \ESelect {l_i} c_2) }
      : \TCase { l_i : S_i } \eqs \TCase { l_i : S_i^\prime }
    }

    \inferrule[\EqUnroll]{
    }{
      \EkwUnroll : \TRec S \eqs S[\TRec S / X]
    }

    \inferrule[\EqRoll]{
    }{
      \EkwRoll : S[\TRec S / X] \eqs S
    }

    \inferrule[\EqUnrollL]{
      g : S_1[\TRec S_1 / X] \eqs S_2
    }{
      \TRec S_1 \eqs S_2
    }

    \inferrule[\EqUnrollR]{
      g : S_1 \eqs S_2[\TRec S_2 / X]
    }{
      S_1 \eqs \TRec S_2
    }
  \end{mathpar}
%  \label{fig:equi-equivalence}
%  \caption{Type equivalence in the equi-recursive system}
%\end{figure}

%%% Local Variables:
%%% mode: latex
%%% TeX-master: "main"
%%% End:


%\subsection{Expressions}

% \declrel{Syntax for values, expressions and processes}
% \begin{align*}
%   c \grmdef&
%     \EkwSend \grmor \EkwRecv \grmor
%     \EkwTerm \grmor \EkwWait \grmor
%     \EkwRoll \grmor \EkwUnroll
%   \\
%   v \grmdef&
%     c                        \grmor
%     \EUnit                   \grmor
%     (v, v)                   \grmor
%     \ELam x e                \grmor
%     \ERec x e                \grmor
%   \\
%   e \grmdef&
%     v                       \grmor
%     x                       \grmor
%     e \; e                  \grmor
%     \ELetU e e              \grmor
%     (e, e)                  \grmor
%     \ELetP x y e e          \grmor
%   %\\ &
%     \ESelect l e            \grmor
%     \ECase e { l_i \rightarrow e } \grmor
%     \EFork e
%   \\
%   p \grmdef&
%     e                       \grmor
%     \PPar p p               \grmor
%     \PScope p
% \end{align*}


%\subsubsection{Expression typing}

% \declrel{Typing contexts}
% \begin{align*}
%   \Ctxt \grmdef&
%     \CNil \grmor \Ctxt, \CBind x T \grmor \Ctxt, \CBind* x T
% \end{align*}
% 
% \declrel{Context exhaustion}[$\CExhausted$]
% \begin{mathpar}
%   \inferrule{ }{\CExhausted[\CNil]} \and
%   \inferrule{\CExhausted}{\CExhausted[\Ctxt, \CBind* x T]}
% \end{mathpar}
% 
% \declrel{Context splitting}[$\Ctxt = \CSplit[\Ctxt][\Ctxt]$]
% \begin{mathpar}
%   \inferrule{}{\cdot = \CSplit[\cdot][\cdot]} \and
%   \inferrule{\Ctxt = \CSplit}{\Ctxt, \CBind  x T = \CSplit[(\Ctxt_1 , \CBind  x T)][ \Ctxt_2               ]} \and
%   \inferrule{\Ctxt = \CSplit}{\Ctxt, \CBind  x T = \CSplit[ \Ctxt_ 1              ][(\Ctxt_2 , \CBind  x T)]} \and
%   \inferrule{\Ctxt = \CSplit}{\Ctxt, \CBind* x T = \CSplit[(\Ctxt_1 , \CBind* x T)][(\Ctxt_2 , \CBind* x T)]}
% \end{mathpar}

%% \begin{figure}
  \declrel{Typing rules for expressions in the equi-recursive system}[$\typingE{e}{T}$]
  \begin{mathpar}
    \inferrule{ }{\typingE{\EUnit}{\TUnit}}

    \inferrule{
      \CExhausted
    }{
      \typingE[\Ctxt, \CBind x T]{x}{T}
    }

    \inferrule{
      \CExhausted
    }{
      \typingE[\Ctxt, \CBind* x T]{x}{T}
    }

    \inferrule{
      \CExhausted \\
      \typingE[\Ctxt, \CBind* x {\TFun {T_1} {T_2}}]{e}{\TFun {T_1} {T_2}}
    }{
      \typingE{\ERec x e}{\TFun{T_1}{T_2}}
    }

    \inferrule{
      \typingE[\Ctxt, \CBind x {T_1}]{e}{T_2}
    }{
      \typingE{\ELam x e}{\TFun{T_1}{T_2}}
    }

    \inferrule{
      \typingE[\Ctxt_1]{e_1}{\TFun {T_1}{T_2}}
      \\
      \typingE[\Ctxt_2]{e_2}{T_1}
    }{
      \typingE[\CSplit]{e_1 e_2}{T_2}
    }

    \inferrule{
      \typingE[\Ctxt_1]{e_1}{T_1} \\
      \typingE[\Ctxt_2]{e_2}{T_2}
    }{
      \typingE[\CSplit]{
        \ELetU{e_1}{e_2}
      }{ T_2 }
    }

    \inferrule{
      \typingE[\Ctxt_1]{e_1}{T_1} \\
      \typingE[\Ctxt_2]{e_2}{T_2}
    }{
      \typingE[\CSplit]{(e_1,e_2)}{\TPair{T_1}{T_2}}
    }

    \inferrule{
      \typingE[\Ctxt_1]{e_1}{\TPair{T_1}{T_2}} \\
      \typingE[\Ctxt_2, \CBind x {T_1}, \CBind y {T_2}]{e_2}{T_3}
    }{
      \typingE[\CSplit]{
        \ELetP xy {e_1} {e_2}
      }{ T_3 }
    }

    \inferrule{
      \typingE{e}{\TEnd ?}
    }{
      \typingE{\EWait e}{\TUnit}
    }

    \inferrule{
      \typingE{e}{\TEnd !}
    }{
      \typingE{\ETerm e}{\TUnit}
    }

    \inferrule{
      \typingE[\Ctxt_1]{e_1} T  \\
      \typingE[\Ctxt_2]{e_2} {\TOut T S}
    }{
      \typingE[\CSplit]{\ESend{e_1}{e_2}} S
    }

    \inferrule{
      \typingE{e_1} {\TIn T S}
    }{
      \typingE{\ERecv{e_1}}{\TPair T S}
    }

    \inferrule{
      \typingE{e}{\TSelect{l_i : S_i}} \\
      j \in I
    }{
      \typingE{\ESelect{l_j}{e}}{S_j}
    }

    \inferrule{
      \typingE[\Ctxt_1]{e}{\TCase{l_i : S_i}} \\
      \typingE[\Ctxt_2]{e_i}{\TFun{S_i}{T}}
    }{
      \typingE[\CSplit]{
        \ECase e { l_i \rightarrow e_i }
      }{T}
    }

    \inferrule{
      \typingE{e}{\TFun {\TDual S} \TUnit}
    }{
      \typingE{\EFork e}{S}
    }
  \end{mathpar}
%   \label{fig:equi-typing-rules}
%   \caption{Typing rules for expressions in the equi-recursive system}
% \end{figure}

%%% Local Variables:
%%% mode: latex
%%% TeX-master: "main"
%%% End:



%\subsubsection{Operational semantics}

% \declrel{Structural congruence of processes}[$p \equiv p$]\medskip\\
% The structural congruence relation on processes is defined as the smallest
% congruence relation that includes the commutative monoidal rules with the
% binary operator being parallel process composition $\PPar \_ \_$ and
% value~$\EUnit$ as the neutral element, and scope extrusion:
% \begin{align*}
%   \PPar{\PScope p}{q} &\equiv \PScope (\PPar p q)
%   \quad\text{if $a,b$ not free in $q$}
%   \\
%   \PScope p &\equiv \PScope[b,a] p
% \end{align*}
% 
% \declrel{Evaluation contexts}
% \begin{align*}
%   E \grmdef&
%     \EHole \grmor
%     E \; e \grmor
%     v \; E \grmor
%     \ELetU E e \grmor
%     (E,e) \grmor
%     (v,E) \grmor
%     \ELetP xy E e \grmor
%   %\\ &
%     \ESelect l E \grmor 
%     \ECase E { l_i \rightarrow e_i }
% \end{align*}
% 
% %\begin{figure}
  \declrel{Reduction relation in the equi-recursive system}[$\reduceE{p}{p}$]
  \begin{mathpar}
    \reduceruleE {
      (\ELam x e) v
    }{
      e[ v / a ]
    }

    \reduceruleE {
      (\ERec x e) v
    }{
      e[ \ERec x e / x ] \; v
    }

    \reduceruleE {
      \ELetU \EUnit e
    }{
      e
    }

    \reduceruleE {
      \ELetP {x_1} {x_2} {(v_1,v_2)} e
    }{
      e[v_1 / x_1][v_2 / x_2]
    }

    \reduceruleE[\reduceE{e_1}{e_2}]{
      E[e_1]
    }{
      E[e_2]
    }

    \reduceruleE{
      \PScope (\PPar* {\ESend v a} {\ERecv b} )
    }{
      \PScope (\PPar* {a} {(v,b)})
    }

    \reduceruleE{
      \PScope (\PPar* {\ESelect {l_j} a} {\ECase b { l_i \rightarrow e_i }})
    }{
      \PScope (\PPar* {a} {e_j \; b})
    }

    \reduceruleE{
      \PScope (\PPar* {\ETerm a} {\EWait b})
    }{
      \PPar* \EUnit \EUnit
    }

    \reduceruleE{
      E[\EFork e]
    }{
      \PScope (\PPar {E[a]} {e \; b})
    }

    \reduceruleE[\reduceE{p}{p^\prime}] { \PPar p q } { \PPar {p^\prime} q } \and
    \reduceruleE[\reduceE{p}{p^\prime}] { \PScope p } { \PScope p^\prime }   \and
    \reduceruleE[p \equiv q \\ \reduceE{q}{q^\prime}] { p } { q^\prime }
  \end{mathpar}
  Dual $\PScope[b,a]$ rules for $\EkwSend$/$\EkwRecv$, $\EkwSelect$/$\EkwCase$,
  $\EkwTerm$/$\EkwWait$ are omitted.
%   \label{fig:equi-reduction}
%   \caption{Reduction relation for the equi-recursive system}
% \end{figure}


%%% Local Variables:
%%% mode: latex
%%% TeX-master: "main"
%%% End:
