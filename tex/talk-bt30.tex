\documentclass[aspectratio=169]{beamer}
\usepackage{amsmath}
\usepackage{amssymb}
\usepackage{mathtools}
\usepackage{mathpartir}


\newtheorem{conjecture}{Conjecture}

% \usepackage{preamble}
\newcommand\note[2]{[\textcolor{red}{#1}: #2]}
\newcommand\js{\note{js}}
\newcommand\vv{\note{vv}}
\newcommand\pt{\note{pt}}
\newcommand\highlight[1]{\colorbox{yellow}{#1}}

% Macros to define type and term grammars
\newcommand\grmdef{\; \Coloneqq \;\;}
\newcommand\grmor{\; \mid \;}

% Formatting of type and expression keywords.
\newcommand\Tkw[1]{\mathtt{#1}}
\newcommand\Ekw[1]{\mathtt{#1}}
\newcommand\Econst[1]{\mathtt{#1}}

% Type syntax macros
\newcommand\TInt{\Tkw{Int}}
\newcommand\TUnit{()}
\newcommand\TPair[2]{#1 \otimes #2}
\newcommand\TFun[2]{#1 \rightarrow #2}
\newcommand\TEnd[1]{\Tkw{end #1}}
\newcommand\TIn[2]{{}?#1.#2}
\newcommand\TOut[2]{{}!#1.#2}
\newcommand\TSelect[2][i \in I]{\oplus\{\, #2 \,\}_{#1}}
\newcommand\TCase[2][i \in I]{\&\{\, #2 \,\}_{#1}}
\newcommand\TRec[1][X]{\mu #1.\;}
\newcommand\TDual[1]{\overline{#1}}

% Expression syntax macros
\newcommand\EUnit{()}
\newcommand\ELam[1]{\lambda #1.\;}
\newcommand\ERec[1]{\Ekw{rec}\; #1.\;}
\newcommand\ELet[2]{\Ekw{let}\; #1 = #2 \;\Ekw{in}\;}
\newcommand\ELetU[1]{#1;\;}%{\ELet{\EUnit}{#1}}
\newcommand\ELetP[3]{\ELet{(#1,#2)}{#3}}
\newcommand\EkwWait{\Econst{wait}}
\newcommand\EkwTerm{\Econst{term}}
\newcommand\EkwSend{\Econst{send}}
\newcommand\EkwRecv{\Econst{recv}}
\newcommand\EkwSelect{\Econst{sel}}
\newcommand\EkwRoll{\Econst{roll}}
\newcommand\EkwUnroll{\Econst{unroll}}
\newcommand\EkwFork{\Econst{fork}}
\newcommand\EWait[1]{\EkwWait\; #1}
\newcommand\ETerm[1]{\EkwTerm\; #1}
\newcommand\ESend[2]{\EkwSend\; #1 \; #2}
\newcommand\ERecv[1]{\EkwRecv\; #1}
\newcommand\ESelect[2]{\EkwSelect\; #1 \; #2}
\newcommand\ECase[2]{\Ekw{case}\; #1 \; \left\{\, #2 \,\right\}}
\newcommand\ERoll[1]{\EkwRoll\; #1}
\newcommand\EUnroll[1]{\EkwUnroll\; #1}
\newcommand\EFork[1]{\EkwFork\; #1}

\newcommand\EHole{[\;]}

% Type equivalence names
\newcommand\EqUnit{Eq-Unit}
\newcommand\EqPair{Eq-Pair}
\newcommand\EqFun{Eq-Fun}
\newcommand\EqS{Eq-S}
\newcommand\EqEnd[1]{Eq-End#1}
\newcommand\EqIn{Eq-In}
\newcommand\EqOut{Eq-Out}
\newcommand\EqSelect{Eq-Select}
\newcommand\EqCase{Eq-Case}
\newcommand\EqUnrollL{Eq-Unroll-L}
\newcommand\EqUnrollR{Eq-Unroll-R}
\newcommand\EqUnroll{Eq-Unroll}
\newcommand\EqRoll{Eq-Roll}

% Outputs a header for defining a new relation.
%   {#1} description/name
%   [#2] relation syntax, optional
\NewDocumentCommand \declrel
  { m o }
  {%
    \noindent%
    \emph{#1}%
    \IfValueT{#2}{\hfill\fbox{#2}}%
  }

% Outputs a rule for definitions of session type dualization.
%   {#1} type to be dualized
%   {#2} dualized result
\newcommand\dualdef[2]{\inferrule{\TDual{#1} {}={} #2}{}}

% Outputs a typing context, can be embellished with an index `i` like this
%
%     \Ctxt_i    or    \Ctxt_{longer_index}
\NewDocumentCommand \Ctxt
  { e{_} }
  { \IfNoValueTF{#1}{\Gamma}{\Gamma_{#1}} }

% The empty context.
\newcommand\CNil{\cdot}

% Typesets a single binding in a context
%
%   *       given => reusable binding, not given => linear binding
%   {#1}    binding name
%   {#2}    binding typing
\NewDocumentCommand \CBind
  { s m m }
  { #2 \IfBooleanTF{#1}{:^*}{:} #3 }

% A split context.
%
%   [#1]   1st sub-context, defaults to \Ctxt_1  
%   [#2]   2nd sub-context, defaults to \Ctxt_2  
\NewDocumentCommand \CSplit
  { O{\Ctxt_1} O{\Ctxt_2} }
  { #1 \circ #2 }

% Predicate on contexts: only unrrestricted bindings remaining
\newcommand\CExhausted[1][\Ctxt]{all^*(#1)}

% Macros for typing relations.
%
%   [#1]  typing context, defaults to  \Ctxt
%   {#2}  expression to type
%   {#3}  type
\newcommand\typing[3][\Ctxt]{#1 \vdash #2 : #3}

\newcommand\eqt{\sim_{\mathsf T}}
\newcommand\eqs{\sim_{\mathsf S}}

% Macros for reductions
\newcommand\reduce[3][]{#2 \stackrel{#1}\rightarrow #3}
\newcommand\reducemany[3][]{#2 \stackrel{#1}{\twoheadrightarrow} #3}
\newcommand\reduceto[3][]{#2 \stackrel{#1}\Downarrow #3}
\newcommand\reducerule[3][]{\inferrule{#1}{\reduce{#2}{#3}}}

\newcommand\PScope[1][a,b]{(\nu #1)}
\NewDocumentCommand \PPar
  { s m m }
  { \IfBooleanTF{#1}
    { \PParImpl {E_1[{#2}]} {E_2[{#3}]} }
    { \PParImpl {#2} {#3} } }
\newcommand\PParImpl[2]{#1 \| #2}

% Macros for translation
%
%   [#1] |- #2 ~> #3 : #4
%
% Context defaults to \Ctxt.
\newcommand\translation[4][\Ctxt]{#1 \vdash #2 \leadsto #3 : #4}

%%% Local Variables:
%%% mode: latex
%%% TeX-master: "main"
%%% End:

\newcommand\RuleActAppLetL{%
\ltsrule \ActAppLetL {\EApp n e} \LLet {\ELet x n \EApp x e}
}
\newcommand\RuleActAppLetR{%
\ltsrule \ActAppLetR {\EApp v n} \LLet {\ELet x n \EApp v x}
}

%%% Local Variables:
%%% mode: latex
%%% TeX-master: "main"
%%% End:


\title{A Silent Semantics for Isorecursive Session Types}
\author{Janek Spaderna, Peter Thiemann, Vasco Vasconcelos}
\institute[]{University of Freiburg, University of Lisbon}
\date{October 22, 2023}

\begin{document}

%%%%%%%%%%%%%%%%%%%%%%%%%%%%%%%%%%%%%%%%%%%%%%%%%%%%%%%%%%%%%%%%%%%%%%%%%%%%%%%%

\begin{frame}
  \titlepage
\end{frame}

%%%%%%%%%%%%%%%%%%%%%%%%%%%%%%%%%%%%%%%%%%%%%%%%%%%%%%%%%%%%%%%%%%%%%%%%%%%%%%%%

\begin{frame}
  \frametitle{Session Types with Recursion}
  \begin{itemize}
  \item needed for real-world protocols
  \item repeated communication, finite and infinite
  \item often an afterthought
  \item interesting problems
  \end{itemize}
\end{frame}

\begin{frame}
  \frametitle{Example: Duality for Recursive Session Types}
  \begin{block}<+->{Naive Definition}
    \declrel{Dualization of session types}[$\TDual S = S$]
\begin{align*}
  \TDual X &= X                               &
  \TDual{\TEnd !} &= \TEnd ?                  &
  \TDual{\TOut T S} &= \TIn T \TDual{S}       &
  \TDual{\TSelect{ l_i: S_i }} &=
    \TCase{ l_i: \TDual{S_i} }                \\
  \TDual{\TRec S} &= \TRec{\TDual S}          &
  \TDual{\TEnd ?} &= \TEnd !                  &
  \TDual{\TIn T S} &= \TOut T \TDual{S}       &
  \TDual{\TCase{ l_i: S_i }} &=
    \TSelect{ l_i: \TDual{S_i} }
\end{align*}


%%% Local Variables:
%%% mode: latex
%%% TeX-master: "main"
%%% End:

  \end{block}
  \begin{itemize}
  \item<+-> Well known to be broken [Bernardi Hennessy 2014/2016],
    unless we restrict the syntax suitably
  \end{itemize}
  \begin{block}<+->{Restricted Type Syntax}
    \footnotesize
    \declrel{Type syntax}
\begin{align*}
  T,U \grmdef&
    S^\emptyset               \grmor
    \TUnit          \grmor
    \TPair TU      \grmor
    \TFun  T U      \\
  R^{\mathcal X}, S^{\mathcal X} \grmdef&
    \TEnd !         \grmor
    \TEnd ?         \grmor
    \TOut T S^{\mathcal X}       \grmor
    \TIn  T S^{\mathcal X}       \grmor
    \TSelect{ l_i: S^{\mathcal X}_i } \grmor
    \TCase{ l_i: S^{\mathcal X}_i }   \grmor
    \TRec[X] S^{\mathcal X \cup \{X\}}      \grmor
    X^{\in \mathcal X}
\end{align*}

\declrel{Dualization of session types}[$\TDual S = S$]
\begin{align*}
  \TDual X &= X                               &
  \TDual{\TEnd !} &= \TEnd ?                  &
  \TDual{\TOut T S} &= \TIn T \TDual{S}       &
  \TDual{\TSelect{ l_i: S_i }} &=
    \TCase{ l_i: \TDual{S_i} }                \\
  \TDual{\TRec S} &= \TRec{\TDual S}          &
  \TDual{\TEnd ?} &= \TEnd !                  &
  \TDual{\TIn T S} &= \TOut T \TDual{S}       &
  \TDual{\TCase{ l_i: S_i }} &=
    \TSelect{ l_i: \TDual{S_i} }
\end{align*}

%%% Local Variables:
%%% mode: latex
%%% TeX-master: "main"
%%% End:

  \end{block}
\end{frame}

\begin{frame}
  \frametitle{Equirecursive or Isorecursive?}
  \begin{itemize}
  \item<+-> Two basic approaches to recursive types
  \item<+-> Equirecursion
    \begin{itemize}
    \item Consider a recursive type modulo unfolding: $\TRec S \equiv
      S[\TRec S/ X]$
    % \item Consider a recursive type as its infinite unfolding.
    \item No special expression syntax needed.
    \end{itemize}
  \item<+-> Isorecursion
    \begin{itemize}
    \item Consider a recursive type isomorphic to its unfolding
    \item Explicit syntax for introduction and elimination
      \begin{mathpar}
        \inferrule{
          \typingI{e}{S[\TRec[X] S / X]}
          }{
            \typingI{\ERoll e}{\TRec S}
          }

          \inferrule{
      \typingI{e}{\TRec S}
    }{
      \typingI{\EUnroll e}{S[\TRec[X] S / X]}
    }
      \end{mathpar}
    \end{itemize}
  \end{itemize}
\end{frame}

\begin{frame}[fragile,fragile]
  \frametitle{Equirecursive Session Types}
  \begin{exampleblock}<+->{Example}
    \vspace{-1.5\baselineskip}
    \begin{align*}
      S_1 &= \TRec{\TOut \TInt X} &
      S_2 &= \TRec{\TOut \TInt {\TOut \TInt X}}
    \end{align*}
    \begin{itemize}
    \item $S_1$ and $S_2$ are equivalent  modulo unfolding.
    \item The common case in session type theories.
    \item A server assuming $S_1$ can communicate with a client
      assuming $\TDual{S_2}$.
    \end{itemize}
  \end{exampleblock}
  \begin{columns}<+->
    \begin{column}{0.48\linewidth}
\begin{verbatim}
server n c =
  let c = send n c in
  server (n+1) c

\end{verbatim}
    \end{column}
    \begin{column}{0.48\linewidth}
\begin{verbatim}
client c =
  let (c, x) = recv c in
  let (c, y) = recv c in
  print (x+y); client c
\end{verbatim}
    \end{column}
  \end{columns}
\end{frame}

\begin{frame}[fragile]
  \frametitle{Isorecursive Session Types}
  \begin{itemize}
  \item<+->  $S_1 = \TRec{\TOut \TInt X}$ and $S_2 = \TRec{\TOut \TInt
      {\TOut \TInt X}}$ are \textbf{different}
  \item<+-> Server assuming $S_1$ \textbf{cannot communicate} with
    client assuming $\TDual{S_2}$:
  \end{itemize}
  \begin{columns}<+->
    \begin{column}{0.48\linewidth}
\begin{verbatim}
server n c =
  let c = unroll c in
  let c = send n c in
  server (n+1) c

\end{verbatim}
    \end{column}
    \begin{column}{0.48\linewidth}
\begin{verbatim}
client c =
  let c = unroll c in
  let (c, x) = recv c in
  let (c, y) = recv c in
  print (x+y); client c
\end{verbatim}
    \end{column}
  \end{columns}
  \begin{itemize}
  \item<+-> \dots{} because \EkwUnroll{} synchronizes!
  \end{itemize}
\end{frame}
\begin{frame}
  \frametitle{However\dots}
  \begin{itemize}
  \item<+-> we can interpose adapter processes $f_{12} : \TFun{S_1}{S_2}$ and $f_{21} : \TFun{S_2}{S_1}$
  \item<+-> for example
    \begin{align*}
  f_{12} &=       \ELam {c_1} \EFork{ \ELam {c_2} (\ERec g %{
      \begin{array}[t]{l}
      \ELam{(c_1, c_2)} \\
      \ELet{(x_2,c_2)}{\ERecv{(\EUnroll{c_2})}} \\
      \ELet{c_1}{\ESend{x_2}{(\EUnroll{c_1})}} \\
      \ELet{(x_2,c_2)}{\ERecv{({c_2})}} \\
      \ELet{c_1}{\ESend{x_2}{(\EUnroll{c_1})}} \\
        g \; ({c_1}, {c_2})) (c_1, c_2) %}
      \end{array}}
\end{align*}
\item<+-> How do we obtain this process?
  \end{itemize}
\end{frame}
\begin{frame}
  \frametitle{The Adapter Witnesses Type Equivalence}
  \begin{block}<+->{Approach}
  \begin{itemize}
  \item Define an inference system for compatibility of isorecursive
    session types
  \item (same as equivalence of equirecursive session types)
  \item coinductive
  \end{itemize}
\end{block}
\begin{block}<+->{Excerpt}
  \vspace{-2\baselineskip}
  \begin{mathpar}
    \inferrule[\EqEnd !]{
    }{
      \eqscorel{
        \ELam {(c_1, c_2)} \ELetU {\EWait {c_2}} {\ETerm {c_1}}
      }{\TEnd !}{\TEnd !}
    }

    \inferrule[\EqIn]{
      T \eqt U \\
      \eqscorel{g}{R}{S}
    }{ \eqscorel{}{\TIn {T} {R}} { \TIn {U} {S}}
    }
    \\
    \inferrule[\EqUnrollL]{
      \eqscorel[\eqsctxt , \convexpr{x} : \TRec R \eqs S]
        {g}{R[\TRec R / X]}{S}
    }{
      \eqscorel{
        \ERec x
        \ELam{(c_1, c_2)}
        g \; (\EUnroll c_1, c_2)
      }{\TRec R}{S}
    }
      
    \inferrule[\EqUnrollR]{
      \eqscorel[\eqsctxt , \convexpr{x} : R \eqs \TRec S]
        {g}{R}{S[\TRec S / X]}
    }{
      \eqscorel{
        \ERec x
        \ELam{(c_1, c_2)}
        g \; (c_1, \EUnroll c_2)
      }{R}{\TRec S}
    }
  \end{mathpar}
\end{block}
\end{frame}
\begin{frame}
  \frametitle{The Adapter Witnesses Type Equivalence (II)}
  \begin{itemize}
  \item<+-> Augment the rules with witness terms
  \end{itemize}
  \begin{block}<+->{Excerpt}\small
  \vspace{-1.5\baselineskip}
  \begin{mathpar}
    \inferrule[\EqEnd !]{
    }{
      \eqsrel[]{
        \ELam {(c_1, c_2)} \ELetU {\EWait {c_2}} {\ETerm {c_1}}
      }{\TEnd !}{\TEnd !}
    }

    \inferrule[\EqIn]{
      \convexpr{f} : T \eqt U \\
      \eqsrel[]{g}{R}{S}
    }{
      \colorlet{outer}{.}
      {\begin{array}{r@{}l}
        % \eqsctxt \vdash
            \convcolor \ELam {(c_1, c_2)}
          & \convcolor \ELetP {t} {c_1} {\ERecv{c_1}} \\
          & \convcolor \hspace{-2em} \ELet {c_2} {\ESend {f \; t} {c_2}} \\
          & \convcolor \hspace{-2em} g \; (c_1, c_2)
            \color{outer} \hspace{4em}
              : \TIn {T} {R} \eqs \TIn {U} {S}
      \end{array}}
    }
    \\
    \inferrule[\EqUnrollL]{
      \eqsrel[]
        {g}{R[\TRec R / X]}{S}
    }{
      \eqsrel[]{
        \ELam{(c_1, c_2)}
        g \; (\EUnroll c_1, c_2)
      }{\TRec R}{S}
    }
      
    \inferrule[\EqUnrollR]{
      \eqsrel[]
        {g}{R}{S[\TRec S / X]}
    }{
      \eqsrel[]{
        \ELam{(c_1, c_2)}
        g \; (c_1, \EUnroll c_2)
      }{R}{\TRec S}
    }
  \end{mathpar}
  \begin{itemize}
  \item<+-> Cute, but unsatisfactory
  \end{itemize}
  \end{block}
\end{frame}
\begin{frame}
  \frametitle{The Adapter Witnesses Type Equivalence (III)}
  \begin{itemize}
  \item<+-> Coinductive definition $\to$ infinite/circular witness terms
  \item<+-> Fortunately, we're dealing with \textbf{regular
      coinduction}
  \item<+-> Standard transformation to an inductively defined
    judgment\\
    ~[Gapeyev, Levin, Pierce] or [Dagnino 2021]
  \item<+-> Witness terms constructed with explicit recursion (see paper)
  \end{itemize}
\end{frame}
\begin{frame}
  \frametitle{Translation from Equirecursion to Isorecursion}
  \begin{itemize}
  \item<+-> Formalize ``modulo unfolding'' by an explicit conversion
    rule that applies equivalence
    \begin{mathpar}
  \inferrule{
    \typing {\tyEqui e} T \\ T \eqt U
  }{
    \typing {\tyEqui e} U
  }
\end{mathpar}
\item<+-> Extend to a translation
  \begin{mathpar}
  \inferrule{
      \translation e {e'} T \\
      \convexpr{f} : T \eqt U
    }{
      \translation e {f \; e'} U
    }
\end{mathpar}
\item<+-> A typing preserving translation!

  \end{itemize}
\end{frame}
\begin{frame}
  \frametitle{Translation from Isorecursion to Equirecursion}
  \begin{itemize}
  \item<+-> Assumption: underlying isorecursive types are contractive
  \item<+-> Translation amounts to erasing $\EkwUnroll$
  \item<+-> Typing preserving translation
  \end{itemize}
\end{frame}
\begin{frame}
  \frametitle{WIre Compatibility}
  \begin{itemize}
  \item<+-> Q: Can isorecursive and equirecursive programs talk to one
    another?
  \item<+-> What if we ask all (iso) recursive types to be
    contractive?
  \item<+-> We can silently reduce $\reduce{\EUnroll a}{a}$.
  \item<+-> Can we?
  \end{itemize}
\end{frame}
\begin{frame}
  \frametitle{WIre Compatibility, Formally}
\begin{conjecture}[V1: Simulation]
  Suppose that $\translation {e}{f}T$. \\
  If $\reduce[\alpha]{e}{e'}$ \\
  and $\translation{e'}{f'}T$, \\
  then  $\reducemany[\alpha]{f}{f'}$.
\end{conjecture}
\begin{itemize}
\item<2-> WRONG
\end{itemize}
\end{frame}
\begin{frame}
  \frametitle{WIre Compatibility, Formally (II)}
\begin{conjecture}[V2: Simulation $+$ administrative reduction]
  Suppose that $\translation {e}{f}T$. \\
  If $\reduce[\alpha]{e}{e'}$ \\
  and $\translation{e'}{f'}T$, \\
  \color{magenta}
  then $\exists f''$,  $\reducemany[\alpha]{f}{f''}$
  and $ \reducemany{f'}{f''}$.
\end{conjecture}
\begin{itemize}
\item<2-> WRONG
\end{itemize}
\end{frame}
\begin{frame}
  \frametitle{WIre Compatibility, Formally (III)}
\begin{conjecture}[V3: Simulation up to equational theory]
  Suppose that $\translation {e}{f}T$. \\
  If $\reduce[\alpha]{e}{e'}$ \\
  and $\translation{e'}{f'}T$, \\
  \color{magenta}
  then  $f \Leftrightarrow \xrightarrow{\alpha} \Leftrightarrow f'$.
\end{conjecture}
\begin{itemize}
\item<2-> Still wrong: extra equations needed beyond reductions.
\end{itemize}
\end{frame}
\begin{frame}
  \frametitle{Questions}
  \begin{itemize}
  \item Result seems so obvious \dots
  \item Is there an equational theory for functional session types?
  \item Can bisimulation help?
  \end{itemize}
\end{frame}
\end{document}

%%% Local Variables:
%%% mode: latex
%%% TeX-master: t
%%% End:
