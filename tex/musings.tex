% The type syntax for the iso-recursive system is the same as in the
% equi-recursive system, including the notion of session type dualization.
% 
% Type equivalence for the iso-recursive system is largely the same as in the
% equi-recursive system except for the rules \textsc{\EqUnrollL} and
% \textsc{\EqUnrollR}, which do not exist.

\vv{
  \begin{itemize}
  \item The dual function for recursive types is a bit more complicated; pls have
    a look at ``the final cut''.
    \pt{restricted payload type to closed types}
  \item Iso-recursive expressions $e$ must appear before type equivalence.
    \js{fixed this by moving sections around}
  \item IMO, $T,U$ reads better than $T_1, T_2$.
    \pt{sorry, messed this up!}
    \js{changed it}
  \item The expression for session types isomorphism may be curried. Then we
    seek these results, right? (do they hold?)

    \pt{Yes, see below in Sec~\ref{sec:properties}}

    Lemma. If $e : T \eqt U$, then $\typing{e}{\TFun{T}{U}}$.

    Lemma. If $e : R \eqs S$, then $\typing{e}{\TFun{R}{\TFun{\TDual S}{\TUnit}}}$.
  \item $f^{-1}$ is confusing; why not a simple $g$?
    \pt{Fixed, differently}
\item Added an easier to read, IMO, rule for \EqFun{} and (curried)
  for \EqOut
  \pt{Messed this up. Sorry again}
\item \EqPair{} uses an abbreviation, right? $\lambda p. \text{let}(x,y) = p
  \text{ in } \dots$
  \js{added note about this abbreviation next to grammar}
\item Added suggestion for roll and unroll; these constants may have function
  types. (types don't work :(
  \end{itemize}
}



TODO (PJT): what do we mean with $\cong$?
\begin{itemize}
\item for ``normal'' types, it should be contextual equivalence(?)
\item for channels, it might be bisimilarity(?)
\item might be simpler with an LTS semantics, see proposed definition:
\end{itemize}

\subsection{Attempt at Vasco's suggestion}
\label{sec:attempt-at-vascos}

\begin{lemma}[Equational correspondence for expressions]
  Suppose $\Ctxt \vdash e_1 \leadsto e'_1 : T$ where $\Ctxt$ contains only
  channel bindings.
  \begin{enumerate}
  \item
    If $\reduce[\alpha]{e_1}{e_2}$
    and $\Ctxt \vdash e_2 \leadsto e'_2 : T$, then
    $\reducemany[\alpha]{e'_1}{e'_3}$ and $e_3' = e_2'$.
  % \item If $\reduce[\alpha]{e'_1}{e'_3}$, then $\reduce[\alpha]{e_1}{e_2}$ and
  %   $\reducemany[\alpha]{e'_3}{e'_2}$ and $\Ctxt \vdash e_2 \leadsto e'_2 : T$.
  \end{enumerate}
\end{lemma}

Define $e_1 = e_2$: reflexive, transitive, congruence, symmetric
closure of all expression reductions, and fork. As well as process
reduction on restricted channels. 
\begin{mathpar}
  \reduce[a!v]{\ESend v a}{a}

  \dots

  \inferrule{
    \reduce[\alpha]{e}{e'}
  }{
    \reduce[\alpha]{E[e]}{E[e']}
  }
  \\
  \text{process reductions}
  \\
  \inferrule{
    \reduce[a!v]{p_1}{p_1'} \\
    \reduce[b?v]{p_2}{p_2'}
  }{
    \reduce[a!v, b?v]{\PPar{p_1}{p_1'} }{\PPar{p_2}{p_2'} }
  }

  \inferrule{
    \reduce[a!v, b?v]{p }{p' }
  }{
    \reduce[\tau]{\PScope p}{\PScope {p'}}
  }
\end{mathpar}

\newpage
\begin{lemma}[Operational correspondence for expressions]
  Suppose $\Ctxt \vdash e_1 \leadsto e'_1 : T$ where $\Ctxt$ contains only
  channel bindings. \begin{enumerate}
  \item If $\reduce[\alpha]{e_1}{e_2}$, then
    $\reducemany[\alpha]{e'_1}{e'_2}$ and $\Ctxt \vdash e_2 \leadsto e'_2 : T$.
  \item If $\reduce[\alpha]{e'_1}{e'_3}$, then $\reduce[\alpha]{e_1}{e_2}$ and
    $\reducemany[\alpha]{e'_3}{e'_2}$ and $\Ctxt \vdash e_2 \leadsto e'_2 : T$.
  \end{enumerate}
\end{lemma}
\begin{proof} Item 1.
  The proof is by rule induction on $\Ctxt \vdash e_1 \leadsto e'_1 : T$.
  %
  Case the top-level translation rule is
  \begin{mathpar}
    \inferrule{
      \Ctxt \vdash e \leadsto e' : T \\
      f : T \eqt U
    }{
      \Ctxt \vdash e \leadsto f e' : U
    }
  \end{mathpar}

  Then, we know (premises) that $\Ctxt \vdash e_1 \leadsto e'_1 : T$ and $f : T
  \eqt U$.
  %
  We need to show that $\reducemany[\alpha]{f e_1'}{e'_2}$ and $\Ctxt \vdash e_2 \leadsto e'_2 : T$.
  %
  At this point I suggest proceeding by rule induction on
  $\reduce[\alpha]{e_1}{e_2}$. Details to be seen.

  \pt{Agreed, this part works ok. But what follows troubles me.}

  However, if $e_1$ is a value, then so is $e_1'$ (lemma!). 

  Proceed by induction on $f : T \eqt U$?

  Or: prove that if I can observe some $\alpha$ on $e_1$, then I can
  observe the same $\alpha$ on $f e_1'$.

  \begin{itemize}
  \item $\alpha = \ELetU\EHole{e''}$
  \item $\alpha = \ELetP xy\EHole{e''}$
  \item $\alpha = \EHole\; v$
  \item $\alpha = \EWait\EHole$
  \item $\alpha = \ETerm\EHole$
  \item $\alpha = \ESend{v}\EHole$
  \item $\alpha = \ERecv\EHole$
  \item $\alpha = \ESelect l \EHole$
  \item $\alpha = \ECase\EHole{l_i: e_i}$
  \end{itemize}
  
\end{proof}


%%% Local Variables:
%%% mode: latex
%%% TeX-master: "main"
%%% End:
