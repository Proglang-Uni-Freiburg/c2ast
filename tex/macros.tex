\newcommand\note[2]{[\textcolor{red}{#1}: #2]}
\renewcommand\note[2]{}
\newcommand\js{\note{js}}
\newcommand\vv{\note{vv}}
\newcommand\pt{\note{pt}}
\newcommand\highlight[1]{\colorbox{yellow}{#1}}

% Macros to define type and term grammars
\newcommand\grmdef{\; \Coloneqq \;\;}
\newcommand\grmor{\; \mid \;}

% Formatting of type and expression keywords.
\newcommand\Tkw[1]{\mathtt{#1}}
\newcommand\Ekw[1]{\mathtt{#1}}
\newcommand\Econst[1]{\mathtt{#1}}

% Type syntax macros
\newcommand\TInt{\Tkw{Int}}
\newcommand\TUnit{()}
\newcommand\TPair[2]{#1 \otimes #2}
\newcommand\TFun[2]{#1 \multimap #2}
\newcommand\TEnd[1]{\Tkw{end #1}}
\newcommand\TIn[2]{{}?#1.#2}
\newcommand\TOut[2]{{}!#1.#2}
\newcommand\TSelect[2][i \in I]{\oplus\{\, #2 \,\}_{#1}}
\newcommand\TCase[2][i \in I]{\&\{\, #2 \,\}_{#1}}
\newcommand\TRec[1][X]{\mu #1.\;}
\newcommand\TDual[1]{\overline{#1}}

% Expression syntax macros
\newcommand\EUnit{()}
\newcommand\ELam[1]{\lambda #1.\;}
\newcommand\EApp[1]{#1 \;}
\newcommand\ERec[1]{\Ekw{rec}\; #1.\;}
\newcommand\EPair[2]{(#1,#2)}
\newcommand\ELet[2]{\Ekw{let}\; #1 = #2 \;\Ekw{in}\;}
\newcommand\ELetU[1]{#1;\;}%{\ELet{\EUnit}{#1}}
\newcommand\ELetP[3]{\ELet{(#1,#2)}{#3}}
\newcommand\EkwWait{\Econst{wait}}
\newcommand\EkwTerm{\Econst{term}}
\newcommand\EkwSend{\Econst{send}}
\newcommand\EkwRecv{\Econst{recv}}
\newcommand\EkwSelect[1][]{\Econst{sel}_{#1}}
\newcommand\EkwRoll{\Econst{roll}}
\newcommand\EkwUnroll{\Econst{unroll}}
\newcommand\EkwFork{\Econst{fork}}
\newcommand\EWait[1]{\EkwWait\; #1}
\newcommand\ETerm[1]{\EkwTerm\; #1}
\newcommand\ESend[2]{\EkwSend\; (#1,#2)}
\newcommand\ERecv[1]{\EkwRecv\; #1}
\newcommand\ESelect[1]{\EkwSelect[#1] \;}
\newcommand\ECase[2]{\Ekw{case}\; #1 \; \left\{\, #2 \,\right\}}
\newcommand\ERoll[1]{\EkwRoll\; #1}
\newcommand\EUnroll[1]{\EkwUnroll\; #1}
\newcommand\EFork[1]{\EkwFork\; #1}
\newcommand\EHole{[\;]}

\newcommand\FV[1]{\mathrm{fv}(#1)}

% Type equivalence names.
\newcommand\EqUnit{Eq-Unit}
\newcommand\EqPair{Eq-Pair}
\newcommand\EqFun{Eq-Fun}
\newcommand\EqS{Eq-S}
\newcommand\EqEnd[1]{Eq-End#1}
\newcommand\EqIn{Eq-In}
\newcommand\EqOut{Eq-Out}
\newcommand\EqSelect{Eq-Select}
\newcommand\EqCase{Eq-Case}
\newcommand\EqUnrollL{Eq-UnrollL}
\newcommand\EqUnrollR{Eq-UnrollR}
\newcommand\EqUnroll{Eq-Unroll}
\newcommand\EqRoll{Eq-Roll}

% Reduction names.
\newcommand\ActAppLetL{Act-AppLetL}
\newcommand\ActAppLetR{Act-AppLetR}
\newcommand\ActPairLetL{Act-PairLetL}
\newcommand\ActPairLetR{Act-PairLetR}
\newcommand\ActLetLet{Act-LetLet}
\newcommand\ActLetEta{Act-LetEta}

\newcommand\ActApp{Act-App}
\newcommand\ActBeta{Act-Beta}
\newcommand\ActBranch{Act-Branch}
\newcommand\ActCase{Act-Case}
\newcommand\ActClose{Act-Close}
\newcommand\ActCong{Act-Cong}
\newcommand\ActEnd{Act-End}
\newcommand\ActForkP{Act-ForkP}
\newcommand\ActFork{Act-Fork}
\newcommand\ActJoin{Act-Join}
\newcommand\ActLet{Act-Let}
\newcommand\ActLetPair{Act-LetPair}
\newcommand\ActMsg{Act-Msg}
\newcommand\ActOpen{Act-Open}
\newcommand\ActPar{Act-Par}
\newcommand\ActRecv{Act-Recv}
\newcommand\ActRec{Act-Rec}
\newcommand\ActScope{Act-Lift}
\newcommand\ActSelect{Act-Select}
\newcommand\ActSend{Act-Send}
\newcommand\ActSession{Act-Session}
\newcommand\ActTerm{Act-Term}
\newcommand\ActUnit{Act-Unit}
\newcommand\ActWait{Act-Wait}

\newcommand\ActRefl{Act-Refl}
\newcommand\ActOne{Act-One}

% Outputs a header for defining a new relation.
%   {#1} description/name
%   [#2] relation syntax, optional
\NewDocumentCommand \declrel
  { m o }
  {%
    \noindent%
    \emph{#1}%
    \IfValueT{#2}{\hfill\fbox{#2}}%
  }

% Outputs a rule for definitions of session type dualization.
%   {#1} type to be dualized
%   {#2} dualized result
\newcommand\dualdef[2]{\inferrule{\TDual{#1} {}={} #2}{}}

% Outputs a typing context, can be embellished with an index `i` like this
%
%     \Ctxt_i    or    \Ctxt_{longer_index}
\NewDocumentCommand \Ctxt
  { e{_} }
  { \IfNoValueTF{#1}{\Gamma}{\Gamma_{#1}} }

% The empty context.
\newcommand\CNil{\cdot}

% Typesets a single binding in a context
%
%   *       given => reusable binding, not given => linear binding
%   {#1}    binding name
%   {#2}    binding typing
\NewDocumentCommand \CBind
  { s m m }
  { #2 \IfBooleanTF{#1}{:^*}{:} #3 }

% A split context.
%
%   [#1]   1st sub-context, defaults to \Ctxt_1  
%   [#2]   2nd sub-context, defaults to \Ctxt_2  
\NewDocumentCommand \CSplit
  { O{\Ctxt_1} O{\Ctxt_2} }
  { #1 \circ #2 }

% Predicate on contexts: only unrrestricted bindings remaining
\newcommand\CExhausted[1][\Ctxt]{all^*(#1)}

% Macros for typing relations.
%
%   [#1]  typing context, defaults to  \Ctxt
%   {#2}  expression to type
%   {#3}  type
\newcommand\typing[3][\Ctxt]{#1 \vdash #2 : #3}
\newcommand\tyEqui{\ensuremath{_{\text{equi}}}}
\newcommand\tyIso{\ensuremath{_{\text{iso}}}}

\newcommand\Ptyping[2][\Ctxt]{#1 \vdash #2}

\newcommand\eqt{\sim_{\mathsf T}}
\newcommand\eqs{\sim_{\mathsf S}}
\newcommand\leftsim{\lesssim}
\newcommand\bisim{\approx}

% Macros for reductions
\newcommand\reduce[3][]{#2 \xrightarrow{#1} #3}
\newcommand\reducemany[3][]{#2 \stackrel{#1}{\twoheadrightarrow} #3}
\newcommand\reduceto[3][]{#2 \stackrel{#1}\Downarrow #3}
\newcommand\reducerule[3][]{\inferrule{#1}{\reduce{#2}{#3}}}
\newcommand\RPar[2]{#1 \| #2}

% \ltsrule {#1} [#2] {#3} {#4} {#3}
%
%   #1  rule name
%   #2  premise (optional)
%   #3  conclusion LHS
%   #4  conclusion transition label
%   #5  conclusion RHS
\NewDocumentCommand \ltsrule
  { m O{} m m m }
  { \inferrule[#1]{#2}{\reduce[{#4}]{#3}{#5}} }

\NewDocumentCommand \ltsrulemany
  { m O{} m m m }
  { \inferrule[#1]{#2}{\reducemany[{#4}]{#3}{#5}} }

% transition label
\newcommand\LLet{\Ekw{let}}

\newcommand\PExp[1]{\langle #1 \rangle}
\newcommand\PScope[1][a,b]{(\nu #1)}
\NewDocumentCommand \PPar
  { s m m }
  { \IfBooleanTF{#1}
    { \PParImpl {E_1[{#2}]} {E_2[{#3}]} }
    { \PParImpl {#2} {#3} } }
\newcommand\PParImpl[2]{#1 \| #2}

% Macros for translation
%
%   [#1] |- #2 ~> #3 : #4
%
% Context defaults to \Ctxt.
\newcommand\translation[4][\Ctxt]{#1 \vdash #2 \leadsto #3 : #4}
\newcommand\Ptranslation[3][\Ctxt]{#1 \vdash #2 \leadsto #3}

\newcommand\IsoToEqui[1]{|#1|}


%%% Local Variables:
%%% mode: latex
%%% TeX-master: "main"
%%% End:
