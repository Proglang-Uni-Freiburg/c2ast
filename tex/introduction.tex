
\section{Introduction}
\label{sec:introduction}

Recursion in session types is almost always interpreted as
equirecursion. For example, the two session types $T_1 = \TRec{\TOut \TInt X} $
and $T_2 = \TRec{\TOut \TInt {\TOut \TInt X}} $ are considered equal because
their unfoldings as infinite trees are equal. That means, servers with
these types are interchangeable without breaking the protocol and
functions that expect arguments of one type also accept arguments of
the other type.

If we were to consider an isorecursive interpretation, the two types
would not be equal; using values of the types would require explicit
unrolling operations; and unrolling is expected to give rise to a distinct communication
step. While channels of these two types are no longer
interchangeable, we can nevertheless define a pair of conversion
functions
$f_{12} : \TFun{T_1}{T_2}$
and
$f_{21} : \TFun{T_2}{T_1}$
that mediate between the two types. As an example, we show $f_{12}$
and omit $f_{21}$, which is very similar:
\begin{align*}
  f_{12} &=       \ELam {c_1} \EFork{ \ELam {c_2} \ERec g %{
      \begin{array}[t]{l}
      \ELam{(c_1, c_2)} \\
      \ELet{(x_2,c_2)}{\ERecv{(\EUnroll{c_2})}} \\
      \ELet{c_1}{\ESend{x_2}{(\EUnroll{c_1})}} \\
      \ELet{(x_2,c_2)}{\ERecv{({c_2})}} \\
      \ELet{c_1}{\ESend{x_2}{(\EUnroll{c_1})}} \\
        g \; ({c_1}, {c_2}) %}
      \end{array}}
\end{align*}
As the session types are recursive, so are the conversion
functions. The conversion function $f_{12}$ creates one additional
channel and forks one additional process. Moreover, it is expected to
lead to additional communication as evidenced by the folklore rule for
isorecursive processes:
\begin{gather}
  \label{eq:3}
  \reducerule{
    \PScope (\PPar* {\EUnroll a} {\EUnroll b})
  }{
    \PScope (\PPar* a b)
  }
\end{gather}
However, if the underlying recursive session type is contractive, then we can
elide this extra communication, a move which essentially reinterprets
isorecursion as equirecursion.
This move makes a type conversion like $f_{12}$ act like an identity
function, which forwards messages from one channel to another, as the
$\EUnroll\_$ operations are just identities. 


Clearly, this consideration goes both ways. If we start from an
equirecursive program, we can transform it into an isorecursive one by
making the type conversion (i.e., the unrolling of the recursion)
explicit. We can execute the isorecursive 
program either with its ``natural'' semantics using
reduction~\eqref{eq:3} or with a silent 
semantics that treats $\EUnroll\_$ as an identity.
The latter interpretation makes the equirecursive version
``wire-compatible'' with the isorecursive one.
  

%%% Local Variables:
%%% mode: latex
%%% TeX-master: "main"
%%% End:
