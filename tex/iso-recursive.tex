\section{Iso-recursion}
\label{sec:iso-recursion}

The traditional equi-recursive system can---with very few and localized changes---be turned
into an iso-recursive system. The type syntax and dualization of
session types require no adjustments.

We extend the syntax with an extra expression to unroll a recursive
type,
\begin{align*}
    e \grmdef&
               \EUnroll e
\end{align*}
%
process reduction with an extra rule:
%
\begin{mathpar}
    % \reducerule{
    %   \PScope (\PPar* {\ERoll a} {\ERoll b})
    % }{
    %   \PPar* a b
    % }
    %   
    \reducerule{
      \PScope (\PPar* {\EUnroll a} {\EUnroll b})
    }{
      \PScope (\PPar* a b)
    }
\end{mathpar}
%
and we replace the typing rule for conversion by a rule for explicit unrolling:
\begin{mathpar}
  \inferrule{
    \typing{\tyIso e}{\TRec S}
  }{
    \typing{\tyIso \EUnroll e}{S[\TRec[X] S / X]}
  }
\end{mathpar}

The example from the introduction demonstrates that the resulting
system is much more tedious. Nevertheless, we can go forth and back
between the equirecursive system and the isorecursive system under
certain mild conditions.

Readers acquainted with isorecursive typing may wonder why we do not
include a $\ERoll\_$ expression in the language. A $\ERoll\_$
expression is only needed to \emph{construct} values of recursive
type. As our language restricts recursion to session types, the only
recursively constructed objects are channels. However, they are
introduced \emph{in toto} by $\EFork\_$; it is not possible to
construct a channel piecewise. All other session operators are
eliminators that need to expose the next actionable part of the type
by unrolling.

(If we allowed recursive types outside of session types, a
$\ERoll\_$ expression would be needed.)




%%% Local Variables:
%%% mode: latex
%%% TeX-master: "main"
%%% End:
