\section{Iso-recursion}

\subsection{Types}

\declrel{Type syntax}
\begin{align*}
  T \grmdef&
    S^\emptyset               \grmor
    \TUnit          \grmor
    \TPair T T      \grmor
    \TFun  T T      \\
  S^{\mathcal X} \grmdef&
    \TEnd !         \grmor
    \TEnd ?         \grmor
    \TOut T S^{\mathcal X}       \grmor
    \TIn  T S^{\mathcal X}       \grmor
    \TSelect{ l_i: S^{\mathcal X}_i } \grmor
    \TCase{ l_i: S^{\mathcal X}_i }   \grmor
    \TRec[X] S^{\mathcal X \cup \{X\}}      \grmor
    X^{\in \mathcal X}
\end{align*}

\declrel{Dualization of session types}[$\TDual S = S$]
\begin{align*}
  \TDual X &= X                               &
  \TDual{\TEnd !} &= \TEnd ?                  &
  \TDual{\TOut T S} &= \TIn T \TDual{S}       &
  \TDual{\TSelect{ l_i: S_i }} &=
    \TCase{ l_i: \TDual{S_i} }                \\
  \TDual{\TRec S} &= \TRec{\TDual S}          &
  \TDual{\TEnd ?} &= \TEnd !                  &
  \TDual{\TIn T S} &= \TOut T \TDual{S}       &
  \TDual{\TCase{ l_i: S_i }} &=
    \TSelect{ l_i: \TDual{S_i} }
\end{align*}

% The type syntax for the iso-recursive system is the same as in the
% equi-recursive system, including the notion of session type dualization.
% 
% Type equivalence for the iso-recursive system is largely the same as in the
% equi-recursive system except for the rules \textsc{\EqUnrollL} and
% \textsc{\EqUnrollR}, which do not exist.


\subsection{Expressions}

\declrel{Syntax for constants, values, expressions and processes}
\begin{align*}
  v \grmdef&
    c                        \grmor
    (v, v)                   \grmor
    \ELam x e                \grmor
    \ERec x e                \grmor
  \\
  e \grmdef&
    v                       \grmor
    x                       \grmor
    e \; e                  \grmor
    \ELetU e e              \grmor
    (e, e)                  \grmor
    \ELetP x y e e          \grmor
  %\\ &
    \ESelect l e            \grmor
    \ECase e { l_i \rightarrow e } \grmor
    \EFork e
  \\
  p \grmdef&
    e                       \grmor
    \PPar p p               \grmor
    \PScope p
\end{align*}

The expression $\ELam {(x,y)} e$ is an abbreviation for $\ELam {x'} \ELetP xy {x'} e$
for some $x'$ not free in $e$.

% The term syntax of the iso-recursive system is the same as for the
% equi-recursive system with two additional values.
% \begin{align*}
%   v \grmdef& \dots \grmor \EkwRoll \grmor \EkwUnroll
% \end{align*}


\subsubsection{Expression typing}

\declrel{Typing contexts}
\begin{align*}
  \Ctxt \grmdef&
    \CNil \grmor \Ctxt, \CBind x T \grmor \Ctxt, \CBind* x T
\end{align*}

\declrel{Context exhaustion}[$\CExhausted$]
\begin{mathpar}
  \inferrule{ }{\CExhausted[\CNil]} \and
  \inferrule{\CExhausted}{\CExhausted[\Ctxt, \CBind* x T]}
\end{mathpar}

\declrel{Context splitting}[$\Ctxt = \CSplit[\Ctxt][\Ctxt]$]
\begin{mathpar}
  \inferrule{}{\cdot = \CSplit[\cdot][\cdot]} \and
  \inferrule{\Ctxt = \CSplit}{\Ctxt, \CBind  x T = \CSplit[(\Ctxt_1 , \CBind  x T)][ \Ctxt_2               ]} \and
  \inferrule{\Ctxt = \CSplit}{\Ctxt, \CBind  x T = \CSplit[ \Ctxt_ 1              ][(\Ctxt_2 , \CBind  x T)]} \and
  \inferrule{\Ctxt = \CSplit}{\Ctxt, \CBind* x T = \CSplit[(\Ctxt_1 , \CBind* x T)][(\Ctxt_2 , \CBind* x T)]}
\end{mathpar}

% \begin{figure}
  \declrel{Typing rules for expressions in the iso-recursive system}[$\typing{e}{T}$]
  \begin{mathpar}
    \inferrule{ 
      \CExhausted
    }{
      \typing{\EUnit}{\TUnit}
    }

    \inferrule{
      \CExhausted
    }{
      \typing[\Ctxt, \CBind x T]{x}{T}
    }

    \inferrule{
      \CExhausted
    }{
      \typing[\Ctxt, \CBind* x T]{x}{T}
    }

    \inferrule{
      \CExhausted \\
      \typing[\Ctxt, \CBind* x {\TFun T U}]{e}{\TFun T U}
    }{
      \typing{\ERec x e}{\TFun{T}{U}}
    }

    \inferrule{
      \typing[\Ctxt, \CBind x T]{e}{U}
    }{
      \typing{\ELam x e}{\TFun T U}
    }

    \inferrule{
      \typing[\Ctxt_1]{e_1}{\TFun T U}
      \\
      \typing[\Ctxt_2]{e_2}{T}
    }{
      \typing[\CSplit]{e_1 e_2}{U}
    }

    \inferrule{
      \typing[\Ctxt_1]{e_1}{\TUnit} \\
      \typing[\Ctxt_2]{e_2}{T}
    }{
      \typing[\CSplit]{
        \ELetU{e_1}{e_2}
      }{ T }
    }

    \inferrule{
      \typing[\Ctxt_1]{e_1}{T} \\
      \typing[\Ctxt_2]{e_2}{U}
    }{
      \typing[\CSplit]{(e_1,e_2)}{\TPair T U}
    }

    \inferrule{
      \typing[\Ctxt_1]{e_1}{\TPair{T_1}{T_2}} \\
      \typing[\Ctxt_2, \CBind x {T_1}, \CBind y {T_2}]{e_2}{U}
    }{
      \typing[\CSplit]{
        \ELetP xy {e_1} {e_2}
      }{ U }
    }

    \inferrule{
      \typing{e}{\TEnd ?}
    }{
      \typing{\EWait e}{\TUnit}
    }

    \inferrule{
      \typing{e}{\TEnd !}
    }{
      \typing{\ETerm e}{\TUnit}
    }

    \inferrule{
      \typing[\Ctxt_1]{e_1} T  \\
      \typing[\Ctxt_2]{e_2} {\TOut T S}
    }{
      \typing[\CSplit]{\ESend{e_1}{e_2}} S
    }

    \inferrule{
      \typing{e_1} {\TIn T S}
    }{
      \typing{\ERecv{e_1}}{\TPair T S}
    }

    \inferrule{
      \typing{e}{\TSelect{l_i : S_i}} \\
      j \in I
    }{
      \typing{\ESelect{l_j}{e}}{S_j}
    }

    \inferrule{
      \typing[\Ctxt_1]{e}{\TCase{l_i : S_i}} \\
      \typing[\Ctxt_2]{e_i}{\TFun{S_i}{T}}
    }{
      \typing[\CSplit]{
        \ECase e { l_i \rightarrow e_i }
      }{T}
    }

    \inferrule{
      \typing{e}{\TFun {\TDual S} \TUnit}
    }{
      \typing{\EFork e}{S}
    }

    \inferrule{
      \typing{e}{S[\TRec[X] S / X]}
    }{
      \typing{\ERoll e}{\TRec S}
    }
  
    \inferrule{
      \typing{e}{\TRec S}
    }{
      \typing{\EUnroll e}{S[\TRec[X] S / X]}
    }
%
%   \inferrule{
%     \typing e T \\ f : T \eqt U : f'
%   }{
%     \typing e U
%   }
  \end{mathpar}
%   \label{fig:iso-typing-rules}
%   \caption{Typing rules for expressions in the iso-recursive system}
% \end{figure}


% Typing of expressions in the iso-recursive system is largely the same as for
% expressions in the equi-recursive system. As the only change, two additional
% typing rules are introduced concerned with $\EkwRoll$ and $\EkwUnroll$.
% \begin{mathpar}
%   \inferrule{
%     \typing{e}{S[\TRec[X] S / X]}
%   }{
%     \typing{\ERoll e}{\TRec S}
%   }
% 
%   \inferrule{
%     \typing{e}{\TRec S}
%   }{
%     \typing{\EUnroll e}{S[\TRec[X] S / X]}
%   }
% \end{mathpar}


\subsubsection{Operational semantics}

\declrel{Structural congruence of processes}[$p \equiv p$]\medskip\\
The structural congruence relation on processes is defined as the smallest
congruence relation that includes the commutative monoidal rules with the
binary operator being parallel process composition $\PPar \_ \_$ and
value~$\EUnit$ as the neutral element, and scope extrusion:
\begin{align*}
  \PPar{\PScope p}{q} &\equiv \PScope (\PPar p q)
  \quad\text{if $a,b$ not free in $q$}
  \\
  \PScope p &\equiv \PScope[b,a] p
\end{align*}

\declrel{Evaluation contexts}
\begin{align*}
  E \grmdef&
    \EHole \grmor
    E \; e \grmor
    v \; E \grmor
    \ELetU E e \grmor
    (E,e) \grmor
    (v,E) \grmor
    \ELetP xy E e \grmor
  %\\ &
    \ESelect l E \grmor 
    \ECase E { l_i \rightarrow e_i }
\end{align*}

\declrel{Transition labels}
\begin{align*}
  \sigma \grmdef&
    !v \grmor ?v \grmor !l \grmor ?l \grmor ! \grmor ?
  && \text{label suffixes for session operations} \\
  \lambda \grmdef&
    a\sigma \grmor \beta \grmor  \LLet \grmor \PScope \EFork v
  &&\text{labels for expression reduction} \\
  \pi \grmdef&
               a\sigma
               \grmor \tau
               % \grmor \PScope c!a
               \grmor \RPar \pi \pi
  &&\text{labels for process reduction}
\end{align*}

\declrel{Duality for label suffixes}[$\TDual\sigma$]
\begin{align*}
  \TDual{!v} &= ?v
  &\TDual{ ?v} &= !v
  &\TDual{ !l} &= ?l
  &\TDual{ ?l} &= !l
  & \TDual ! &= ?
  &\TDual ? &= !
\end{align*}

\declrel{Free variables of process labels}[$\FV{p}$]
\begin{mathpar}
  \FV{a!v} = \FV{a?v} = \{a\} \cup \FV{v} \and
  \FV{a!l} = \FV{a?l} = \FV{a!} = \FV{a?} = \{a\} \and
  \FV{\tau} = \emptyset \and
  \FV{\PScope[c,d] a!x} = \{a,x\} \setminus \{c,d\} \and
  \FV{\RPar {\pi_1} {\pi_2}} = \FV{\pi_1} \cup \FV{\pi_2}
\end{mathpar}

\declrel{Labeled transition system for expressions (where $n$ ranges over
  non-value expressions)}[$\reduce[\lambda] e e$]
\begin{mathpar}
  \ltsrule \ActAppLetL {\EApp n e} \LLet {\ELet x n \EApp x e}

  \ltsrule \ActAppLetR {\EApp v n} \LLet {\ELet x n \EApp v x}

  \ltsrule \ActPairLetL {\EPair n e} \LLet {\ELet x n \EPair x e}

  \ltsrule \ActPairLetR {\EPair v n} \LLet {\ELet x n \EPair v x}

  \ltsrule \ActLetLet {\ELet {t_2} {(\ELet {t_1}{e_1}{e_2} )}{e_3}}
  \LLet {\ELet{t_1}{e_1}{(\ELet{t_2}{e_2}{e_3})}}

  \ltsrule \ActLetEta {\ELet t e t} \LLet e

  \ltsrule \ActApp { (\ELam x e) v } \beta { e[v / x] }

  \ltsrule \ActRec { (\ERec x v) v' } \beta { v[\ERec x v / x] v' }

  \ltsrule \ActLetBeta { \ELet x {v} e } \beta { e[v/x] }

  \ltsrule \ActLetPair { \ELetP xy {(v_1,v_2)} e } \beta { e[v_1/x, v_2/y] }

  \ltsrule \ActUnit { \ELetU \EUnit e } \beta { e }

  \ltsrule \ActSend { \ESend v a } {a!v} { a }

  \ltsrule \ActRecv { \ERecv a } {a?v} { (v,a) }

  \ltsrule \ActSelect { \ESelect l a } {a!l} { a }

  \ltsrule \ActCase { \ECase a { l_i \rightarrow v_i } } {a?l_j} { v_{\!j} \, a }

  \ltsrule \ActWait { \EWait a } {a?} { \EUnit }

  \ltsrule \ActTerm { \ETerm a } {a!} { \EUnit }

  \ltsrule \ActFork { \EFork v } { \PScope \EFork v } { a }
\end{mathpar}
The standard structural call-by-value rules are omitted.

\declrel{Labeled transition system for processes}[$\reduce[\pi] p p$]
\begin{mathpar}
  \ltsrule \ActSession [
    \reduce[a\sigma]{ e }{ e' }
  ]{ \PExp e } {a\sigma} { \PExp {e'} }

  \ltsrule \ActBeta [
    \reduce[\beta]{ e }{ e' }
  ]{ \PExp e } \tau { \PExp {e'} }

  \ltsrule \ActLet [
    \reduce[\LLet]{ e }{ e' }
  ]{ \PExp e } \tau { \PExp {e'} }

  \ltsrule \ActForkP [
    \reduce[ \PScope \EFork v ]{ e }{ e' }
  ]{ \PExp e } \tau { \PScope (\PPar {\PExp{e'}} {\PExp{v\;b}}) }

  \ltsrule \ActJoin [
    \reduce[\pi_1]{p}{p'} \\
    \reduce[\pi_2]{q}{q'} \\
  ]{ \PPar p q } {\RPar{\pi_1\!}{\pi_2}} { \PPar {p'} {q'} }

  \ltsrule \ActSync [
    \reduce[\RPar{a\sigma}{b\TDual\sigma}]{p}{p'}
  ]{ \PScope p } \tau { \PScope p' }

  % \ltsrule \ActMsg [
  %   \reduce[\RPar{a!v}{b?v}]{p}{p'}
  % ]{ \PScope p } \tau { \PScope p' }
  % 
  % \ltsrule \ActBranch [
  %   \reduce[\RPar{a!l}{b?l}]{p}{p'}
  % ]{ \PScope p } \tau { \PScope p' }
  % 
  % \ltsrule \ActEnd [
  %   \reduce[\RPar{a!}{b?}]{p}{p'}
  % ]{ \PScope p } \tau { p' }
  % 
  \ltsrule \ActScope [
    \reduce[\pi]{p}{p'} \\
    a,b \not\in \FV{\pi}
  ]{ \PScope p } \pi { \PScope p' }

  \ltsrule \ActPar [
    \reduce[\pi]{p}{p'}
  ]{ \PPar p q } \pi { \PPar {p'} q }

  \ltsrule \ActCong [
    p \equiv q \\
    \reduce[\pi]{q}{q'}
  ]{ p } \pi { q' }
\end{mathpar}

We define multistep labeled reductions in the usual way by
concatenating the labels using $\_\cdot\_$ as the concatenation
operator. We treat the silent label $\tau$ as a 
neutral element in the concatenation, i.e., $\tau \cdot \overline\pi = \overline\pi
\cdot \tau = \overline\pi$, where $\overline\pi$ is a sequence of labels. 
\begin{mathpar}
  \ltsrulemany \ActRefl {p} \tau p

  \ltsrulemany \ActOne[
  \reduce[\pi] p {p'} \\
  \reducemany[\overline\pi] {p'} {p''}
  ] p {\pi \cdot \overline\pi } {p''}
\end{mathpar}
We write $\reducemany[\alpha] p {p'}$ to indicate that the
$\alpha$-labeled transition comes from the final $\textsc{\ActOne}$ step in the
reduction sequence and all preceding transitions are $\tau$ transitions.

%%% Local Variables:
%%% mode: latex
%%% TeX-master: "main"
%%% End:


% Evaluation contexts for the iso-recursive system have the same shape as the
% contexts in the equi-recursive system. The reduction relation is extended by
% two rules concerned with $\EkwRoll$ and $\EkwUnroll$.
% \begin{mathpar}
%   \reducerule{
%     \PScope (\PPar {E[\ERoll a]} {p})
%   }{
%     \PScope (\PPar {E[a]} {p})
%   }
% 
%   \reducerule{
%     \PScope (\PPar {E[\EUnroll a]} {p})
%   }{
%     \PScope (\PPar {E[a]} {p})
%   }
% \end{mathpar}
% Dual $\PScope[b,a]$ rules omitted.
