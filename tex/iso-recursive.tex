\section{Iso-recursion}

\subsection{Types}

The type syntax for the iso-recursive system is the same as in the
equi-recursive system, including the notion of session type dualization.

Type equivalence for the iso-recursive system is largely the same as in the
equi-recursive system except for the rules \textsc{\EqUnrollL} and
\textsc{\EqUnrollR}, which do not exist.


\subsection{Expressions}

The term syntax of the iso-recursive system is the same as for the
equi-recursive system with two additional values.
\begin{align*}
  v \grmdef& \dots \grmor \EkwRoll \grmor \EkwUnroll
\end{align*}


\subsubsection{Expression typing}

Typing of expressions in the iso-recursive system is largely the same as for
expressions in the equi-recursive system. As the only change, two additional
typing rules are introduced concerned with $\EkwRoll$ and $\EkwUnroll$.
\begin{mathpar}
  \inferrule{
    \typingI{e}{S[\TRec[X] S / X]}
  }{
    \typingI{\ERoll e}{\TRec S}
  }

  \inferrule{
    \typingI{e}{\TRec S}
  }{
    \typingI{\EUnroll e}{S[\TRec[X] S / X]}
  }
\end{mathpar}


\subsubsection{Operational semantics}

Evaluation contexts for the iso-recursive system have the same shape as the
contexts in the equi-recursive system. The reduction relation is extended by
two rules concerned with $\EkwRoll$ and $\EkwUnroll$.
\begin{mathpar}
  \reduceruleI{
    \PScope (\PPar {E[\ERoll a]} {p})
  }{
    \PScope (\PPar {E[a]} {p})
  }

  \reduceruleI{
    \PScope (\PPar {E[\EUnroll a]} {p})
  }{
    \PScope (\PPar {E[a]} {p})
  }
\end{mathpar}
Dual $\PScope[b,a]$ rules omitted.

