\section{Iso-recursion}

\subsection{Types}

\declrel{Type syntax}
\begin{align*}
  T \grmdef&
    S^\emptyset               \grmor
    \TUnit          \grmor
    \TPair T T      \grmor
    \TFun  T T      \\
  S^{\mathcal X} \grmdef&
    \TEnd !         \grmor
    \TEnd ?         \grmor
    \TOut T S^{\mathcal X}       \grmor
    \TIn  T S^{\mathcal X}       \grmor
    \TSelect{ l_i: S^{\mathcal X}_i } \grmor
    \TCase{ l_i: S^{\mathcal X}_i }   \grmor
    \TRec[X] S^{\mathcal X \cup \{X\}}      \grmor
    X^{\in \mathcal X}
\end{align*}

\declrel{Dualization of session types}[$\TDual S = S$]
\begin{align*}
  \TDual X &= X                               &
  \TDual{\TEnd !} &= \TEnd ?                  &
  \TDual{\TOut T S} &= \TIn T \TDual{S}       &
  \TDual{\TSelect{ l_i: S_i }} &=
    \TCase{ l_i: \TDual{S_i} }                \\
  \TDual{\TRec S} &= \TRec{\TDual S}          &
  \TDual{\TEnd ?} &= \TEnd !                  &
  \TDual{\TIn T S} &= \TOut T \TDual{S}       &
  \TDual{\TCase{ l_i: S_i }} &=
    \TSelect{ l_i: \TDual{S_i} }
\end{align*}

% The type syntax for the iso-recursive system is the same as in the
% equi-recursive system, including the notion of session type dualization.
% 
% Type equivalence for the iso-recursive system is largely the same as in the
% equi-recursive system except for the rules \textsc{\EqUnrollL} and
% \textsc{\EqUnrollR}, which do not exist.


\subsection{Expressions}

\declrel{Syntax for constants, values, expressions and processes}
\begin{align*}
  v \grmdef&
    c                        \grmor
    (v, v)                   \grmor
    \ELam x e                \grmor
    \ERec x e                \grmor
  \\
  e \grmdef&
    v                       \grmor
    x                       \grmor
    e \; e                  \grmor
    \ELetU e e              \grmor
    (e, e)                  \grmor
    \ELetP x y e e          \grmor
  %\\ &
    \ESelect l e            \grmor
    \ECase e { l_i \rightarrow e } \grmor
    \EFork e
  \\
  p \grmdef&
    e                       \grmor
    \PPar p p               \grmor
    \PScope p
\end{align*}

The expression $\ELam {(x,y)} e$ is an abbreviation for $\ELam {x'} \ELetP xy {x'} e$
for some $x'$ not free in $e$.

% The term syntax of the iso-recursive system is the same as for the
% equi-recursive system with two additional values.
% \begin{align*}
%   v \grmdef& \dots \grmor \EkwRoll \grmor \EkwUnroll
% \end{align*}


\subsubsection{Expression typing}

\declrel{Typing contexts}
\begin{align*}
  \Ctxt \grmdef&
    \CNil \grmor \Ctxt, \CBind x T \grmor \Ctxt, \CBind* x T
\end{align*}

\declrel{Context exhaustion}[$\CExhausted$]
\begin{mathpar}
  \inferrule{ }{\CExhausted[\CNil]} \and
  \inferrule{\CExhausted}{\CExhausted[\Ctxt, \CBind* x T]}
\end{mathpar}

\declrel{Context splitting}[$\Ctxt = \CSplit[\Ctxt][\Ctxt]$]
\begin{mathpar}
  \inferrule{}{\cdot = \CSplit[\cdot][\cdot]} \and
  \inferrule{\Ctxt = \CSplit}{\Ctxt, \CBind  x T = \CSplit[(\Ctxt_1 , \CBind  x T)][ \Ctxt_2               ]} \and
  \inferrule{\Ctxt = \CSplit}{\Ctxt, \CBind  x T = \CSplit[ \Ctxt_ 1              ][(\Ctxt_2 , \CBind  x T)]} \and
  \inferrule{\Ctxt = \CSplit}{\Ctxt, \CBind* x T = \CSplit[(\Ctxt_1 , \CBind* x T)][(\Ctxt_2 , \CBind* x T)]}
\end{mathpar}

% \begin{figure}
  \declrel{Typing rules for expressions in the iso-recursive system}[$\typingE{e}{T}$]
  \begin{mathpar}
    \inferrule{ }{\typingE{\EUnit}{\TUnit}}

    \inferrule{
      \CExhausted
    }{
      \typingE[\Ctxt, \CBind x T]{x}{T}
    }

    \inferrule{
      \CExhausted
    }{
      \typingE[\Ctxt, \CBind* x T]{x}{T}
    }

    \inferrule{
      \CExhausted \\
      \typingE[\Ctxt, \CBind* x {\TFun {T_1} {T_2}}]{e}{\TFun {T_1} {T_2}}
    }{
      \typingE{\ERec x e}{\TFun{T_1}{T_2}}
    }

    \inferrule{
      \typingE[\Ctxt, \CBind x {T_1}]{e}{T_2}
    }{
      \typingE{\ELam x e}{\TFun{T_1}{T_2}}
    }

    \inferrule{
      \typingE[\Ctxt_1]{e_1}{\TFun {T_1}{T_2}}
      \\
      \typingE[\Ctxt_2]{e_2}{T_1}
    }{
      \typingE[\CSplit]{e_1 e_2}{T_2}
    }

    \inferrule{
      \typingE[\Ctxt_1]{e_1}{T_1} \\
      \typingE[\Ctxt_2]{e_2}{T_2}
    }{
      \typingE[\CSplit]{
        \ELetU{e_1}{e_2}
      }{ T_2 }
    }

    \inferrule{
      \typingE[\Ctxt_1]{e_1}{T_1} \\
      \typingE[\Ctxt_2]{e_2}{T_2}
    }{
      \typingE[\CSplit]{(e_1,e_2)}{\TPair{T_1}{T_2}}
    }

    \inferrule{
      \typingE[\Ctxt_1]{e_1}{\TPair{T_1}{T_2}} \\
      \typingE[\Ctxt_2, \CBind x {T_1}, \CBind y {T_2}]{e_2}{T_3}
    }{
      \typingE[\CSplit]{
        \ELetP xy {e_1} {e_2}
      }{ T_3 }
    }

    \inferrule{
      \typingE{e}{\TEnd ?}
    }{
      \typingE{\EWait e}{\TUnit}
    }

    \inferrule{
      \typingE{e}{\TEnd !}
    }{
      \typingE{\ETerm e}{\TUnit}
    }

    \inferrule{
      \typingE[\Ctxt_1]{e_1} T  \\
      \typingE[\Ctxt_2]{e_2} {\TOut T S}
    }{
      \typingE[\CSplit]{\ESend{e_1}{e_2}} S
    }

    \inferrule{
      \typingE{e_1} {\TIn T S}
    }{
      \typingE{\ERecv{e_1}}{\TPair T S}
    }

    \inferrule{
      \typingE{e}{\TSelect{l_i : S_i}} \\
      j \in I
    }{
      \typingE{\ESelect{l_j}{e}}{S_j}
    }

    \inferrule{
      \typingE[\Ctxt_1]{e}{\TCase{l_i : S_i}} \\
      \typingE[\Ctxt_2]{e_i}{\TFun{S_i}{T}}
    }{
      \typingE[\CSplit]{
        \ECase e { l_i \rightarrow e_i }
      }{T}
    }

    \inferrule{
      \typingE{e}{\TFun {\TDual S} \TUnit}
    }{
      \typingE{\EFork e}{S}
    }

    % \inferrule{
    %   \typingI{e}{S[\TRec[X] S / X]}
    % }{
    %   \typingI{\ERoll e}{\TRec S}
    % }
    %   
    \inferrule{
      \typingI{e}{\TRec S}
    }{
      \typingI{\EUnroll e}{S[\TRec[X] S / X]}
    }
%
%   \inferrule{
%     \typingE e T \\ f : T \eqt U : f'
%   }{
%     \typingE e U
%   }
  \end{mathpar}
%   \label{fig:iso-typing-rules}
%   \caption{Typing rules for expressions in the iso-recursive system}
% \end{figure}

%%% Local Variables:
%%% mode: latex
%%% TeX-master: "main"
%%% End:


% Typing of expressions in the iso-recursive system is largely the same as for
% expressions in the equi-recursive system. As the only change, two additional
% typing rules are introduced concerned with $\EkwRoll$ and $\EkwUnroll$.
% \begin{mathpar}
%   \inferrule{
%     \typingI{e}{S[\TRec[X] S / X]}
%   }{
%     \typingI{\ERoll e}{\TRec S}
%   }
% 
%   \inferrule{
%     \typingI{e}{\TRec S}
%   }{
%     \typingI{\EUnroll e}{S[\TRec[X] S / X]}
%   }
% \end{mathpar}


\subsubsection{Operational semantics}

\declrel{Structural congruence of processes}[$p \equiv p$]\medskip\\
The structural congruence relation on processes is defined as the smallest
congruence relation that includes the commutative monoidal rules with the
binary operator being parallel process composition $\PPar \_ \_$ and
value~$\EUnit$ as the neutral element, and scope extrusion:
\begin{align*}
  \PPar{\PScope p}{q} &\equiv \PScope (\PPar p q)
  \quad\text{if $a,b$ not free in $q$}
  \\
  \PScope p &\equiv \PScope[b,a] p
\end{align*}

\declrel{Evaluation contexts}
\begin{align*}
  E \grmdef&
    \EHole \grmor
    E \; e \grmor
    v \; E \grmor
    \ELetU E e \grmor
    (E,e) \grmor
    (v,E) \grmor
    \ELetP xy E e \grmor
  %\\ &
    \ESelect l E \grmor 
    \ECase E { l_i \rightarrow e_i }
\end{align*}

%\begin{figure}
  \declrel{Expression reduction}[$\reduceE{e}{e}$]
  \begin{mathpar}
    \reduceruleI {
      (\ELam x e) v
    }{
      e[ v / a ]
    }

    \reduceruleI {
      (\ERec x e) v
    }{
      e[ \ERec x e / x ] \; v
    }

    \reduceruleI {
      \ELetU \EUnit e
    }{
      e
    }

    \reduceruleI {
      \ELetP {x_1} {x_2} {(v_1,v_2)} e
    }{
      e[v_1 / x_1][v_2 / x_2]
    }

    \reduceruleI[\reduceE{e_1}{e_2}]{
      E[e_1]
    }{
      E[e_2]
    }
  \end{mathpar}
  \declrel{Process reduction (iso-recursive system)}[$\reduceE{p}{p}$]
  \begin{mathpar}
    \reduceruleI{
      \PScope (\PPar* {\ESend v a} {\ERecv b} )
    }{
      \PScope (\PPar* {a} {(v,b)})
    }

    \reduceruleI{
      \PScope (\PPar* {\ESelect {l_j} a} {\ECase b { l_i \rightarrow e_i }})
    }{
      \PScope (\PPar* {a} {e_j \; b})
    }

    \reduceruleI{
      \PScope (\PPar* {\ETerm a} {\EWait b})
    }{
      \PPar* \EUnit \EUnit
    }

    \reduceruleI{
      E[\EFork e]
    }{
      \PScope (\PPar {E[a]} {e \; b})
    }

    \reduceruleI{
      \PScope (\PPar {E[\ERoll a]} {p})
    }{
      \PScope (\PPar {E[a]} {p})
    }
  
    \reduceruleI{
      \PScope (\PPar {E[\EUnroll a]} {p})
    }{
      \PScope (\PPar {E[a]} {p})
    }

    \reduceruleI[\reduceE{p}{p^\prime}] { \PPar p q } { \PPar {p^\prime} q } \and
    \reduceruleI[\reduceE{p}{p^\prime}] { \PScope p } { \PScope p^\prime }   \and
    \reduceruleI[p \equiv q \\ \reduceE{q}{q^\prime}] { p } { q^\prime }
  \end{mathpar}
%   \label{fig:equi-reduction}
%   \caption{Reduction relation for the equi-recursive system}
% \end{figure}

%%% Local Variables:
%%% mode: latex
%%% TeX-master: "main"
%%% End:


% Evaluation contexts for the iso-recursive system have the same shape as the
% contexts in the equi-recursive system. The reduction relation is extended by
% two rules concerned with $\EkwRoll$ and $\EkwUnroll$.
% \begin{mathpar}
%   \reduceruleI{
%     \PScope (\PPar {E[\ERoll a]} {p})
%   }{
%     \PScope (\PPar {E[a]} {p})
%   }
% 
%   \reduceruleI{
%     \PScope (\PPar {E[\EUnroll a]} {p})
%   }{
%     \PScope (\PPar {E[a]} {p})
%   }
% \end{mathpar}
% Dual $\PScope[b,a]$ rules omitted.
