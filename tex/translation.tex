\section{Translating equi-recursion to iso-recursion}

The set of syntactically valid expressions in the equi-recursive system is a
subset of the syntactically valid expressions in the iso-recursive system. A
simple one-to-one mapping, however, does not preserve well-typedness because
the target type system is more strict.

\declrel{Translation from equi-recursive to iso-recursive expressions}[$\dtrans{e}=e$]
\begin{mathpar}
  \dtranslhs{ }{ \typing \EUnit \TUnit } = \EUnit \and
  \dtranslhs{ }{ \typing x T } = x

  \dtranslhs{
    \dtbind[\Ctxt,\CBind x T] e U
  }{
    \typing {\ELam x e} {\TFun T U}
  } = \ELam x \dtvar e

  \dtranslhs{
    \dtbind[\Ctxt,\CBind x T] e U
  }{
    \typing {\ERec x e} {\TFun T U}
  } = \ERec x \dtrans e

  \dtranslhs{
    \dtbind[\Ctxt_1] {e_1} {\TFun T U} \\
    \dtbind[\Ctxt_2] {e_2} T
  }{
    \typing[\CSplit] {e_1 \; e_2} U
  } = \dtvar{e_1} \; \dtvar{e_2}

  \dtranslhs{
    \dtbind[\Ctxt_1] {e_1} \TUnit \\
    \dtbind[\Ctxt_2] {e_2} T
  }{
    \typing[\CSplit] {\ELetU {e_1} {e_2}} T
  } = \ELetU {\dtvar{e_1}} {\dtvar{e_2}}

  \dtranslhs{
    \dtbind[\Ctxt_1] {e_1} T \\
    \dtbind[\Ctxt_2] {e_2} U
  }{
    \typing[\CSplit] {(e_1,e_2)} {\TPair T U}
  } = \left( \dtvar{e_1},\dtvar{e_2} \right)

  \dtranslhs{
    \dtbind[\Ctxt_1] {e_1} {\TPair T_1 T_2} \\
    \dtbind[\Ctxt_2, \CBind x T, \CBind y U] {e_2} U
  }{
    \typing[\CSplit]{\ELetP xy {e_1} {e_2}} U
  } = \ELetP xy {\dtvar{e_1}} {\dtvar{e_2}}

  \dtranslhs{
    \dtbind e {\TEnd ?}
  }{
    \typing {\EWait e} \TUnit
  } = \EWait {\dtvar e}

  \dtranslhs{
    \dtbind e {\TEnd !}
  }{
    \typing {\ETerm e} \TUnit
  } = \ETerm {\dtvar e}

  \dtranslhs{
    \dtbind {e_1} T \\
    \dtbind {e_2} {\TOut T S}
  }{
    \typing[\CSplit] {\ESend {e_1} {e_2}} S
  } = \ESend {\dtvar{e_1}} {\dtvar{e_2}}

  \dtranslhs{
    \dtbind e {\TIn T S}
  }{
    \typing {\ERecv e} {(T,S)}
  } = \ERecv {\dtvar e}

  \dtranslhs{
    \dtbind e {\TSelect{l_i : S_i}} \\
    j \in I
  }{
    \typing{\ESelect{l_j}{e}}{S_j}
  } = \ESelect {l_j} {\dtvar e}

  \dtranslhs{
    \dtbind[\Ctxt_1]{e}{\TCase{l_i : S_i}} \\
    \dtbind[\Ctxt_2]{e_i}{\TFun{S_i}{T}}
  }{
    \typing[\CSplit]{
      \ECase e { l_i \rightarrow e_i }
    }{T}
  } = \ECase {\dtvar e} { l_i \rightarrow \dtvar{e_i} }

  \dtranslhs{
    \dtbind e {\TFun {\TDual S} \TUnit}
  }{
    \typing{\EFork e}{S}
  } = \EFork {\dtvar e}

  \dtranslhs{
    \dtbind e T \\
    f : T \eqt U : f'
  }{
    \typing e U
  } = f \; \dtvar e
\end{mathpar}


Instead, the translation judgment $\translation e {e'} T$ in
Figure~\ref{fig:translation} describes the translation of an equi-recursively
typed expression $e$ to an equivalent expression $e'$ in the iso-recursive
system. The only interesting part of the translation process is where the
equi-recursive system makes use of type conversions. Here, an appriopriate
conversion function is inserted on the iso-recursive side.
\begin{mathpar}
  \inferrule{
      \translation e {e'} T \\
      f : T \eqt U
    }{
      \translation e {f \; e'} U
    }
\end{mathpar}

The translation preserves well-typedness and the original and the
resulting process are \highlight{``wire-compatible''}. \js{what is the correct
word?} \pt{I've used that word, but the actual technical content is more tricky
than I imagined.}

%%% Local Variables:
%%% mode: latex
%%% TeX-master: "main"
%%% End:
