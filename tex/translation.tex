\section{Translating equi-recursion to iso-recursion}

The set of syntactically valid expressions in the equi-recursive system is a
subset of the syntactically valid expressions in the iso-recursive system. A
simple one-to-one mapping, however, does not preserve well-typedness because
the target type system is more strict.

Instead, we describe a translation which uses a derivation tree to translate
from equi-recursive typed expressions to iso-recursive typed expressions. The
translation preserves well-typedness and the original and the resulting process
are \highlight{``wire-compatibility''}. \js{what is the correct word?}

A derivation tree in the equi-recursive system with the overall conclusion
$\typingE e T$ can be translated to the iso-recursive system. The function
$\llbracket \_ \rrbracket$ performs the translation; its argument is a full
derivation tree. $\derivTree e T$ describes the set of derivation trees with
the overall conclusion $\typingE e T$.

\declrel{Translation from equi-recursive to iso-recursive expressions}[$\dtrans{e}=e$]
\begin{mathpar}
  \dtranslhs{ }{ \typing \EUnit \TUnit } = \EUnit \and
  \dtranslhs{ }{ \typing x T } = x

  \dtranslhs{
    \dtbind[\Ctxt,\CBind x T] e U
  }{
    \typing {\ELam x e} {\TFun T U}
  } = \ELam x \dtvar e

  \dtranslhs{
    \dtbind[\Ctxt,\CBind x T] e U
  }{
    \typing {\ERec x e} {\TFun T U}
  } = \ERec x \dtrans e

  \dtranslhs{
    \dtbind[\Ctxt_1] {e_1} {\TFun T U} \\
    \dtbind[\Ctxt_2] {e_2} T
  }{
    \typing[\CSplit] {e_1 \; e_2} U
  } = \dtvar{e_1} \; \dtvar{e_2}

  \dtranslhs{
    \dtbind[\Ctxt_1] {e_1} \TUnit \\
    \dtbind[\Ctxt_2] {e_2} T
  }{
    \typing[\CSplit] {\ELetU {e_1} {e_2}} T
  } = \ELetU {\dtvar{e_1}} {\dtvar{e_2}}

  \dtranslhs{
    \dtbind[\Ctxt_1] {e_1} T \\
    \dtbind[\Ctxt_2] {e_2} U
  }{
    \typing[\CSplit] {(e_1,e_2)} {\TPair T U}
  } = \left( \dtvar{e_1},\dtvar{e_2} \right)

  \dtranslhs{
    \dtbind[\Ctxt_1] {e_1} {\TPair T_1 T_2} \\
    \dtbind[\Ctxt_2, \CBind x T, \CBind y U] {e_2} U
  }{
    \typing[\CSplit]{\ELetP xy {e_1} {e_2}} U
  } = \ELetP xy {\dtvar{e_1}} {\dtvar{e_2}}

  \dtranslhs{
    \dtbind e {\TEnd ?}
  }{
    \typing {\EWait e} \TUnit
  } = \EWait {\dtvar e}

  \dtranslhs{
    \dtbind e {\TEnd !}
  }{
    \typing {\ETerm e} \TUnit
  } = \ETerm {\dtvar e}

  \dtranslhs{
    \dtbind {e_1} T \\
    \dtbind {e_2} {\TOut T S}
  }{
    \typing[\CSplit] {\ESend {e_1} {e_2}} S
  } = \ESend {\dtvar{e_1}} {\dtvar{e_2}}

  \dtranslhs{
    \dtbind e {\TIn T S}
  }{
    \typing {\ERecv e} {(T,S)}
  } = \ERecv {\dtvar e}

  \dtranslhs{
    \dtbind e {\TSelect{l_i : S_i}} \\
    j \in I
  }{
    \typing{\ESelect{l_j}{e}}{S_j}
  } = \ESelect {l_j} {\dtvar e}

  \dtranslhs{
    \dtbind[\Ctxt_1]{e}{\TCase{l_i : S_i}} \\
    \dtbind[\Ctxt_2]{e_i}{\TFun{S_i}{T}}
  }{
    \typing[\CSplit]{
      \ECase e { l_i \rightarrow e_i }
    }{T}
  } = \ECase {\dtvar e} { l_i \rightarrow \dtvar{e_i} }

  \dtranslhs{
    \dtbind e {\TFun {\TDual S} \TUnit}
  }{
    \typing{\EFork e}{S}
  } = \EFork {\dtvar e}

  \dtranslhs{
    \dtbind e T \\
    f : T \eqt U : f'
  }{
    \typing e U
  } = f \; \dtvar e
\end{mathpar}


% \begin{align*}
%   \dtransDef[\TUnit] \EUnit &= \EUnit &
%   \dtransDef x &= x \\
%   \dtransDef {\ELam x e} &= \ELam x \dtrans e &
%   \dtransDef {\ERec x e} &= \ERec x \dtrans e \\
%   \dtransDef {e_1 \; e_2} &= \dtrans{e_1} \; \dtrans{e_2} &
%   \dtransDef {\ELetU {e_1} {e_2}} &= \ELetU {\dtrans{e_1}} {\dtrans{e_2}} \\
%   \dtransDef {(e_1,e_2)} &= (\dtrans{e_1},\dtrans{e_2}) &
%   \dtransDef {\ELetP xy {e_1} {e_2}} &= \ELetP xy {\dtrans{e_1}} \dtrans{e_2} \\
% \end{align*}
% 
% \declrel{Typing derivation translation}
% \begin{mathpar}
%   \derivTransRule{ }{\typingE \EUnit \TUnit}{\EUnit} \and
%   \derivTransRule{ }{\typingE x T}{x} \and
%   
%   \derivTransRule{\DT e}{
%     \typingE{\ELam x e}{U}
%   }{
%     \ELam x \derivTrans e
%   }
% 
%   \derivTransRule{\DT e}{
%     \typingE{\ERec x e}{U}
%   }{
%     \ERec x \derivTrans e
%   }
% 
%   \derivTransRule{\DT {e_1} \\ \DT {e_2}}{
%     \typingE[\CSplit]{e_1 e_2}{T_2}
%   }{
%     \derivTrans{e_1} \derivTrans{e_2}
%   }
% 
%   \derivTransRule{\DT {e_1} \\ \DT {e_2}}{
%     \typingE{ \ELetU{e_1}{e_2} }{ T_2 }
%   }{
%     \ELetU {\derivTrans {e_1}} {\derivTrans {e_2}}
%   }
% 
%   \derivTransRule{\DT e \\ f : T \eqt U : f'}{
%     \typingE e U
%   }{
%     f \; \derivTrans e
%   }
% \end{mathpar}
