\section{Expression typing}

\declrel{Typing contexts}
\begin{align*}
  \Ctxt \grmdef&
    \CNil \grmor \Ctxt, \CBind x T \grmor \Ctxt, \CBind* x T
\end{align*}

\declrel{Context exhaustion}[$\CExhausted$]
\begin{mathpar}
  \inferrule{ }{\CExhausted[\CNil]} \and
  \inferrule{\CExhausted}{\CExhausted[\Ctxt, \CBind* x T]}
\end{mathpar}

\declrel{Context splitting}[$\Ctxt = \CSplit[\Ctxt][\Ctxt]$]
\begin{mathpar}
  \inferrule{}{\cdot = \CSplit[\cdot][\cdot]} \and
  \inferrule{\Ctxt = \CSplit}{\Ctxt, \CBind  x T = \CSplit[(\Ctxt_1 , \CBind  x T)][ \Ctxt_2               ]} \and
  \inferrule{\Ctxt = \CSplit}{\Ctxt, \CBind  x T = \CSplit[ \Ctxt_ 1              ][(\Ctxt_2 , \CBind  x T)]} \and
  \inferrule{\Ctxt = \CSplit}{\Ctxt, \CBind* x T = \CSplit[(\Ctxt_1 , \CBind* x T)][(\Ctxt_2 , \CBind* x T)]}
\end{mathpar}

% \begin{figure}
  \declrel{Typing rules for expressions in the iso-recursive system}[$\typing{e}{T}$]
  \begin{mathpar}
    \inferrule{ 
      \CExhausted
    }{
      \typing{\EUnit}{\TUnit}
    }

    \inferrule{
      \CExhausted
    }{
      \typing[\Ctxt, \CBind x T]{x}{T}
    }

    \inferrule{
      \CExhausted
    }{
      \typing[\Ctxt, \CBind* x T]{x}{T}
    }

    \inferrule{
      \CExhausted \\
      \typing[\Ctxt, \CBind* x {\TFun T U}]{e}{\TFun T U}
    }{
      \typing{\ERec x e}{\TFun{T}{U}}
    }

    \inferrule{
      \typing[\Ctxt, \CBind x T]{e}{U}
    }{
      \typing{\ELam x e}{\TFun T U}
    }

    \inferrule{
      \typing[\Ctxt_1]{e_1}{\TFun T U}
      \\
      \typing[\Ctxt_2]{e_2}{T}
    }{
      \typing[\CSplit]{e_1 e_2}{U}
    }

    \inferrule{
      \typing[\Ctxt_1]{e_1}{\TUnit} \\
      \typing[\Ctxt_2]{e_2}{T}
    }{
      \typing[\CSplit]{
        \ELetU{e_1}{e_2}
      }{ T }
    }

    \inferrule{
      \typing[\Ctxt_1]{e_1}{T} \\
      \typing[\Ctxt_2]{e_2}{U}
    }{
      \typing[\CSplit]{(e_1,e_2)}{\TPair T U}
    }

    \inferrule{
      \typing[\Ctxt_1]{e_1}{\TPair{T_1}{T_2}} \\
      \typing[\Ctxt_2, \CBind x {T_1}, \CBind y {T_2}]{e_2}{U}
    }{
      \typing[\CSplit]{
        \ELetP xy {e_1} {e_2}
      }{ U }
    }

    \inferrule{
      \typing{e}{\TEnd ?}
    }{
      \typing{\EWait e}{\TUnit}
    }

    \inferrule{
      \typing{e}{\TEnd !}
    }{
      \typing{\ETerm e}{\TUnit}
    }

    \inferrule{
      \typing[\Ctxt_1]{e_1} T  \\
      \typing[\Ctxt_2]{e_2} {\TOut T S}
    }{
      \typing[\CSplit]{\ESend{e_1}{e_2}} S
    }

    \inferrule{
      \typing{e_1} {\TIn T S}
    }{
      \typing{\ERecv{e_1}}{\TPair T S}
    }

    \inferrule{
      \typing{e}{\TSelect{l_i : S_i}} \\
      j \in I
    }{
      \typing{\ESelect{l_j}{e}}{S_j}
    }

    \inferrule{
      \typing[\Ctxt_1]{e}{\TCase{l_i : S_i}} \\
      \typing[\Ctxt_2]{e_i}{\TFun{S_i}{T}}
    }{
      \typing[\CSplit]{
        \ECase e { l_i \rightarrow e_i }
      }{T}
    }

    \inferrule{
      \typing{e}{\TFun {\TDual S} \TUnit}
    }{
      \typing{\EFork e}{S}
    }

    \inferrule{
      \typing{e}{S[\TRec[X] S / X]}
    }{
      \typing{\ERoll e}{\TRec S}
    }
  
    \inferrule{
      \typing{e}{\TRec S}
    }{
      \typing{\EUnroll e}{S[\TRec[X] S / X]}
    }
%
%   \inferrule{
%     \typing e T \\ f : T \eqt U : f'
%   }{
%     \typing e U
%   }
  \end{mathpar}
%   \label{fig:iso-typing-rules}
%   \caption{Typing rules for expressions in the iso-recursive system}
% \end{figure}


% Typing of expressions in the iso-recursive system is largely the same as for
% expressions in the equi-recursive system. As the only change, two additional
% typing rules are introduced concerned with $\EkwRoll$ and $\EkwUnroll$.
% \begin{mathpar}
%   \inferrule{
%     \typing{e}{S[\TRec[X] S / X]}
%   }{
%     \typing{\ERoll e}{\TRec S}
%   }
% 
%   \inferrule{
%     \typing{e}{\TRec S}
%   }{
%     \typing{\EUnroll e}{S[\TRec[X] S / X]}
%   }
% \end{mathpar}

% Evaluation contexts for the iso-recursive system have the same shape as the
% contexts in the equi-recursive system. The reduction relation is extended by
% two rules concerned with $\EkwRoll$ and $\EkwUnroll$.
% \begin{mathpar}
%   \reducerule{
%     \PScope (\PPar {E[\ERoll a]} {p})
%   }{
%     \PScope (\PPar {E[a]} {p})
%   }
% 
%   \reducerule{
%     \PScope (\PPar {E[\EUnroll a]} {p})
%   }{
%     \PScope (\PPar {E[a]} {p})
%   }
% \end{mathpar}
% Dual $\PScope[b,a]$ rules omitted.

%%% Local Variables:
%%% mode: latex
%%% TeX-master: "main"
%%% End:
